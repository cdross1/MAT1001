\chapter{Tutorial Sheets}
\label{sec: tutorial sheets}

Here we collect all of the tutorial problems for the module. They are split into different weeks depending on the topic they relate to and when they were given out.\\

Many of these questions are taken from or adapted from the recommended  books for the module or from some of the linked resources. These problems are to be attempted in the tutorial sessions and are there to help you familiarise yourself with the material that we have covered in the lectures.\\

Problems marked with a star, $(\star)$ are particularly worth attempting. Problems marked with a dagger, $(\dagger)$, are more challenging and often go beyond what we directly discussed in the lectures.\\

The challenge problem sections contain extra problems. Some of them are just there for extra practice, but others are significantly more difficult than what you need to be able to solve to pass the module. If you are finding the content too easy then have a go at the challenge problems. Sometime the challenge problems from one week will be quite similar to the ordinary problems of the next week. as the problems will become more accessible the more material that we cover.

\section{Week 1}
\label{sec: Tutorial sheet 1}
\paragraph{Functions}

\begin{problem}[$\star$]
Find the roots of the polynomial
\begin{equation*}
g(x)=x^2-2x-12
\end{equation*}
and plot the function.
\end{problem}

\begin{problem}
Find the roots of the polynomial
\begin{equation*}
g(x)=x^3+x^2-x-1
\end{equation*}
and plot the function.
\end{problem}

\begin{problem}[$\star$]
Consider the function
\begin{equation*}
g(x)=x^2+2x+2,
\end{equation*}
produce a plot of the function by calculating its value at a selection of points. What do you notice as $x$ gets very large? What happens for $x=0$?
\end{problem}

\begin{problem}[$\star$]
Consider the function
\begin{equation*}
g(x)=\frac{1}{x-4},
\end{equation*}
and plot the function.what happens as $x$ gets large? What happens as $x$ approaches $4$?\\

Plot the function and comment on its behaviour.
\end{problem}

\begin{problem}
Draw a schematic of a one-to-one function between the sets
\begin{equation*}
\begin{split}
&\left\{1,2,3,4,5\right\}\\
&\left\{a,b,c,d,e,f,g\right\}.
\end{split}
\end{equation*}
Is there only one way to do this?
\end{problem}




\begin{problem}[$\star$]
Given the function
\begin{equation*}
f(x)=x^3-2x^2-x+2
\end{equation*}
find:
\begin{itemize}
    \item $f(0)$,
    \item $f(1)$,
    \item $f(-1)$,
    \item $f(2)$,
    \item $f(-2)$,
    \item $f(t)$,
    \item $f(x-1)$.
\end{itemize}
\end{problem}

\begin{problem}
Consider the function
\begin{equation*}
h(x)=\frac{x}{\sqrt{x^{2}-9}}.
\end{equation*}
Find the points where the denominator vanishes, then plot the function avoiding these points. What happens to the plot as the function approaches these points?
\end{problem}


\begin{problem}
Given the functions
\begin{align*}
f(x)&=x^2-x+1,\\
g(x)&=2-x,
\end{align*}
find:
\begin{itemize}
\item $\left(f\circ g\right)(2)$,
\item $\left(g\circ f\right)(2)$,
\item $\left(f\circ g\right)(x)$,
\item $\left(g\circ f\right)(x)$.
\end{itemize}
\end{problem}


\begin{problem}[$\star$]
Given the functions
\begin{align*}
f(x)&=3x-2\\
g(x)&=\frac{x}{3}+\frac{2}{3},
\end{align*}
find:
\begin{itemize}
\item $\left(f\circ g\right)(x)$,
\item $\left(g\circ f\right)(x)$,
\item What is the relationship between $f(x)$ and $g(x)$?
\end{itemize}
\end{problem}


\begin{problem}[$\dagger$]
Given the function
\begin{equation*}
h(x)=\frac{x+4}{2x-5},
\end{equation*}
identify when it has an inverse and calculate the inverse.
\end{problem}

\paragraph{Challenge Problems}

\begin{problem}
Consider the function 
\begin{equation*}
f(x)=\frac{x^{2}-x-12}{x-1}.
\end{equation*}
Identify the points where the numerator and denominator vanish. Plot the function and explain what happens to the function near these points.
\end{problem}

\begin{problem}
Given the function 
\begin{equation*}
f(x)=2x-3,
\end{equation*}
find the inverse function $f^{-1}(x)$.
\end{problem}


\begin{problem}[$\dagger\dagger$]
Build a schematic of a bijection between the natural numbers
\begin{equation*}
\N=\{1,2,3,4,5,\dots\},
\end{equation*}
and the integers
\begin{equation*}
\Z=\{0,1,-1,2,-2,3,-3,\dots\}.
\end{equation*}

Are there more integers than natural numbers?\\

Could you do the same for the real numbers $\R$? If you find this interesting you may want to explore the work of Cantor.
\end{problem}




\section{Week 2}
\label{sec: Tutorial sheet 2}

\paragraph{Polynomials}
\begin{problem}[$\star$]
For the polynomial equation
\begin{equation*}
x^{3}-3x+2=0,
\end{equation*}
express it as a product of its factors.\\

Hint: this means write it as 
\begin{equation*}
\left(x-a\right)\left(x-b\right)\left(x-c\right),
\end{equation*}
where $a,b,c$ are the roots of the polynomial.
\end{problem}

\begin{problem}
Find the roots of the following polynomial equation
\begin{equation*}
x^{4}-4x^{3}+6x^{2}-4x+1=0.
\end{equation*}
\end{problem}

\paragraph{Trig functions}
Remember to work in radians for any problems related to trigonometry.

\begin{problem}
Find the solutions to 
\begin{equation*}
\sqrt{2}\cos x =1.
\end{equation*}
\end{problem}

\begin{problem}
Identify any solutions to the equation
\begin{equation*}
\sin(2x)=-2.
\end{equation*}
\end{problem}

\begin{problem}[$\star$]
Solve
\begin{equation*}
2x\sin x = x.
\end{equation*}
\end{problem}

\paragraph{Exponentials and Logarithms}
\begin{problem}[$\star$]
Solve the equation
\begin{equation*}
x=xe^{4x}.
\end{equation*}
\end{problem}

\begin{problem}[$\star$]
Solve
\begin{equation*}
2\ln x -\ln(x+2)=1.
\end{equation*}
\end{problem}

\paragraph{Limits and Asymptotes}
\begin{problem}[$\star$]
For the function
\begin{equation*}
f(x)=\frac{x^{2}+4x+12}{x^{2}-2x}
\end{equation*}
evaluate the limit
\begin{equation*}
\lim_{x\to 2}f(x).
\end{equation*}
\end{problem}

\begin{problem}[$\star$]
For the function
\begin{equation*}
g(x)=\frac{2x^{4}-x^{2}+8x}{7-5x^{4}}
\end{equation*}
evaluate the limits
\begin{align*}
&\lim_{x\to -\infty}g(x),\\
&\lim_{x\to \infty}g(x).
\end{align*}
\end{problem}

\begin{problem}[$\dagger$]
For the function
\begin{equation*}
h(x)=\frac{6e^{4x}-e^{-2x}}{8e^{4x}-2e^{2x}+3e^{-x}}
\end{equation*}
evaluate the limit
\begin{align*}
&\lim_{x\to \infty}h(x).
\end{align*}
\end{problem}

\begin{problem}
Using an appropriate plot evaluate the limits
\begin{align*}
&\lim_{x\to \frac{\pi}{2}}\tan x,\\
&\lim_{x\to 0}\tan x.
\end{align*}
\end{problem}

\begin{problem}[$\star$]
Determine where the function
\begin{equation*}
g(x)=\frac{x^{3}+x^{2}-x-1}{x^{3}-2x^{2}-4x+8}
\end{equation*}
fails to be continuous. What do these points correspond to on a plot of $g(x)$?
\end{problem}

\paragraph{Challenge Problems}
\begin{problem}
By making a table of values of $f(x), x$ estimate the limit
\begin{equation*}
\lim_{x\to 2}f(x)
\end{equation*}
from above and below for the function
\begin{equation*}
f(x)=\frac{x^{2}+4x-12}{x^{2}-2x}.
\end{equation*}
\end{problem}

\begin{problem}
Use the definition of continuity to determine in the function
\begin{equation*}
g(x)=\frac{4x+10}{x^{2}-2x-15}
\end{equation*}
is continuous and if not find the points where it has discontinuities.
\end{problem}


\begin{problem}[$\dagger$]
For the function $g(x)$ in the previous problem look at the definition of differentiability and work out where $g(x)$ fails to be differentiable.
\end{problem}

\paragraph{Hyperbolic Trig functions}
\begin{problem}[$\dagger$]
Show that
\begin{equation*}
\sinh(x+y)=\sinh x \cosh y+\cosh x \sinh y.
\end{equation*}
\end{problem}

\begin{problem}[$\dagger$]
Following some of the examples in \cref{sec: hyperbolic functions} show that 
\begin{equation*}
\cosh^{-1}x=\ln\left(x\pm\sqrt{x^{2}+1}\right).
\end{equation*}
\end{problem}

\section{Week 3}
\label{sec: Tutorial sheet 3}
\paragraph{Differentiation}

\begin{problem}[$\star$]
Find the derivative of 
\begin{equation*}
f(x)=7x^{2}
\end{equation*}
\begin{itemize}
\item[a)] using first principles,
\item[b)] using the rule for differentiating monomials.
\end{itemize}
\end{problem}

\begin{problem}
Find the derivative of 
\begin{equation*}
f(x)=3x^{3}-8x.
\end{equation*}
\end{problem}



\paragraph{Challenge Problems}



