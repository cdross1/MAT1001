\chapter{Further Reading}
\label{sec:further reading}

\epigraph{Mathematics, you see, is not a spectator sport. To understand mathematics means to be able to do mathematics. And what does it mean doing mathematics? In the first place, it means to be able to solve mathematical problems. }{\textit{How to Solve It by George P\'{o}lya}}

What we discuss in the lectures is just a guide to calculus its many uses. Depending on the modules that you take later on in your degree you will meet Calculus in different guises, particularly when you come across optimisation problems . In this section I will provide links to further resources and suggestions for further reading based on the material that we met each week. If you have any questions about the topics linked here or you want to get even more information then drop me an \href{mailto:rossc@edgehill.ac.uk}{email}.\\

It is important to remember that the lectures are their to introduce you to topics and signpost where to find out more information. If you are just attending the lectures and not doing any further reading or solving practice problems then you will struggle to pass the course. Each 20 credit module is considered 200 hours of work, only 36 hours of which are the lectures and seminars. You are expected to put in around 164 hours of self study during a 12 week module. The resources linked here will help with this self study.\\

\section{Why Calculus Extra Reading}
MIT have a course called \textbf{Calculus for Beginners and Artists} which contains some overlap with this module.  The webpage for the course is available \href{https://math.mit.edu/~djk/calculus_beginners/index.html}{here} and some sections of it, particularly \textbf{Chapter 0: Why Study Calculus?} are worth a read to complement what we will do in this module. I will occasionally suggest reading sections of these notes or sections of \citep{calcI} for a complementary explanation. 



\section{Functions Extra Reading}


\section{Differentiation Extra Reading}
The main reference for this section and many of the other sections of these lecture notes is the wonderful book \textbf{Mathematical Methods for Physics and Engineering}, \citep{riley_mathematical_2006}. There are several copies of this book and its student solution manual in the library and I recommend that you try and have a look at this at some stage.\\

\section{Integration Extra Reading}

\section{Differential Equations Extra Reading}

\section{Numerical Methods Extra Reading}

\section{Advanced Topics Extra Reading}