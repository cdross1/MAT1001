\chapter{Extra Proofs and Derivations}
\label{sec: proofs}
The material in this chapter is not examinable and is included for completeness. Here we will give a selection of proof and derivations for results and formulas that were used in the rest of the notes. You could consider this whole chapter to be one big mathematical deviation.

To prove the product rule from first principles we proceed as follows. Consider $f(x)=p(x)q(x)$ then
\begin{align*}
\frac{\ud f}{\ud x}		&=\lim_{h\to 0}\frac{f(x+h)-f(x)}{h}\\
				&=\lim_{h\to 0}\frac{p(x+h)q(x+h)-p(x)q(x)}{h}\\
				&=\lim_{h\to 0}\frac{p(x+h)q(x+h)-p(x+h)q(x)+p(x+h)q(x)-p(x)q(x)}{h}\\
				&=\lim_{h\to0}\frac{p(x+h)\left(q(x+h)-g(x)\right)}{h}+\lim_{h\to 0}\frac{\left(p(x+h)-p(x)\right)q(x)}{h}\\
				&=\lim_{h\to 0}p(x+h)\frac{q(x+h)-q(x)}{h}+q(x)\lim_{h\to 0}\frac{p(x+h)-p(x)}{h}\\
				&=p(x)\lim_{h\to 0}\frac{q(x+h)-q(x)}{h}+q(x)\lim_{h\to 0}\frac{p(x+h)-p(x)}{h}\\
				&=p(x)\frac{\ud q}{\ud x}+q(x)\frac{\ud p}{\ud x},
\end{align*}
which is the product rule.\\

For the quotient rule, we can either prove it from first principles, which is fiddly, or we can use the product rule on $f(x)=p(x)g(x)$ where $g(x)=1/q(x)$.  Here we will take the second approach as hopefully that will be easier to follow.  Consider $f(x)=p(x)/q(x)$ and let $g(x)=1/q(x)$ then applying the product rule means that
\begin{align*}
\frac{\ud f}{\ud x}		&=\frac{\ud }{\ud x}\left(p(x)g(x)\right)\\
				&=p(x)\frac{\ud g}{\ud x}+g(x)\frac{\ud p}{\ud x}\\
				&=p(x)\frac{\ud}{\ud x}\left(q^{-1}\right)+\frac{1}{q(x)}\frac{\ud p}{\ud x}\\
				&=-p(x)q^{-2}\frac{\ud q}{\ud x}+\frac{1}{q(x)}\frac{\ud p}{\ud x}\\
				&=\frac{1}{q^{2}}\left(q(x)\frac{\ud p}{\ud x}-p(x)\frac{\ud q}{\ud x}\right).
\end{align*}

\begin{mdiv}
Note that in the above calculation we have used the chain rule, $\left((f\circ g)(x)\right)'=f'(g(x))g'(x)$, which we did not discuss until after we had introduced the quotient rule.  Also, note that if we were mathematicians we would need to carefully think about when $p(x)/q(x)$ is differentiable, and implementing the product rule does not need that, it just gives that $p(x)(q(x))^{-1}$ is differentiable if $p(x)$ and $(q(x))^{-1}$ is.  This is why for mathematicians, they would prove the quotient rule using differentiation from first principles, or using logarithmic differentiation. The interested reader should have a look at the Proof of various derivative properties section of \citep{calcI} to see the proof in this way.
\end{mdiv}

Next, following \citep{calcI}, we prove the chain rule as follows: let $y=f(u), u=g(x)$ then we know that
\begin{equation*}
\frac{\ud u}{\ud x}=\lim_{h\to 0}\frac{u(x+h)-u(x)}{h},
\end{equation*}
and that 
\begin{equation*}
\lim_{h\to 0}\left(\frac{u(x+h)-u(x)}{h}-\frac{\ud u}{\ud x}\right)=\lim_{h\to 0}\frac{u(x+h)-u(x)}{h}-\lim_{h\to 0}\frac{\ud u}{\ud x}=\frac{\ud u}{\ud x}-\frac{\ud u}{\ud x}=0.,
\end{equation*}
Now we can define
\begin{equation*}
v(h)=\begin{cases}
\frac{u(x+h)-u(x)}{h}-\frac{\ud u}{\ud x} \quad \text{if } h\neq 0\\
0 \quad \text{if } h=0
\end{cases}
\end{equation*}
which is continuous at $h=0$ since $\lim_{h\to 0}v(h)=0=v(0)$. \dots{}
