% Save this as tutorial.tex for the lwarp package tutorial.
\documentclass{book}
\usepackage{iftex}
% --- LOAD FONT SELECTION AND ENCODING BEFORE LOADING LWARP ---
\ifPDFTeX
\usepackage{lmodern} % pdflatex or dvi latex
\usepackage[T1]{fontenc}
\usepackage[utf8]{inputenc}
\else
\usepackage{fontspec} % XeLaTeX or LuaLaTeX
\fi
% --- LWARP IS LOADED NEXT ---
\usepackage[
HomeHTMLFilename=index, % Filename of the homepage.
% HTMLFilename={node-}, % Filename prefix of other pages.
% IndexLanguage=english, % Language for xindy index, glossary.
% latexmk, % Use latexmk to compile.
% OSWindows, % Force Windows. (Usually automatic.)
 mathjax, % Use MathJax to display math.
%mathsvg% Show math using SVG images.
GlossaryCmd={perl makeglossaries},
GlossaryCmd={makeglossaries -L english},
]{lwarp}
% \boolfalse{FileSectionNames} % If false, numbers the files.
% --- LOAD PDFLATEX MATH FONTS HERE ---
% --- OTHER PACKAGES ARE LOADED AFTER LWARP ---
\usepackage{calum_lecture} % My style file contains all of the packages, page, geometries, and custom commands that I like to use.
\makeglossaries
\usepackage{mat1001_glos}
\glstoctrue %Adds glossary to the table of contents

% --- LATEX AND HTML CUSTOMIZATION ---
%\title{MAT1001 Differential Calculus: Lecture Notes}
%\author{Calum Ross}
\setcounter{tocdepth}{2} % Include subsections in the \TOC.
\setcounter{secnumdepth}{2} % Number down to subsections.
\setcounter{FileDepth}{0} % Split \HTML\ files at sections
\booltrue{CombineHigherDepths} % Combine parts/chapters/sections
\setcounter{SideTOCDepth}{1} % Include subsections in the side\TOC
\HTMLTitle{MAT1001 Differential Calculus: Lecture Notes} % Overrides \title for the web page.
\HTMLAuthor{Calum Ross} % Sets the HTML meta author tag.
\HTMLLanguage{en-UK} % Sets the HTML meta language.
\HTMLDescription{Lecture notes for the module MAT1001 Differential Calculus.}% Sets the HTML meta description.
%\HTMLFirstPageTop{Name and \fbox{HOMEPAGE LOGO}}
%\HTMLPageTop{\fbox{LOGO}}
\HTMLPageBottom{Contact \href{mailto:rossc@edgehill.ac.uk}{rossc[at]edgehill[dot]ac[dot]uk} and Copyright CC BY 4.0}
%\CSSFilename{STM0005-physics.css} % This loads a css file that enables us to customise the webpage more. Currently I have not written a css file to be included here.

\MathJaxFilename{lwarp_mathjax.txt}

%%% Compact lists slightly less compact
%%\setlength{\plitemsep}{3pt}
\begin{document}
\pagenumbering{Alph} % to stop hyperref warnings
%\frontmatter
\title{MAT1001 Differential Calculus: Lecture Notes}
\author{Calum Ross}
\affiliation{Edge Hill University}
\date{Last modified: \today}
\maketitle % Or titlepage/titlingpage environment.
% An article abstract would go here.
\pagenumbering{roman}
\tableofcontents % MUST BE BEFORE THE FIRST SECTION BREAK!
%\listoffigures

\mainmatter

\chapter{Why Do We Need Calculus}
\label{sec:course intro}
%\pagenumbering{arabic}

\epigraph{\textbf{Calculus}: A branch of mathematics using the idea of a limit and generally divided into two parts: integral and differential calculus. }{\textit{Penguin Dictionary of Mathematics}}


\section{Course Overview}
\label{sec:overview}
These lecture notes cover the same material as the lectures for \textbf{MAT1001 Differential Calculus}. They will be updated during the module so please check back regularly for the latest version, the date of the most recent update is shown on the front page. In particular, most of the figures used in the lectures are hand drawn and it takes some time to produce good quality digital versions of these.\\

It is important to note that \textbf{reading these notes is not a substitute for attending the lectures and tutorials!} While I have included everything that I intend to cover in the lectures, and have often given more detail in these notes, the direction that we go in lectures will be guided by student questions and the example problems that you are given to solve. These are not reflected in the notes as I cannot anticipate all of the questions that may be asked. If you want to pass the module then the best way to do this is to attend the lectures and tutorials and to solve all of the practice problems that you are given.\\

In week one there was  overview information about the module including a rough schedule, how you will be assessed, and how to contact me. That will not be duplicated here and can be found on the module's virtual learning environment (VLE) page. If you need have any questions during the module then you can send me an \href{mailto:rossc@edgehill.ac.uk}{email} or come by my office during the office hour times. The module reading list, available on the VLE, contains details of recommended and supplementary reading for the module, there will frequently be suggested reading given to complement the material from a given weeks lecture. In these notes I will occasionally cite one of these books in the lecture notes, there will also be extra information and resources given in \cref{sec:further reading}. My favourite reference for much of this material is \citep{riley_mathematical_2006}, which covers all the areas of mathematics that we need for this course.\\

We will mainly be using content from chapters 1, 2, and parts of 27 from \citep{riley_mathematical_2006}.  Another good resource is the calculus page on the website \href{https://tutorial.math.lamar.edu/classes/calci/calci.aspx}{Paul's Online Notes}. There are lots of examples given there, some of which I have included in this lecture notes.\\

Another book that I have sometimes used is \citep{jordan2008mathematical}, it contains lots of background material but does not always have the simplest explanations. For numerical methods \citep{lissamen2004mei} contains a lot of material, and there are several copies in the library.\\


\section{Why Calculus}
\label{sec:why calculus}
\epigraph{\textbf{So what does calculus add for me?} It provides a way for us to construct relatively simple quantitative models of change, and to deduce their consequences. }{\textit{\href{https://math.mit.edu/~djk/calculus_beginners/chapter00/section02.html}{MIT Calculus for Beginners}}}

Calculus is an important topic in mathematics and at its most basic is the study of how functions change. It makes an appearance whenever we are interested in solving an optimisation problem, e.g. finding the best solution to a given problem. This could be finding out the largest area that can be enclosed by a fence of a given length, or what the fastest way to travel between two different points is. One example of this is sometimes called the \textbf{Ocado} or \textbf{vehicle routing problem} and wants to find the optimal route for a delivery van\footnote{More info on this can be found \href{https://www.cardiff.ac.uk/news/view/2585000-improving-vehicle-routing-with-ocado-group}{here}. Sometimes the Ocado problem refers to routing of the robots that pick up groceries in the warehouse which is another example of a vehicle routing problem.}. \\

\begin{figure}[ht]
    \centering
    %\pdftooltip{
    \includegraphics[width=0.35\textwidth, alt ={A schematic of the vehicle routing problem.}]{figures/vehicle_routing_problem}
    %}{. }
    \caption{A schematic of the vehicle routing problem. This is a programming problem where you want to find the optimal set of routes for a fleet of vehicles to travel when making deliveries to a given set of customers, the dots in the figure. This sort of optimisation problem is important for any delivery service as making their routes more efficient can save them money. }
\label{fig: vehicle routing}
\end{figure}

You may be thinking, ``that all sounds sensible in mathematics or physics, but as a computer science why should I care about calculus?'' If so you are not alone, many people ask the same question and there are many answers given online. For example the Medium post \href{https://medium.com/@18bhavyasharma/the-use-of-calculus-in-computer-science-a6917dbe33b9}{The uses of Calculus in Computer Science} highlights several areas:
\begin{itemize}
%\setlength{\itemsep}{-5pt}
    \item \textbf{Algorithms and optimisation:}  As highlighted above, whenever you have an optimisation problem calculus invariably rears its head.
    \item \textbf{Numerical analysis/ Computer graphics and computer vision:} Calculus provides a language for solving complex physical and mathematical problems that we are unable to solve by hand. This is important if you want to simulate a physical system, whether in a piece of scientific research or because you are designing a computer game, but also in graphics rendering and computer vision where you need to understand and simulate the path that light rays will follow.
   \item \textbf{AI and machine learning:} Most machine learning algorithms involve calculus in their optimisation and training.
   \item \textbf{Data science:} Much of data science involves statistical and mathematical modelling to analyse and understand the data. Understanding these techniques relies on understanding calculus. 
\end{itemize}
This all goes to show that calculus is an essential tool in many field beyond pure mathematics and is indispensable for Physicists, Engineers, and computer programmers. \\

\begin{figure}[ht]
    \centering
    %\pdftooltip{
    \includegraphics[width=0.6\textwidth, alt = {A schematic of ray tracing.}]{figures/Ray_trace_diagram}
    %}{A schematic of ray tracing. }
    \caption{A schematic showing how a ray tracung algorithm works to build up an image by extending light rays. This image was made by \href{https://commons.wikimedia.org/wiki/File:Ray\_trace\_diagram.svg}{Henrik} for Wikimedia Commons.}
\label{fig: ray tracing}
\end{figure}


Calculus is separated into two main areas: \textbf{\gls{Differentiation}} and \textbf{\gls{Integration}}. Differentiation is the study of how quickly or slowly a function varies. If we have a a graph of a function the derivative is the gradient to this graph, this notion will be made precise in a later section. Integration is the ``opposite'' of differentiation and corresponds to finding the area under the graph of a function.\\

Both concepts can be extended to much more abstract settings but the intuition gained from thinking about graphs of functions, their gradients, and the area under them, will set you in good stead for everything that comes after this.\\

\begin{figure}[ht]
    \centering
    %\pdftooltip{
    \includegraphics[width=0.3\textwidth, alt ={A schematic of the derivative as a tangent to a curve.}]{figures/Tangent_to_a_curve}
    %}{A schematic of a derivative as a tangent to a curve. }
    \caption{The derivative of a function at a point is equivalent to finding the tangent to a curve at this point.  }
\label{fig: derivative as tangent}
\end{figure}

\begin{figure}[ht]
    \centering
   % \pdftooltip{
   \includegraphics[width=0.4\textwidth, alt ={A schematic of an integral as the area under the curve.}]{figures/Integral_approximations}
   %}{A schematic of an integral as the area under the curve. }
    \caption{The integral, or area under the curve, can be approximated by rectangles. Depending on the size of the rectangles this can be an over estimate or an underestimate, but it will improve the thinner the rectangles become. This figure comes from \href{https://commons.wikimedia.org/wiki/File:Integral\_approximations\_J.svg}{Wikimedia Commons}. }
\label{fig: integral approximation}
\end{figure}
The YouTube channel 3Blue1Brown has an excellent playlist called the essence of calculus available \href{https://www.youtube.com/playlist?list=PLZHQObOWTQDMsr9K-rj53DwVRMYO3t5Yr}{here} which goes through the basics of calculus and its implications in a fairly unique way. Grant's goal in the videos is to give you an intuition for why Calculus works the way it does, rather than just giving you a list of facts and equations to memorise. Hopefully you will find that this module takes a similar approach, the goal is for you to understand the how and why of the mathematics and not just approach it as a set of rules to memorise. \\

If you find that you enjoy any of the topics in this module, or just want to read about interesting mathematics in a popular science setting I recommend having a look at \href{https://chalkdustmagazine.com/}{\textit{\gls{Chalkdust}} magazine}. As one of the editors of \textit{Chalkdust} I may be biased, but I think that it is a nice way to find out about some interesting maths without having to go into all of the details.\\

Hopefully, this has given you a feel for why you are doing this module. Now we can get started with the mathematics. As calculus is the study of how functions change we first need to remind ourselves of some of the properties of functions and introduce some notation.

\section*{How to use these notes }
As stated above, these notes are here to complement the in person lectures rather than to replace them.  For some topics there will be more detail given here, while at other points there may be explanations that I give in the lectures, or examples that I use which do not appear in these notes.  Also, I will frequently draw pictures in the lectures, as the module goes on I will try to add these figures to the lecture notes. However, that will not always be possible. These notes also contain exercises, some of which will appear in the tutorial sheets, that you should solve to help reinforce the concepts from the module.\\

To complement the theory presented in these notes you will find a range to examples and exercises to help you build familiarity with the mathematical concepts that we will meet during this module. As this is not a module for mathematicians we will not always completely precise and give all the details, but rather focus on the practical applications of the mathematics. Occasionally I will include a \textbf{\gls{Mathematical Diversion}} where I give some more of these details. If you are interested in the mathematics for its own sake then these will hopefully be interesting to you. If you are only interested in the mathematics that you need to pass this module, then feel free to skip these diversions.\\

\begin{warpprint} % For print only output ...
These lecture notes will continue to develop as the module goes on so make sure to check back frequently to see what has been added. There is also an online html version of these notes that you can check out as well. The html version should be accessible with alttext on the figures. If you have any difficulty viewing either these notes or the html version let me know.
\end{warpprint}


\newpage

%%%%%%%%%%%%%%%%%%%%%%%%%%%%%%%%%%%%%%%%%%%%%%

\chapter{Functions}
\label{sec:functions}

\epigraph{The art of doing mathematics is finding that special case that contains all the germs of generality. }{\textit{David Hilbert}}

\section{What is a function}
The concept of a function is essential to understand not just calculus but also computer programming. We can think of a function as being a black box that takes in information, potentially changes it in some way, and then outputs information.\\

\begin{figure}[ht]
    \centering
    %\pdftooltip{
    \includegraphics[width=0.3\textwidth, alt ={A schematic of how a function behaves produced by Bin Im Garten for Wikimedia Commons.}]{figures/Injection_keine_Injektion_2a.png}
    %}{A schematic of how a function behaves produced by \href{Bin Im Garten} for Wikimedia Commons. }
    \caption{A schematic of how a function behaves, it maps objects $1,2,3,\dots$ in the \gls{set} $X$ to objects $A,B,C,D,\dots$ in the set $Y$. This image was produced by \href{https://commons.wikimedia.org/wiki/File:Injection\_keine\_Injektion\_2a.svg}{Bin Im Garten} for Wikimedia Commons.}
\label{fig: function schematic}
\end{figure}

Being more precise, a function is a map between two spaces, the domain, and the codomain. e.g.
\begin{equation*}
f:X\to Y
\end{equation*}
and it sends a point $x\in X$ to a point $f(x)=y\in Y$.  Functions have the property that they map a point $x\in X$ to a single point $y\in Y$,for example:
\begin{equation*}
y=f(x)=x^{2}+1,
\end{equation*}
is a function since every value of $x$ gives a single value of $y$. However, 
\begin{equation*}
y^{2}=x+1,
\end{equation*}
is not a function since every $x$ corresponds to two values of $y$, e.g. if $x=3$ then $y^{2}=3+1=4$ and $y$ can be both $2$ and $-2$.\\

Note that when we write $f(x)$ we are not saying $f$ times $x$, but mean that $f$ is a function of $x$, e.g. $f$ takes in a value of $x$ and returns a number $y=f(x)$.\\

Throughout this module we will meet several different types of functions so you will need to get familiar with this notation and understand how to evaluate functions. There will be some questions in the tutorial sheet to help you practice this.\\

A key property of a function is its \textbf{\gls{roots}} or zeros, these are the values of $x$ such that $f(x)=0$. For linear functions finding the roots is just a matter of rearranging the equation, sometimes referred to as changing the subject. In this module I am assuming that you have some familiarity with this, if not take a look at the background material in \cref{sec:background} or ask me to point you towards more resources. There will also be some revision questions on this topic in the first couple of weeks tutorials.

\begin{ex} Find all the roots of $f(x)=9x^{3}-18x^{2}+6x$.\\

Remember the roots are where $f(x)=0$ so we are solving $9x^{3}-18x^{2}+6x=0$. First notice that there is a common factor of $3x$ in all of the terms so we can factor that out and get
\begin{equation*}
0=9x^{3}-18x^{2}+6x=3x\left(3x^{2}-6x+2\right).
\end{equation*}
This means that either $x$ is zero or the quadratic expression in brackets, $3x^{2}-6x+2$, is zero. This means that $x=0$ is a root of the equation, and there will be two more that we find by solving the quadratic. This can be done by using the quadratic formula:
\begin{align*}
x&=\frac{6\pm\sqrt{(-6)^{2}-4(3)(2)}}{2(3)}\\
&=\frac{6\pm\sqrt{12}}{6}\\
&=1\pm\frac{\sqrt{3}}{3}\\
&=1\pm\frac{1}{\sqrt{3}}.
\end{align*}

So the three roots of $f(x)$ are
\begin{equation*}
x=0,\quad x=1+\frac{\sqrt{3}}{3},\quad x=1-\frac{\sqrt{3}}{3}.
\end{equation*}

\end{ex}

If you do not remember how to use the quadratic formula then I suggest that you look at the background material in \cref{sec:background}.\\

If this was a course for mathematicians this is where we would spend time talking about the domain and range of a function. How they are defined, and when we need to be careful to avoid dividing by zero. Here, we will not go through this, and I will just remind you that in your algebraic manipulations you should not divide by any quantity that is zero.\\

A simple, yet useful, class of functions are the rational functions. There are functions who are the ratio of two polynomials, e.g. 
\begin{equation*}
f(x)=\frac{4x+10}{x^{2}-2x-15}.
\end{equation*}

A useful way to understand a function is draw a graph. In a graph, the domain of the function $X$ is drawn horizontally and the codomain, $Y$, is drawn vertically, using \textbf{\gls{Cartesian} axes} $(x,y)$. The graph consists of the points $(x,y)$ with $y=f(x)$. In many books $x$ is called the independent variable and $y$ the dependent variable. In the tutorials we will discuss how to produce plots using a computer, either using \textbf{MATLAB}, \textbf{Python}, or \href{https://www.wolframalpha.com/}{\textbf{WolframAlpha}}. There are also online graphing programs like \href{https://www.desmos.com/}{\textbf{desmos}} and \href{https://www.geogebra.org/}{\textbf{GeoGebra}}. In these notes most of the graphs are produced using the Tikz package for \LaTeX{}, which is a typesetting and markup language\footnote{If you want to produce professional looking documents which include mathematical formulas then it is worth your time learning how to use \LaTeX{}.}.

\begin{ex}
Plot the graph of $f(x)=2x^{2}-1$.\\
\begin{figure}[htbp]
    \begin{center}
\ThisAltText{Graph of a quadratic.}
   % \pdftooltip{
    \begin{tikzpicture}[line width=1pt,line cap=round,line join=round,domain=-1:1, smooth,variable=\x,scale=2]
     \draw[->] (-1.2,0) -- (1.2,0) node[above] {$x$};
  \draw[->] (0,-1.2) -- (0,1.2) node[above] {$y$};
 \draw[color=red]   plot (\x,{2*\x*\x -1}) node[right] {$f(x)=2x^{2}-1$};
    \end{tikzpicture}
%    }{graph of a quadratic}
    \caption{The graph of the quadratic function  $f(x)=2x^{2}-1$ is a parabola which intersects the $y$-axis at $x=0$ and the $x$-axis at $x=\pm\frac{1}{\sqrt{2}}$.}
        \label{fig: quadratic function graph}
\end{center}
\end{figure}
\end{ex}
Note that the equation $y^{2}=x+1$ also defines a parabola, see \cref{fig: non equation}. However if we plot this function the curve is not the graph of a function. You should think about why this is.

\begin{figure}[ht]
    \centering
    %\pdftooltip{
    \includegraphics[width=0.4\textwidth, alt={An equation which does not give a graph.}]{figures/desmos-graph_non-function}
    %}{An equation which does not give a graph. }
    \caption{A plot of $y^{2}=x+1$, you should think about why this is not the graph of a function. }
\label{fig: non equation}
\end{figure}

\begin{ex}The function $f(x)=\sin(x)$ has the graph shown in \cref{fig: sine function graph}.\\
\begin{figure}[htbp]
    \centering
\ThisAltText{graph of sin(x).}
  %  \pdftooltip{
    \begin{tikzpicture}[line width=1pt,line cap=round,line join=round,domain=-6.282:6.282, smooth,variable=\x]
     \draw[->] (-6.282,0) -- (6.282,0)node[above] {$x$};
  \draw[->] (0,-1.2) -- (0,1.2) node[above] {$y$};
 \draw[color=CDred]   plot (\x,{sin(\x r)}) node[right] {$f(x)=\sin(x)$};
    \end{tikzpicture}
  %  }{graph of sin}
    \caption{The graph of the sine function  $f(x)=\sin(x)$.}
        \label{fig: sine function graph}
\end{figure}
\end{ex}

\begin{ex}
Consider the two functions $f(x)=3x-2$ and $g(x)=x/3 +2/3$. These satisfy the relationship
\begin{align*}
\left(f\circ g\right)(x)	&=f\left(g(x)\right)\\
				&=f\left(\frac{x}{3}+\frac{2}{3}\right)\\
				&=3\left(\frac{x}{3}+\frac{2}{3}\right)-2\\
				&=x+2-2=x
\end{align*}
and 
\begin{align*}
\left(g\circ f\right)(x)=x.
\end{align*}

When we have two functions whose composition leaves $x$ unchanged, we say that they are \textbf{\gls{inverse}} to each other, and $g$ is the inverse\footnote{Usually the inverse function is denoted as $f^{-1}$, though we need to be careful to understand that this does not mean $1/f$.} of $f$. As can be seen in Section 1.2 of \citep{calcI} we can understand this a meaning that if $f:x\mapsto y$ then $g:y\mapsto x$, e.g. $f(-1)=-5$ while $g(-5)=-1$.
\end{ex}

The concept of an inverse function makes intuitive sense. However, if we pretend to be mathematicians and treat this carefully we quickly encounter some problems. For example, what happens if we have a function which sends two different values of $x$ to the same value? Then we are unable to know which value we started with, so cannot build an inverse function. For example $f(x)=x^{2}$ sends both $x$ and $-x$ to $x^{2}$ so we cannot find an inverse that works everywhere, this is why when we firts meet the square root function you often see it as $\pm\sqrt{\phantom{+}}$.\\

Mathematicians fix this by introducing the notion of a \textbf{\gls{one-to-one}} function, sometimes called an \textbf{injective} function. A function is called one-to-one if no two values of $x$ produce the same value of $y$, so that
\begin{equation*}
f(x_{1})\neq f(x_{2}) \qquad \text{ for } x_{1}\neq x_{2}.
\end{equation*}

The advantage of one-to-one functions is that we can find inverses for them. If we have two one-to-one functions which satisfy 
\begin{equation*}
\left(f\circ g\right)(x)=x=\left(g\circ f\right)(x),
\end{equation*}
then $f$ and $g$ are inverses and we write $g(x)=f^{-1}(x)$.

\section{Trigonometric functions}
\label{sec: trig func}

Now that we have discussed functions and how to graph them, we can focus on some specific functions which it is very common to encounter.\\ 

From your previous maths experience you have probably come across \textbf{\gls{trigonometric}} (trig) functions in the context of triangles, where they are used to calculate lengths and angles. If you do not remember how this works then I suggest that you have a look at the Triganometry primer in \cref{sec:background} or check out some revision material available \href{https://corbettmaths.com/2013/03/30/trigonometry-introduction/}{here}. Another important thing to remember is that you should always work in radians rather than degrees when using trig functions. This is because radians are a more natural unit for angles and if we did not use them lots of formulas would need extra factors of $\uppi$ to be added for them to be valid. \\

\begin{figure}[ht]
    \centering
\ThisAltText{Converting from a circle to trig functions.}
  %  \pdftooltip{
    \begin{tikzpicture}[line width=1pt,line cap=round,line join=round, scale = 1.5]
   % \draw[step=1cm,gray,very thin] (3,-3) grid (3,3);
   \draw[->,ultra thick] (0,-2.4)--(0,2.4);
     \draw[->,ultra thick] (-2.4,0)--(2.4,0);
   \filldraw[color = blue, ultra thick](0,0) circle (0.05);
    \draw[ ultra thick](0,0) circle (2);
    \draw[ultra thick] (0,0) -- (2,0);
    \draw[ultra thick] (1,0) arc (0:45:1) node[anchor=north]{$\ut$};
     %\draw[ultra thick, color=red] (2,0) arc (0:45:2) node[anchor=north]{$s$};
     \draw[color = blue, ultra thick](0,0)--node[anchor=south]{$r$}(1.414,1.414) ;
     \draw[dashed, ultra thick] (0,1.414)node[left]{$\sin\ut$} --(1.414,1.414);
      \draw[dashed, ultra thick] (1.414,0) node[below]{$r\cos\ut$}--(1.414,1.414);
    \end{tikzpicture}
 %   }{Converting from a circle to trig functions}
    \caption{Trig functions and their relationship to a unit circle, this is true whatever the angle $\ut$ is, though we need to be careful about the signs of $x$ and $y$ }
        \label{fig:trig and circles}
\end{figure}


In the setting of triangles the angles are restricted to run between $0$ and $\uppi$ radians as for angles greater than this we would no longer have a triangle. However, the functions are valid for any real values of the argument, $x$. Typically we will be interested in angles within the range $[0,2\uppi)$, but need to remember that, as shown in \cref{fig: trig functions}, these functions are $2\uppi$ periodic. This means that
\begin{align*}
\sin(x+2\uppi)=\sin(x),\\
\cos(x+2\uppi)=\cos(x),\\
\tan(x+2\uppi)=\tan(x).
\end{align*}

We can think of these as functions from $[0,2\uppi)\to [-1,1]$ which reduce to the familar trig functions for angles between $0$ and $\uppi$.\\

\begin{figure}[ht]
    \centering
\ThisAltText{Graph of sin(x) and cos(x).}
   % \pdftooltip{
    \begin{tikzpicture}[line width=1pt,line cap=round,line join=round,domain=-6.28:6.28, smooth,variable=\x]
     \draw[->] (-7,0) -- (7,0) node[above] {$x$};
  \draw[->] (0,-1.3) -- (0,1.3);
 \draw[color=CDnavy]   plot[samples=300] (\x,{cos(\x r)}) node[right] {$\cos(x)$};
  \draw[color=CDgreen, dashed]   plot[samples=300] (\x,{sin(\x r)}) node[anchor = north west] {$\sin(x)$};
    \end{tikzpicture}
 %   }{plots of sin and cos }
    \caption{Plots of the trig functions $\sin(x)$ and $\cos(x)$ for $x$ between $-2\uppi$ and $2\uppi$.}
        \label{fig: trig functions}
\end{figure}


\begin{figure}[ht]
    \centering
  %  \pdftooltip{
  \includegraphics[width=0.45\textwidth, alt ={Graph of a tan(x).}]{figures/tan_graph}
  %}{A plot of the tan function. }
    \caption{A plot of the trig function $\tan(x)$, unlike $\sin$ and $\cos$ above, $\tan$ cannot be drawn without taking your pen off the page.  }
\label{fig: tan function}
\end{figure}

The trig functions are clearly not one-to-one, since they are periodic, in fact even restricted to $x\in[0,2\uppi)$ there are repeated values. However, just like with the square root, it is convenient to introduce inverse functions, where we need to be careful which quadrant our angle is in to get a single value out. These inverse functions are often denoted
\begin{align*}
\sin^{-1}(x)&=\arcsin(x),\\
\cos^{-1}(x)&=\arccos(x),\\
\tan^{-1}(x)&=\arctan(x).
\end{align*}

Your calculator will have a button for each trig function, and a way to select the inverse functions, typically calculators return answers in the following ranges:
\begin{equation*}
0\leq \arccos(x)\leq \uppi,\qquad -\frac{\uppi}{2}\leq \arcsin(x)\leq \frac{\uppi}{2}, \qquad -\frac{\uppi}{2}< \arctan(x)< \frac{\uppi}{2}.
\end{equation*}

The inverse trig functions are useful if we need to solve equations involving trig functions.\\

\begin{ex}
Find the solutions to $4\cos(t)=3$ for $t\in[-8,10]$. \\

The first step is to divide both sides by $4$ to give us
\begin{equation*}
\cos(t)=\frac{4}{3},
\end{equation*}
now we can use
\begin{equation*}
t=\arccos\left(\frac{3}{4}\right)=0.7227.
\end{equation*}

Not that this is just one of an infinite\footnote{``Why infinitely many?'' you ask. This is because $\cos(x)$ is $2\uppi$ periodic so if $t$ is a solution in the range $[0,2\uppi)$ then $t+2\uppi n$ is another solution for $n\in \Z$. } number of solutions to the above equation. in the range from $0$ to $2\uppi$ there are two solutions, $t=0.7227$ and $t=2\uppi-0.7227=5.5605$. We now add $2\uppi n$ to both of our values, testing values of $n$ so that the result stays within the interval :
%\begin{align*}
%&n=-2 \quad t=0.7227-4\uppi = \sout{-11.8437} \quad \text{and} \quad 5.5605-4\uppi=-7.0059\\
%&n=-1 \quad t=0.7227-2\uppi = -5.5605\quad \text{and} \quad 5.5605-2\uppi=-0.7227\\
%&n=0 \quad t=0.7227\quad \text{and} \quad 5.5605\\
%&n=1 \quad t=0.7227+2\uppi = 7.0059 \quad \text{and} \quad 5.5605+2pi=\sout{11.8437}
%\end{align*}

\begin{itemize}
%\setlength{\itemsep}{-5pt}
    \item $n=-2$  then $ t=0.7227-4\uppi =\cancel{-11.8437}$ and $t=5.5605-4\uppi=-7.0059$,
    \item $n=-1$ then $ t=0.7227-2\uppi = -5.5605$ and $t=5.5605-2\uppi=-0.7227$,
    \item $n=0$ then $t=0.7227$ and $t=5.5605$,
    \item $n=1$  then $t=0.7227+2\uppi = 7.0059$ and $ t=5.5605+2\uppi=\cancel{11.8437}$.
\end{itemize}
Thus there are six solutions in the interval $[-8,10]$,
\begin{equation*}
t=-7.0059, -5.5605, -0.7227, 0.7227, 5.5605, 7.0059.
\end{equation*}
Not that the solutions come in positive and negative pairs.
\end{ex}

You will have the opportunity to practice more problems like this either by going through the week one tutorial sheet or by looking at Section 1.5 in \citep{calcI}.\\

The trig functions satisfy some nice relationships that you should try to learn, and which we quote here without any proof. Some of these are fairly easy to prove and are left as an exercise, while others are harder and can be looked up if you are curious.\\

The first one is that 
\begin{equation*}
\tan(x)=\frac{\sin(x)}{\cos(x)}.
\end{equation*}

\paragraph{Squares:} The squares of trig functions have the nice property that
\begin{equation*}
\sin^{2}x+\cos^{2}x =1,
\end{equation*}
which is a consequence of Pythagoras' theorem.

\paragraph{Multiple angles:} There are some very useful identities when we consider trig functions for the sum and difference of an angle:
\begin{align*}
\sin\left(x\pm y\right)&=\sin x \cos y \pm \cos x \sin y,\\
\cos\left(x\pm y\right)&=\cos x \cos y \mp \sin x \sin y.
\end{align*}
For the special case of $x=y$ this leads to 
\begin{align*}
\sin 2x &= 2\sin x \cos x,\\
\cos 2x &=\cos^{2}x-\sin^{2}x = 2\cos^{2}x -1 =1-2\sin^{2}x.
\end{align*}

You should try to become familiar with these identities, at least for this module, as they will be useful wherever trig functions appear.\\

\paragraph{Further trig functions:} There are three more trig functions that you should be aware of. Again, these are not independent trig functions but are one over the familiar trig functions. These are:
\begin{align*}
\sec x &=\frac{1}{\cos x},\\
\csc x &=\frac{1}{\sin x},\\
\cot x &=\frac{1}{\tan x}.
\end{align*}
Plots of these three functions are shown in \cref{fig: sec function}.

\begin{figure}[ht]
    \centering
  %  \pdftooltip{
  \includegraphics[width=0.8\textwidth, alt ={Graphs of sec(x), csc(x), and cot(x).}]{figures/sec-cot-csc-fig.png}
  %}{A plot of the other trig function. }
    \caption{A plot of the trig functions produced using GeoGebra: $\sec(x)$ the green dashed line, $\csc(x)$ the blue dotted line, and $\cot(x)$ the solid red line. It is not possible to draw any of these functions without taking the pen off the page, we will see later that this is because these functions have vertical asymptotes where the functions head off to infinity.  }
\label{fig: sec function}
\end{figure}

\section{Logarithms and Exponentials}
\label{sec: log and exp}

Another pair of very useful functions which we need to be familiar with are the logarithm and the exponential function. \\

In your previous maths courses you may have come across the fact that we call powers exponents, e.g. in an expression $f(x)=x^{b}$ then $b$ is called the exponent,, or the index. An exponential function is then a function like
\begin{equation*}
f(x)=b^{x},
\end{equation*}
where a constant\footnote{here $b>0$ and $b\neq 1$.} $b$ is raised to the power of the variable $x$. In this case the number $b$ is called the \textbf{\gls{base}} of the exponential function.\\

The exponential function with base $2$ is shown in \cref{fig: exp and log 1}.  It turns out that by appropriate scaling of $x$ different bases are related\footnote{To get an idea of this consider that $2^{x}=4^{x/2}$ so we can always transform from base $2$ into base $4$. In general we need to use the logarithm function that we will meet shortly.} so we only need one standard basis. \\

When working with binary the basis $b=2$ is chosen, while in scientific notation $b=10$ is chosen.  Nowadays the basis in general use is denoted $e$ or $\exp$ with $\exp(x)$ called the \textbf{exponential function}. \\

\begin{mdiv}
\textbf{Warning}! When talking to mathematicians there is a difference between $e$ and $\exp$, the first one is a number, close to $2.71828\dots$, while $\exp$ is a function. Often we will use $e^{x}$ and $\exp(x)$ interchangeably, but occasionally it is important to know the difference.  The main difference is that $\exp(x)$ is a one-to-one function, in particular  when $x=1/2$ we have that $\exp(1/2)\simeq 1.65$ while when taking $\sqrt{e}=e^{1/2}$ we need to pick either the positive or negative square root.  Most likely you can just forget this distinction and work with $e^{x}$ and $\exp(x)$ as if they were the same thing. However, it is worth seeing the distinction at least once.
\end{mdiv}

\begin{figure}[ht]
    \centering
\ThisAltText{Graph of exponential and logarithm functions with base 2.}
 %   \pdftooltip{
    \begin{tikzpicture}[line width=1pt,line cap=round,line join=round, smooth,variable=\x]
     \draw[->] (-3.8,0) -- (5,0) node[above] {$x$};
  \draw[->] (0,-4) -- (0,5)node[above]{$y$};
 \draw[color=CDnavy, domain=-3.8:2.3]   plot[samples=300] (\x,{2^(\x)}) node[right] {$2^{x}$};
  \draw[color=CDgreen, domain=0.06:4]   plot[samples=300] (\x,{log2(\x )}) node[anchor = north west] {$\log_2(x)$};
  \draw[ dashed, color = gray, domain = -2.5:4.5] plot (\x,\x)node[right]{$y=x$};
    \end{tikzpicture}
%    }{plots of an exponential and logarithm }
    \caption{Plots of the exponential and logarithm functions, $y=2^{x}$ and $y=\log_{2}(x)$.}
        \label{fig: exp and log 1}
\end{figure}

The reason for picking $e$ as the basis is that simplifies some of the algebra associated with exponential functions, it also has the advantage that the tangent to the curve has gradient $1$ at $x=0$. This last part boils down to 
\begin{equation}
e\simeq \left(1+h\right)^{\frac{1}{h}} 
\label{eq: exp approximation}
\end{equation}
for $h$ close to zero. 

\begin{figure}[ht]
    \centering
\ThisAltText{Graph of exponential and logarithm functions with base e.}
%    \pdftooltip{
    \begin{tikzpicture}[line width=1pt,line cap=round,line join=round, smooth,variable=\x, ]
     \draw[->] (-3.6,0) -- (5,0) node[above] {$x$};
  \draw[->] (0,-4) -- (0,5)node[above]{$y$};
 \draw[color=CDnavy, domain=-3.8:1.6]   plot[samples=300] (\x,{exp(\x)}) node[right] {$\exp(x)$};
  \draw[color=CDgreen, domain=0.03:4]   plot[samples=300] (\x,{ln(\x )}) node[anchor = north west] {$\ln(x)$};
  \draw[ dashed, color = gray, domain = -2.5:4.5] plot (\x,\x)node[right]{$y=x$};
    \end{tikzpicture}
%    }{plots of exp and ln }
    \caption{Plots of the exponential and logarithm functions, $y=\exp(x)$ and $y=\ln(x)$.}
        \label{fig: exp and log 2}
\end{figure}

Closely related to $\exp$ is its inverse function, called the \textbf{logarithm} function. You will see it written as either $\log(x)$ or $\ln(x)$, or rarely as $\log_{e}(x)$ where the base is made explicit. \\

As this is an inverse function it is defined as the solution to the equation $e^{y}=x$. For example $e^{y}=4$ is solved by $y=\ln(4)$.\\

The graphs of both $\exp(x)$ and $\ln(x)$ are shown in \cref{fig: exp and log 2}. Note that we are, currently, not defining $\ln$ when $x$ is negative \footnote{If you study more mathematics you will discover that we can make sense of $\ln(x)$ when $x$ is negative. However, it is no longer real and we need to understand complex numbers to make sense of it.}.  You may have guessed that we could take a logarithm in a different basis since it is defined so that
\begin{equation*}
\ln(\exp(x))=x=\exp(\ln(x)).
\end{equation*}

This is why there are several slightly different ways of writing the logarithm. If you learnt about it in school you probably used $\log(x)$ to mean log with base $10$. In this module if we work with a basis other than $e$ we will make this explicit by writing $\log_{b}(x)$ to mean log with base $b$.\\

To be clear $\log_{b}(x)$ is the function defined such that
\begin{equation*}
\log_{b}(b^{x})=x=b^{\log_{b}(x)}.
\end{equation*}

A nice property of exponentials and logarithms is that they map addition to multiplication. What this means is that
\begin{align*}
\exp(a)\exp(b)&=\exp(a+b),\\
\ln(ab)&=\ln(a)+\ln(b),\\
\ln\left(\frac{a}{b}\right)&=\ln(a)-\ln(b).
\end{align*}

Some of the other properties of these functions are that:
\begin{align*}
\exp(-x)&=\frac{1}{\exp(x)},\\
\exp(x)&>0,\\
\exp(x)&\to \infty \quad \text{as } x\to \infty,\\
\exp(x)&\to 0 \quad \text{as } x\to -\infty,\\
\ln(x) &\to \infty \quad \text{as } x\to \infty,\\
\ln(x)&\to -\infty \quad \text{as } x\to 0^{+}.
\end{align*}
We will return to discussing limits shortly, but here the statement $\exp(x)\to \infty$ as $x\to \infty$ means that the function $\exp(x)$ becomes infinitely large as $x$ becomes infinitely large. The terminology for this is that $\exp(x)$ \textbf{diverges} as $x$ tends to infinity.\\

The final, useful, identity about logarithms is how to change the basis. The change of basis formula is
\begin{equation}
\log_{b}(x)=\frac{\log_{a}(x)}{\log_{a}(b)}.
\label{eq:log change of base}
\end{equation}

As with trig functions in \cref{sec: trig func}, $\ln$ and $\exp$ are useful when solving equations. There are lots of examples in \citep{calcI} but we will go through a couple of examples here, there will be more on the tutorial sheets.

\begin{ex}
Consider the equation
\begin{equation*}
x e^{-x} -x =0.
\end{equation*}
The first step in solving this is to factor out the $x$ common to both terms. Note that we \textbf{cannot} divide by $x$ as at this stage we do not know if it will be zero. This gives us
\begin{align*}
x e^{-x} -x &=0\\
x\left(e^{-x}-1\right)&=0,
\end{align*}
so we get two possibilities\footnote{Sometimes we will get more than two possibilities, remember with trig functions their periodicity meant that we had infinitely many solutions.}: either $x=0$ or $e^{-x}-1=0$. If we had divided by $x$ we would have missed that $x=0$ is a solution, and would not have completely solved the problem. Consider the second case,
\begin{align*}
e^{-x}&=1\\
-x&=\ln(1)=0
\end{align*}
so this other case also reduces to $x=0$ and we only have one solution in this case.
\end{ex}

\begin{ex}
\label{ex: log equation}
Consider the equation
\begin{equation*}
\frac{1}{2}\ln\left(x^{2}\right)-\ln\left(x-1\right)=4.
\end{equation*}
If we make use of the log rules we have that $\ln(x^{2})/2=\ln(\sqrt{x^{2}})=\ln x$, where we take the positive square root as currently we do not know how to make sense of logarithms with negative arguments. The equation thus becomes
\begin{align*}
4&=\frac{1}{2}\ln\left(x^{2}\right)-\ln\left(x-1\right)\\
&=\ln\left(x\right)-\ln\left(x-1\right)\\
&=\ln\left(\frac{x}{x-1}\right).
\end{align*}
Now we can exponentiate both sides and rearrange to solve for $x$:
\begin{align*}
\frac{x}{x-1}&=e^{4}\\
x&=e^{4}\left(x-1\right)\\
x&=-e^{4}+xe^{4}\\
x\left(1-e^{4}\right)&=-e^{4}\\
x&=-\frac{e^{4}}{1-e^{4}}=1.01866.
\end{align*}

When we get an answer like this we need to check that it does not give a negative argument in any of the original logarithms, here we are fine since both $x^{2}$ and $x-1$ are positive.

\end{ex}

It may seem like example~\ref{ex: log equation} took quite a bit of working out, but with practice you will get faster at solving problems like this. In the lectures you will frequently hear me paraphrasing George P\'{o}lya and saying that Mathematics is not a spectator sport. This means that while reading these notes and attending the lectures can help you in your learning there is no substitute for actually rolling your sleeves up and solving problems. Remember that most modules \textit{expect} you to be putting in around three hours of self study for every hour of contact time.

\section{Hyperbolic functions}
\label{sec: hyperbolic functions}
The next class of functions that it is useful to know about are the \textbf{hyperbolic} trig functions. As shown in \cref{fig: hyperbolic functions}, the hyperbolic trig functions are the analogue of the standard trig functions of \cref{sec: trig func} but adapted to the geometry of the hyperbola, $x^{2}-y^{2}=1$ rather than the circle $x^{2}+y^{2}=1$. For the standard trig functions, their argument was an angle related to how far round the unit circle we had gone. In the hyperbolic case the \textit{angle} is now twice the shaded area shown in \cref{sec: trig func}. \\

\begin{figure}[ht]
    \centering
  %  \pdftooltip{
  \includegraphics[width=0.4\textwidth, alt ={Hyperbolic trig functions are related to the geometry of hyperbolas rather than circles.}]{figures/Hyperbolic_functions}
  %}{A plot of the tan function. }
    \caption{Hyperbolic trig functions are defined in terms of  the area enclosed by a radial ray and the hyperbola $x^{2}-y^{2}=1$.  The convention is to assume that $a$ is negative for points below the $x$-axis. The image is from \href{https://commons.wikimedia.org/wiki/File:Hyperbolic_functions-2.svg}{Wikimedia commons}. }
\label{fig: hyperbolic functions}
\end{figure}


There is a hyperbolic equivalent of all of the standard trig functions, and the notation is very similar just with an added $h$ at the end:
\begin{itemize}
%\setlength{\itemsep}{-5pt}
    \item \textbf{hyperbolic sine}  $\sinh$,
    \item \textbf{hyperbolic cosine} $\cosh$,
   \item \textbf{hyperbolic tangent} $\tanh$,
   \item \textbf{inverse hyperbolic sine} $\arcsinh$,
   \item \textbf{inverse hyperbolic cosine} $\arccosh$,
   \item \textbf{inverse hyperbolic tangent} $\arctan$.
\end{itemize}
There are also hyperbolic versions of the other trig functions but we will not be as concerned with them here.\\

As in the regular trig case the hyperbolic tangent is defined in terms of the other two functions as
\begin{equation}
\tanh(x)=\frac{\sinh(x)}{\cosh(x)}.
\label{eq: hyperbolic tangent}
\end{equation}

The hyperbolic trig functions are actually defined in terms of the odd and even parts of the exponential function\footnote{If you have come across complex numbers then you may know that this is only true for real $x$, if the argument is imaginary then relationships like this hold between the exponential function and the standard trig functions. We may see this in the advanced topics section of the module if there is enough time.} as follows:
\begin{align}
\sinh(x)&=\frac{e^{x}-e^{-x}}{2}, \label{eq: sinh exp}\\
\cosh(x)&=\frac{e^{x}+e^{-x}}{2}. \label{eq: cosh exp}
\end{align}
They have the following useful properties:
\begin{align*}
\sinh(-x)&=-\sinh(x),\\
\cosh(-x)&=\cosh(x),\\
\cosh^{2}(x)-\sinh^{2}(x)&=1,\\
1-\tanh^{2}(x)&=\sech^{2}(x).
\end{align*}
These are very similar to the identities satisfied by the ordinary trig functions. However, the sign in front of $\sinh(x)$ and $\tanh(x)$ is negative while that in front of $\sin(x)$ and $\tan(x)$ in the equivalent formulae was positive. The rule of thumb when converting from identities for trig functions to identities for hyperbolic trig functions is to send $\cos^{2}(x)\to \cosh^{2}(x)$ and $\sin^{2}(x)\to -\sinh^{2}(x)$.  There are also analogues of the addition of angle identity formulas that you can try to derive if you are interested.\\


\begin{ex}
Consider the hyperbolic equation $\cosh(x)-5\sinh(x)-5=0$ from chapter 3.7 in \citep{riley_mathematical_2006}. The easiest way to solve this is to use the definitions of the hyperbolic functions in terms of the exponential function. This transforms the equation into
\begin{equation*}
\frac{1}{2}\left(e^{x}+e^{-x}\right)-\frac{5}{2}\left(e^{x}-e^{-x}\right)-5=0,
\end{equation*}
which we rearrange to give
\begin{equation*}
0=e^{x}\left(\frac{1}{2}-\frac{5}{2}\right)+e^{-x}\left(\frac{1}{2}+\frac{5}{2}\right)-5=-2e^{x}+3e^{-x}-5.
\end{equation*}
Then multiply through by $-e^{x}$, which is allowed since this is never zero, to get
\begin{equation*}
2e^{2x}+5e^{x}-3=0.
\end{equation*}
Now we could either let $y=e^{x}$ and use the quadratic formula to solve for $y$ or we can factorise this to get
\begin{equation*}
\left(2e^{x}-1\right)\left(e^{x}+3\right)=0,
\end{equation*}
so there are two solutions: $e^{x}=1/2$ or $x=-\ln(2)$, and $e^{x}=-3$ or $x=\ln(-3)$. Remember we have not discussed how to make sense of the logarithm of a negative number so for us there is only one real solution, $x=-\ln(2)$.
\end{ex}

Since the hyperbolic trig functions are related to the exponential function you can probably guess that their inverses are related to the logarithm. It is worth thinking about how to show this relationship. I will show you how to do this for $\sinh(x)$ and $\tanh(x)$ but leave deriving the identify for $\cosh(x)$ as an exercise for the interested reader.

\begin{ex}
Consider $y=\arcsinh(x)$, we can invert this to give $x=\sinh(y)$. Next if we make use of \cref{eq: sinh exp,eq: cosh exp} we have that
\begin{align*}
e^{y}	&=\cosh(y)+\sinh(y)\\
	&=\sqrt{1+\sinh^{2}(y)}+\sinh(y)\\
	&=\sqrt{1+x^{2}}+x.
\end{align*}
Taking $\ln$ of both sides then gives that
\begin{equation}
\arcsinh(x)=y=\ln\left(\sqrt{1+x^{2}}+x\right).
\label{eq: arcsinh log}
\end{equation}
\end{ex} 

If you do a similar calculation for $y=\arccosh(x)$ you will find that
\begin{equation}
\arccosh(x)=\ln\left(x\pm \sqrt{x^{2}-1}\right).
\label{eq: arccosh log}
\end{equation} 

You should think about why there is a $\pm$ in this formula but there was not one in \cref{eq: arcsinh log}.

\begin{ex}
Consider $y=\arctanh(x)$, which inverts to $x=\tanh(y)$. Using the definition of $\tanh(y)$ as being $\sinh(y)/\cosh(y)$ and \cref{eq: sinh exp,eq: cosh exp}  we have that
\begin{equation*}
x=\frac{e^{y}-e^{-y}}{e^{y}+e^{-y}}
\end{equation*}
which is equivalent to 
\begin{equation*}
\left(x+1\right)e^{-y}=\left(1-x\right)e^{y}.
\end{equation*}
which can be further rearranged to give
\begin{align*}
e^{2y}&=\frac{1+x}{1-x},\\
\Rightarrow e^{y}&=\sqrt{\frac{1+x}{1-x}},\\
y&=\ln\left(\sqrt{\frac{1+x}{1-x}}\right).
\end{align*}
This gives that 
\begin{equation}
\arctanh(x)=\frac{1}{2}\ln\left(\frac{1+x}{1-x}\right).
\label{eq: arctanh log}
\end{equation} 

\end{ex}

You may be asking why we have spend the time deriving these identities for the hyperbolic functions, and why we discussed their relationship with the exponential function. This is because when we start to differentiate or integrate hyperbolic functions it is often more straightforward to use the expressions involving exponentials and logarithms. 


\section{Limits and asymptotics}
We have now spent quite a bit of time discussing different examples of functions and their properties, as well as how to plot them. If we consider the plot of the tangent function in \cref{fig: tan function} you may have asked at the time, what happens when the red line goes off the top of the page and reappear at the bottom? If we followed both lines we would see that they are getting closer and closer to the point $x=\uppi/2$. This idea of zooming in on a particular value $x=a$ and asking what happens to a function there is known as \textbf{taking a limit} as $x$ approaches $a$ and is denoted $\lim_{x\to a}f(x)$.\\

If the function is \textit{well behaved} at the point $a$, then the limit is just the value of the function evaluated at $a$, $f(a)$. For example
\begin{equation*}
\lim_{x\to 0}\cos(x)=1=\cos(0).
\end{equation*}
The value of the limit does not have to be finite, it can head off to infinity. A good example of this is the exponential function which just keeps getting larger as $x$ increases. We denote this by
\begin{equation*}
\lim_{x\to \infty}\exp(x)\to \infty.
\end{equation*}
Notice that since the value of the limit is infinite we do not write an equals sign but instead use $\to$ to denote that the limit \textbf{diverges}\footnote{This is just the term that mathematicians use for a function that heads off to infinity.}. There are lots of other examples of functions that diverge, and not always for large $x$. A nice example to have in mind is $y=1/x^{2}$ which diverges as $x\to 0$.\\

In the case of $\tan(x)$ we can observe that $\lim_{x\to \uppi/2}\tan(x)$ will take on different values depending on if $x$ is approaching $\uppi/2$ from above or below. This means that the limit \textbf{does not exist}. In \cref{sec: continuity} we will discuss the consequences of this for the function in more detail. If we are taking the limit from above we often say $x\to a^{+}$, while the limit from below is denoted $x\to a^{-}$. In this case
\begin{align*}
\lim_{x\to \frac{\uppi}{2}^{+}}\tan(x)&\to \infty,\\
\lim_{x\to \frac{\uppi}{2}^{-}}\tan(x)&\to -\infty,
\end{align*}
so $\tan(x)$ diverges in different directions depending on how we approach $\uppi/2$.

\begin{mdiv}
Putting our mathematician's hats on for a brief moment, we would say that limit of a function $f(x)$ as $x$ approaches $a$ is $L$ and write it as
\begin{equation*}
\lim_{x\to a}f(x)=L,
\end{equation*}
as long as we can make $f(x)$ as close to $L$ as we want for all $x$ sufficiently close to $a$ on both sides, without taking $x=a$.\\

If this was a maths module we would be even more precise and give what is called an $\upepsilon,\updelta$ definition. Fortunately for all of us this is not a maths module and we can focus on a more practical/ working  definition rather than getting bogged down in the abstract details. If you are interested to know how to do this in detail, the section ``The Definition of a Limit'' in \citep{calcI} is a good place to look.  \\

\textbf{Remember} that $x\to a$ does not mean that $x$ becomes equal to $a$, it just means that $x$ is getting close to $a$. It is also possible that $f(x)$ may never equal $L$, but also just get close to it. This is especially true when we take the limit  near a point that the function is not defined at, e.g. $x\to \uppi/2$ for $\tan(x)$.
\end{mdiv}

Curve sketching is a useful way to understand the main features of a function and can help us to know if the limit exists and if we have the same limit when approaching from different directions. 

\begin{ex}
Consider the function 
\begin{equation}
f(x)=\frac{1}{x-2},
\label{eq: reciprocal of x-2}
\end{equation}
defined for $x\neq 2$.\\

Notices that for $x\to\pm \infty$ $f(x)\to 0$, in \cref{fig: asymptotes function} we see that the plot approaches the horizontal axis \textbf{asymptotically}\footnote{The definition of asymptotically is ``approaching a given value or condition, as a variable or an expression containing a variable approaches a limit, usually infinity'' \citep{collins:asymptotically}}. We can also observe that for $a\neq 2$ $f(x)\to f(a)$ for $x\to a$.  The only point where we need to be careful is when $x$ tends to $2$ looking at \cref{fig: asymptotes function} we see that if we approach $2$ from above that $f(x)\to \infty$ while if we approach $2$ from below $f(x)\to -\infty$. \\

Note that the fact that $f(x)$ is not defined at $x=2$ has no bearing on the limiting behaviour.

\end{ex}

\begin{figure}[ht]
    \centering

  \includegraphics[width=0.6\textwidth, alt={A plot of the reciprocal of (x-2) which tends to zero as x goes to either plus or minus infinity and diverges as x tends to 2.}]{figures/asymptotes}
    \caption{A plot of the function from \cref{eq: reciprocal of x-2} produced using GeoGebra. The blue line is the function and the red line shows where the function diverges and has a vertical asymptote.}
\label{fig: asymptotes function}
\end{figure}

The general rule when curve sketching is to set $y=f(x)$ for the function of interest $f(x)$, and then to investigate the following:
\begin{itemize}
%\setlength{\itemsep}{-5pt}
    \item The \textbf{intercepts} of $f(x)$, these are the points where the function meets or crosses the coordinate axes, e.g. $f(x=0)$ and $f(x)=0$, as these position the graph relative to the coordinate axes.
    \item What happens to $y$ as $x\to\pm\infty$?
   \item Which values of $x$ make $y\to \pm\infty$?
   \item Does the function have any \textbf{symmetry}? For example does $f(-x)=f(x)$ ($f$ is even), or $f(-x)=-f(x)$ ($f$ is odd).
   \item Is the function periodic? 
\end{itemize}
Note that a function can have several of these interesting features. e.g. the trig functions are all periodic, while $\sin(x)$ and $\tan(x)$ are odd functions and $\cos(x)$ is an even function.

\begin{mdiv}
An asymptote to a curve is a straight line that becomes close to the curve as either $x$ or $y$ tend to $\pm\infty$. In our above example the curve of \cref{eq: reciprocal of x-2} has two asymptotes, the horizontal $x$-axis for $x\to \pm\infty$ and the vertical line $x=2$ for $x\to 2$.\\

When sketching a curve it is often useful to start by drawing the asymptotes in as they help you to know how the function will behave in certain regions of the plot. In some books you may see $f(x)\asymp mx +c$ as $x\to \infty$ to signify that the straight line $y=mx+c$ is an asymptote to the graph of $f(x)$ as $x$ tends to infinity.
\end{mdiv}

\begin{exercise}
Apply the above rules to produce a sketch of the function
\begin{equation*}
f(x)=\frac{x+2}{x-1}.
\end{equation*}
\end{exercise}

\begin{ex}
Consider the rational function 
\begin{equation*}
f(x)=\frac{x^{2}+4x-12}{x^{2}-2x}
\end{equation*}
and compute its limit as $x\to 2$.

Looking at the function you may think that it will diverge as $x\to 2$ since the denominator vanishes there. However the first step is always to look at if there are any common factors between the numerator and denominator. Factorising both we have
\begin{align*}
f(x)&=\frac{x^{2}+4x-12}{x^{2}-2x}\\
&=\frac{(x+6)(x-2)}{x(x-2)}\\
&=\frac{x+6}{x}.
\end{align*}
So the denominator does not vanish at $x=2$. We can now evaluate the limit to be
\begin{equation*}
\lim_{x\to 2}f(x)=\lim_{x\to 2}\frac{x+6}{x}=\frac{8}{2}=4.
\end{equation*}

The limit is the same regardless of which direction we approach $2$ from.
\end{ex}

\begin{exercise}
Estimate the limit from the above example by constructing a table of values of $x$ and $f(x)$ for $x$ approaching 2 from both above and below.
\end{exercise}

As a warning, this approach of tabulating the values does not work if the function is oscillating. For example, if you try to evaluate the limit as $x\to 0$ of $\cos\left(\uppi/x\right)$ it can look like it is tending to a constant if we are not careful with the values that we pick. However, if we plot the function we see that it is highly oscillatory around zero.

\begin{ex}
Consider the rational function
\begin{equation*}
f(x)=\frac{x^{3}+x^{2}-5x-2}{2x^{3}-7x^{2}+4x+4}.
\end{equation*}
It has three interesting limits, $x\to 0,\infty,2$. We can evaluate the first two of these here, but need to leave the third, $x\to 2$ until we have learnt about L'H\^{o}pital's rule in \cref{sec:advanced topics}. We will do $x\to 0$ first. As usual we check that the numerator and denominator both make sense as $x\to 0$ and then can evaluate the limit
\begin{align*}
\lim_{x\to 0}f(x)&=\lim_{x\to 0}\frac{x^{3}+x^{2}-5x-2}{2x^{3}-7x^{2}+4x+4}\\
&=\frac{-2}{4}\\
&=-\frac{1}{2}.
\end{align*}

Next we do the $x\to \infty$ limit.  Before we do this we need to multiply $f(x)$ by $x^{-3}/x^{-3}$ so that it becomes
\begin{align*}
\lim_{x\to \infty}f(x)	&=\lim_{x\to \infty}\left(\frac{x^{-3}}{x^{-3}}\frac{x^{3}+x^{2}-5x-2}{2x^{3}-7x^{2}+4x+4}\right)\\
				&=\lim_{x\to \infty}\left(\frac{1+x^{-1}-5x^{-2}-2x^{-3}}{2-7x^{-1}+4x^{-2}+4x^{-3}}\right)\\
				&=\frac{1}{2}.
\end{align*}
\end{ex}

Remember that when evaluating a limit we want to look for as many cancellations as possible to simplify the calculation. This can be looking for common factors between the numerator and denominator, but can also mean fully expanding out all terms as there may be cancellations hidden by the way the function has been written.

Often it is useful to plot a function when we are estimating the limit, while this is not compulsory it can be very helpful, particularly if as in the case of $\tan(x)$, the value of the limit depends on the direction of approach.\\

\begin{figure}[ht]
    \centering
  %  \pdftooltip{
  \includegraphics[width=0.5\textwidth,alt={A plot of a step function.}]{figures/step_function}
  %}{A plot of the tan function. }
    \caption{A plot of a step function from \cref{eq: step function} produced using GeoGebra.}
\label{fig: step function}
\end{figure}

\begin{ex}
Consider the step function
\begin{equation}
f(x)=\begin{cases}
&0 \quad x < 0\\
&1 \quad x\geq 0,
\end{cases}
\label{eq: step function}
\end{equation}
shown in \cref{fig: step function}. By looking at the graph we see that the limit of $f(x)$ as $x\to 0$ will depend on the direction of approach. If we start from below zero it is clear that $f(x)\to 0$, while starting above zero it is clear that $f(x)\to 1$. In contrast to the case of $\tan(x)$ the function is not diverging in the limit, but we still end up with a \textbf{One Sided Limit}\footnote{This just means that the function is well behaved in the limit as long as we only look at one side. }
\end{ex} 



\begin{mdiv}
Note that it can be dangerous to use $\infty$ in calculations as if it were a number. This is because there are certain limits and other expressions that we may see which do not make sense:
\begin{equation*}
\infty^{0}, \quad \infty-\infty, \quad \frac{0}{0}, \quad 0^{0}, \quad \frac{\infty}{\infty}, \quad \frac{\infty}{0}, \quad \frac{1}{0}.
\end{equation*}
Just because we see one of these expressions coming out a limit does not necessarily mean that the limit does not exist. It can instead mean that we need to be very careful how we evaluate the limit. In \cref{sec:advanced topics} we will discuss L'H\^{o}pital's rule which is a techniques for evaluating limits that initially look like they do not make sense.
\end{mdiv}

There are a few limits that we need to know that we will not prove here, time permitting we will prove these in \cref{sec:advanced topics}, but you can also look at the ``Proof of Trig Limits'' section of \cite{calcI} to see one approach to proving them. These limits are
\begin{align}
\lim_{x\to 0}\frac{\sin x}{x}&=1, \label{eq: sin limit}\\
\lim_{x\to 0}\frac{\cos x -1}{x}&=0, \label{eq: cos limit}\\
\lim_{h\to 0}\frac{e^{h}-1}{h}&=1, \label{eq: exp limit}\\
\lim_{h\to 0}\frac{\ln(1+h)}{h}&=1. \label{eq: ln limit}
\end{align}
Note that \cref{eq: exp limit} is equivalent to \cref{eq: exp approximation} which we discussed above as one of the definitions of the exponential function.

\section{Continuity and differentiability}
\label{sec: continuity}
Now that we have some examples of functions and understand how to take limits, we can define two properties of a function which will be very important later in the course: \textbf{continuity} and \textbf{differentiability}. \\

A function $f(x)$ is said to be continuous at a point $x=a$ if
\begin{equation}
\lim_{x\to a}f(x)=f(a).
\label{eq: continuity}
\end{equation}

If $X$ is the domain of $f(x)$, we say that $f$ is continuous on $X$ if it is continuous at each point in $X$, often we would just say that $f$ is a continuous function. As a rule of thumb, we can say that a function is continuous if its graph can be drawn from start to finish without taking your pen off the paper. Functions which are not continuous will have jumps or divergences at the point that fails to be continuous.\\

If we have a rational function then we can find where it is not continuous, called being \textbf{discontinuous}, by finding the roots of the denominator.\\

\begin{mdiv}
If we are being careful we need to give three parts to the definition of continuity: 
\begin{itemize}
%\setlength{\itemsep}{-5pt}
    \item[1)] The limit of $f(x)$ as $x\to a$ exists.
    \item[2)] The value of $f(x)$ is defined at $x=a$. i.e. $f(a)$ is defined and is finite.
   \item[3)] The limit of $f(x)$ as $x\to a$ agrees with the value of $f(a)$.
\end{itemize}
This is another place where as this is not a course for mathematicians we can combine all three of these into the one statement in \cref{eq: continuity}. This is another example of being able to use a working definition and not needing to get sidetracked by all of the technical details.
\end{mdiv}

\begin{ex}
We have already met several examples of continuous and discontinuous functions:
\begin{itemize}
%\setlength{\itemsep}{-5pt}
    \item The trig functions $\cos(x)$ and $\sin(x)$ are continuous on all of $\R$.
    \item The exponential function $\exp(x)$ is continuous on all of $\R$.
   \item The natural logarithm is continuous on the positive real numbers $\R$ as, currently we have not defined it for negative $x$, and it diverges in the limit $x\to 0$.
   \item The tangent function $\tan(x)$ are not continuous on $\R$ due to its divergences at $\pm\frac{\uppi}{2},\pm\frac{3\uppi}{2},\dots{}$. However, it is continuous on its domain\footnote{Strictly speaking this is the principle domain of $\tan(x)$ its domain is really $\{x\in \R\vert x\neq \uppi/2 +n\uppi, n\in \Z\}$} $\left(-\frac{\uppi}{2},\frac{\uppi}{2}\right)$
   \item The step function of \cref{eq: step function} is not continuous at $x=0$.
\item The function in \cref{eq: reciprocal of x-2} has a discontinuity at $x=2$
\end{itemize}
\end{ex}
\begin{exercise}
Use the definition of continuity to determine if the function
\begin{equation}
g(x)=\frac{4x+10}{x^{2}-2x-15}
\end{equation}
is continuous and if not find the points where it has discontinuities.
\end{exercise}

Related to continuity is differentiability. A function $f(x)$ is differentiable at a point $a$ if the limit
\begin{equation*}
\lim_{x\to a}\frac{f(x)-f(a)}{x-a}
\end{equation*}
exists. We call this limit the derivative of the function at the point $a$,
\begin{equation}
f'(a)=\lim_{x\to a}\frac{f(x)-f(a)}{x-a},
\label{eq: rate of change at a}
\end{equation}
the notation $\frac{\ud f}{\ud x}(a)$ is sometimes used instead of $f'(a)$. A function is called continuously differentiable if its derivative is also continuous.\\

It is important to note that differentiability at $a$ implies continuity at $a$, but continuity does not imply differentiability.  For example the absolute value function shown in \cref{fig: abs function} is continuous everywhere, but is only differentiable everywhere except at $x=0$.

\begin{figure}[htbp]
    \centering
\ThisAltText{Graph of the absolute value function.}
%    \pdftooltip{
    \begin{tikzpicture}[line width=1pt,line cap=round,line join=round,domain=-2.5:2.5, smooth,variable=\x]
     \draw[->] (-3,0) -- (3.3,0) node[above] {$x$};
  \draw[->] (0,-0.2) -- (0,3) node[above] {$y$};
 \draw[color=red]   plot (\x,{abs(\x)}) node[right] {$f(x)=\vert x\vert$};
    \end{tikzpicture}
%    }{absolute value function}
    \caption{The absolute value function $f(x)=\vert x\vert$ is continuous at every point, but it is not differentiable at the point $x=0$.}
        \label{fig: abs function}
\end{figure}

\begin{ex}
Consider the modulus or absolute value function shown in \cref{fig: abs function}. We have said that this is a continuous function which is not differentiable at $x=0$, but how do we show this? We do it by checking all of the limits.\\

Checking continuity at $x=0$ is left as an exercise. For differentiability we need to check the limits as $x\to 0^{+}$ and $x\to 0^{-}$. 

In the first one we have:
\begin{equation*}
\lim_{x\to 0^{+}}\frac{f(x)-f(0)}{x}=\lim_{x\to 0^{+}}\frac{x-0}{x}\lim_{x\to 0^{+}}1=1,
\end{equation*}
while approaching from the other direction gives:
\begin{equation*}
\lim_{x\to 0^{-}}\frac{f(x)-f(0)}{x}=\lim_{x\to 0^{-}}\frac{(-x)-0}{x}\lim_{x\to 0^{-}}-1=-1.
\end{equation*}
These do not agree, so the limit does not exist and the modulus function is not differentiable.
\end{ex}

\begin{figure}[ht]
    \centering
    %\pdftooltip{
\ThisAltText{Graph of a straight line function.}
\begin{tikzpicture}[line width=1pt,line cap=round,line join=round]
\draw[black, ultra thick,->] (0,0) --(5,0) node[anchor=west]{$x$ axis};
\draw[black, ultra thick,->] (0,0) --(0,5) node[anchor=south]{$y$ axis};
%\draw[step=1cm,gray,very thin] (-2,-2) grid (6,6);
%\draw[blue, ultra thick] (0,0) parabola (6,6);
\draw[red, ultra thick] (0,0) -- (5,5);
 \filldraw[black] (2,2) circle (2pt) node[right,xshift=3mm]{$y_{1}$ at $x_{1}$};
 \filldraw[black] (4,4) circle (2pt) node[right,xshift=3mm]{$y_{2}$ at $x_{2}$};
\end{tikzpicture}
%}{A displacement time graph showing the difference between constant velocity motion and motion with a changing velocity.}
    \caption{A plot of the straight line $y=x$ and example of $y=mx+c$ with gradient $m=1$ and $y$-intercept $c=0$, with two points on the line marked.}
    \label{fig: straight line graph.}
\end{figure}

The fraction in \cref{eq: rate of change at a} may look familiar to you. If we had a straight line $y=f(x)=mx+c$ where $m$ is the gradient of the straight line and $c$ is the $y$-intercept, then calculating this fraction gives
\begin{equation*}
\frac{f(x)-f(a)}{x-a}=\frac{mx+c-(ma+c)}{x-a}=\frac{m(x-a)}{x-a}=m,
\end{equation*}
which is the gradient of the curve. For a function that is not a straight line, this procedure gives the gradient of the straight line between $x$ and $a$. In the limit that $x\to a$ this fraction becomes the gradient of the \textbf{tangent} line\footnote{A tangent is a line that touches a curve at one point} to the curve at $a$.   See \cref{fig: parabola tangent} for the example of the tangent to parabola $y=x^{2}$.


\begin{mdiv}
The fraction
\begin{equation*}
\frac{f(x)-f(a)}{x-a}
\end{equation*}
is sometimes referred to as the Newton quotient of the function $f(x)$ at the point $a$. This is after Isaac Newton because when calculating a derivative we are calculating this quotient for smaller and smaller differences $x-a$.
\end{mdiv}

\begin{figure}[ht]
    \centering
\ThisAltText{Graph of a parabola with a tangent curve shown .}
\begin{tikzpicture}[line width=1pt,line cap=round,line join=round]
    \begin{axis}[
            xtick = \empty,    ytick = \empty,
            xlabel = {$x$},
            x label style = {at={(1,0)},anchor=west},
            ylabel = {$y$},
            y label style = {at={(0,1)},rotate=-90,anchor=south},
            axis lines=left,
            enlargelimits=0.2,
        ]
        \addplot[color=black,smooth,thick,-] {(x)^2};
        \addplot[mark=none, red] coordinates {(-6,20) (0,-4)};
    \end{axis}
\end{tikzpicture}
 \caption{A plot of a parabola in black and its tangent in red}
    \label{fig: parabola tangent}
\end{figure}
\newpage

%%%%%%%%%%%%%%%%%%%%%%%%%%%%%%%%%%%%%%%%%%%%%%


\chapter{Differentiation}
\label{sec:differentiation}
\epigraph{You take a function of $x$ and you call it $y$.\\ Take any $x_{0}$ that you care to try.\\ You make a little change and call it $\Delta x$.\\ The corresponding change in $y$ is what you find next. }{\textit{The Derivative Song by Tom Lehrer}}

\section{Differentiation from first principles}
In the previous \namecref{sec:functions}, we encountered the derivative of a function at a point in \cref{eq: rate of change at a} and its interpretation as the tangent to a curve at the point $a$. With a small amount of rewriting, setting $x=a+h$ for $x$ ``near'' $a$, this becomes
\begin{align*}
f'(a)&=\lim_{x\to a}\frac{f(x)-f(a)}{x-a}\\
&=\lim_{h\to 0}\frac{f(a+h)-f(a)}{a+h-a}\\
&=\lim_{h\to 0}\frac{f(a+h)-f(a)}{h}.
\end{align*} 
This is still an expression for the tangent to a curve at a specific point $x=a$. However, if we are interested in the gradient at an arbitrary point $x$, then we can rewrite it as
\begin{equation}
f'(x)=\lim_{h\to 0}\frac{f(x+h)-f(x)}{h}.
\label{eq: derivative definition}
\end{equation}

The formula in \cref{eq: derivative definition} is the definition of the \textbf{derivative} of the function $f(x)$. Not that only functions where the derivative exists for all points are called \textbf{differentiable}. There is a range of notation\footnote{We will only use the two most common notations in this course. However, another notation that you may see in books is $f_{x}(x)$, where the subscript shows what we are differentiating with respect to. This notation shows up a lot if we have functions of more than one variable where we need to make it clear which variable we are differentiating with respect to. In the assessed part of this course we will only care about functions of one variable so do not need this notation.} and terminology used to denote derivative. The notation used above,  $f'(x)$ is known as the \textbf{Lagrange} or \textbf{Euler} notation\footnote{In an example of Sigler's law of eponymy, this is most commonly called Lagrange's notation even though it was first used by Euler. It is also quite close to the original notation that Newton used when he discovered calculus and referred to it as the \textbf{method of fluxions}. }.  The other very common notation is due to \textbf{Leibniz} where we write
\begin{equation}
\frac{\ud f}{\ud x}=\lim_{h\to 0}\frac{f(x+h)-f(x)}{h},
\label{eq: Leibniz definition}
\end{equation}
this notation is particularly useful when we learn about integration as in certain contexts we can treat the derivative like a fraction.\\

\begin{mdiv}
In \cref{eq: Leibniz definition} it looks like the right hand side is a fraction. This is not true, other than in certain very specific circumstances, we will not see $\ud f$ or $\ud x$ appearing on their own. In Newton's approach the Newton quotient is sometimes written as
\begin{equation*}
\frac{\Delta f}{\Delta x}=\frac{f(x)-f(a)}{x-a},
\end{equation*}
which does make sense as a fraction. The limit where this becomes the derivative is when the change in $x$, $\Delta x=x-a$, goes to zero. In this limit if we really had a fraction it would look like $0/0$, which is one of the nonsense expressions that we mentioned earlier. The power of calculus is that it enables us to make sense of this limit, but what we loose is the ability to treat it as a fraction. When we discuss integration and differential equations later on in the module we will return to this idea.
\end{mdiv}

If we use \cref{eq: derivative definition} to calculate the derivative of a function this is called \textbf{differentiation by first principles}. As you might expect, when the function $f(x)$ becomes more complicate calculating the derivative in this way becomes more complicated as well. Fortunately, there are certain standard rules and techniques that we can learn to simplify matters. With a little work all of these can be proved from the definition of the derivative, some of these proofs will be given here but others are left as an exercise to the interested reader.\\

As a warm up we will use \cref{eq: derivative definition} to calculate the derivative of a straight line.
\begin{ex}
Consider $f(x)=mx+c$ and calculate the derivative from first principles:
\begin{align*}
f'(x)	&=\lim_{h\to 0}\frac{f(x+h)-f(x)}{h}\\
	&=\lim_{h\to 0}\frac{m(x+h)+c-(mx+c)}{h}\\
	&=\lim_{h\to 0}\frac{mx+mh+c-mx-c}{h}\\
	&=\lim_{h\to 0}m\frac{h}{h}\\
	&=\lim_{h\to 0}m=m.
\end{align*}
So derivative of a straight lime is a constant, the gradient of the line. We already knew this, but it is a good consistency check to ensure that our definition of the derivative is working as expected.
\end{ex}

\begin{ex}
Consider the function $f(x)=4x^2 -6x +2$, using \cref{eq: derivative definition} we calculate its derivative as follows:
\begin{align*}
f'(x)	&=\lim_{h\to 0}\frac{f(x+h)-f(x)}{h}\\
	&=\lim_{h\to 0}\frac{4(x+h)^2 -6(x+h) +2)-(4x^2 -6x +2)}{h}\\
	&=\lim_{h\to 0}\frac{4(x^{2}+2xh+h^{2})-6x-6h+2-4x^{2}+6x-2}{h}\\
	&=\lim_{h\to 0}\frac{8xh+4h^{2}}{h}\\
	&=\lim_{h\to 0}\left(8x+4h\right)=8x.
\end{align*}
Notice that since the curve is no longer a straight line the derivative , and thus the gradient of the tangent to the curve, depends where the point is along the curve. 
\end{ex}

Remember that if the limit in \cref{eq: derivative definition} does not exist at a particular value of $x$, then the derivative does not exist. In other words, the definition of the derivative only makes sense for functions which satisfy the condition of differentiability.

\begin{ex}
Consider the function 
\begin{equation*}
g(x)=\frac{1}{x+1},
\end{equation*}
and calculate its derivative. Note that this function has a discontinuity at $x=-1$ so it will not be differentiable at that point.\\

Calculating $g'(x)$ is good practice as we need to be careful when we have fractions.
\begin{align*}
g'(x)	&=\lim_{h\to 0}\frac{g(x+h)-g(x)}{h}\\
	&=\lim_{h\to 0}\frac{1}{h}\left(\frac{1}{x+h+1}-\frac{1}{x+1}\right)\\
	&=\lim_{h\to 0}\frac{1}{h}\left(\frac{x+1}{(x+h+1)(x+1)}-\frac{x+h+1}{(x+1)(x+h+1)}\right)\\
	&=\lim_{h\to 0}\frac{1}{h}\left(\frac{x+1-x-h-1}{(x+h+1)(x+1)}\right)\\
	&=\lim_{h\to 0}\frac{1}{h}\left(\frac{-h}{(x+h+1)(x+1)}\right)\\
	&=\lim_{h\to 0}\frac{-1}{(x+h+1)(x+1)}\\
	&=-\frac{1}{(x+1)^{2}}.
\end{align*}
\end{ex}

We can also calculate the derivatives of some of the special functions from first principles.
\begin{ex}
Consider $f(x)=\sin(x)$, this is a continuous function so we can hope that the derivative exists. Note that we can use the addition formula for $\sin(x)$ to expand $\sin(x+h)$ as
\begin{equation*}
\sin(x+h)=\sin(x)\cos(h)+\sin(h)\cos(x).
\end{equation*}
Thus \cref{eq: derivative definition} becomes
\begin{align*}
f'(x)	&=\lim_{h\to 0}\frac{f(x+h)-f(x)}{h}\\
	&=\lim_{h\to 0}\frac{\sin(x+h)-\sin(x)}{h}\\
	&=\lim_{h\to 0}\frac{\sin(x)\cos(h)+\sin(h)\cos(x)-\sin(x)}{h}\\
	&=\lim_{h\to 0}\left(\sin(x)\frac{\cos(h)-1}{h}+\cos(x)\frac{\sin(h)}{h}\right)\\
	&=\sin(x)\lim_{h\to 0}\frac{\cos(h)-1}{h}+\cos(x)\lim_{h\to 0}\frac{\sin(h)}{h}\\
	&=\cos(x),
\end{align*}
where we have used the trig limits from \cref{eq: sin limit,eq: cos limit}
\end{ex}

\begin{exercise}
Show that the derivative of $f(x)=\cos(x)$ is $f'(x)=-\sin(x)$ using differentiation from first principles.
\end{exercise}

\begin{ex}
Consider $f(x)=e^{x}$, its derivative is
\begin{align*}
f'(x)	&=\lim_{h\to 0}\frac{f(x+h)-f(x)}{h}\\
	&=\lim_{h\to 0}\frac{e^{x+h}-e^{x}}{h}\\
	&=\lim_{h\to 0}\frac{e^{x}e^{h}-1}{h}\\
	&=e^{x}\lim_{h\to 0}\frac{e^{h}-1}{h}\\
	&=e^{x}.
\end{align*}
Where we use \cref{eq: exp limit} to evaluate the limit in the final line. This result that the derivative of $e^{x}$ is equal to $e^{x}$, is sometimes taken to be a definition of the exponential function. 
\end{ex}

The other standard derivative that you need to know is the derivative of the natural logarithm.
\begin{ex}
Consider $f(x)=\ln(x)$, we calculate its derivative as
\begin{align*}
f'(x)	&=\lim_{h\to 0}\frac{f(x+h)-f(x)}{h}\\
	&=\lim_{h\to 0}\frac{\ln(x+h)-\ln(x)}{h}\\
	&=\lim_{h\to 0}\frac{1}{h}\ln\left(\frac{x+h}{x}\right)\\
	&=\lim_{h\to 0}\frac{1}{h}\ln\left(1+\frac{h}{x}\right),
\end{align*}
now we can let  $\epsilon =h/x$ which is tending to zero as $h$ tends to zero. Thus the derivative becomes
\begin{align*}
f'(x)	&=\lim_{\epsilon\to 0}\frac{1}{x\epsilon}\ln\left(1+\epsilon\right)\\
	&=\frac{1}{x}\lim_{\epsilon\to 0}\frac{\ln\left(1+\epsilon\right)}{\epsilon}\\
	&=\frac{1}{x},
\end{align*}
where we have made use of \cref{eq: ln limit}
\end{ex}

Using the first principles definition to calculate derivatives is hard work and involves careful manipulation of limits. You will be please to know that we will only use it in certain relatively simple cases. For more complicated expressions we can use a range of other techniques and formulas. Eventually you will internalise some of the standard derivative expressions or use a formula sheet like that in \cref{sec: deriv sheet}.

%\section{Standard derivatives}


\section{Differentiation techniques}

\subsection*{Properties of the derivative}
In this section we will see the various formulae and rules in both the Lagrange and Leibniz notation so you should be familiar with both and use the notation that you feel most comfortable with.\\

The first thing that we need is to know is how to differentiate the sum of two functions and the product of a function with a number.  The proof of these results will be given in \cref{sec: proofs}. These formulae are:
\begin{align}
\frac{\ud}{\ud x}\left(f(x)+g(x)\right)	&=\frac{\ud f(x)}{\ud x}+\frac{\ud g(x)}{\ud x}, \label{eq: derivative of sum}\\
\frac{\ud}{\ud x}\left(f(x)-g(x)\right)	&=\frac{\ud f(x)}{\ud x}-\frac{\ud g(x)}{\ud x}, \label{eq: derivative of difference}\\
\frac{\ud}{\ud x}\left(cf(x)\right)	&=c\frac{\ud f(x)}{\ud x},\label{eq: derivative scalar multiplication}
\end{align}
where $c$ is any number.\\

In the Lagrange/Euler notation these formulae are:
\begin{align}
\left(f(x)+g(x)\right)'	&=f'(x)+g'{x}, \label{eq: derivative of sum 2}\\
\left(f(x)-g(x)\right)'&=f'(x)-g'(x), \label{eq: derivative of difference 2}\\
\left(cf(x)\right)'	&=cf'(x). \label{eq: derivative scalar multiplication 2}
\end{align}

These are useful formulae that we will frequently use in calculations as it enables us to split up the functions that we are differentiating into smaller tractable parts. The other shortcuts that we have is that the derivative of a constant is zero,
\begin{equation}
\frac{\ud c}{\ud x}=0, \label{eq: derivative of constant}
\end{equation}
and that the derivative of a \textbf{monomial} is
\begin{equation}
\frac{\ud}{\ud x}\left(x^{n}\right)=nx^{n-1}. \label{eq: monomial derivative}
\end{equation}
Sometimes the second formula is referred to as the \textbf{power rule}, and is one of the most important rules that you can learn as it enables you to differentiate any polynomial.\\

The formula in \cref{eq: derivative of constant}, which says that the derivative of a constant is zero makes sense since the derivative measures the rate of change of a function. A constant is, by definition, not changing so its rate of change is zero.\\

We can now use these rules to calculate some example derivatives.
\begin{ex}
Consider the function 
\begin{equation*}
f(x)=15x^{20}-3x^{5}+2x+4.
\end{equation*}
We calculate its derivative as follows:
\begin{align*}
\frac{\ud f}{\ud x}	&=\frac{\ud}{\ud x}\left(15x^{20}-2x^{5}+2x+4\right)\\
			&=15\frac{\ud}{\ud x}\left(x^{20}\right)-2\frac{\ud}{\ud x}\left(x^{5}\right)+2\frac{\ud}{\ud x}\left(x\right)+\frac{\ud}{\ud x}\left(4\right)\\
			&=15 \times 20 x^{19}-2\times 5 x^{4}+2 +0\\
			&=300 x^{19}-10x^{4}+2.
\end{align*}
\end{ex}

\begin{ex}
Consider the function 
\begin{equation*}
g(x)=\frac{6}{x^{2}}-4x^{2}+2x.
\end{equation*}
Its derivative is calculated as follows 
\begin{align*}
\frac{\ud g}{\ud x}	&=\frac{\ud}{\ud x}\left(\frac{6}{x^{2}}-4x^{2}+2x\right)\\
			&=6\frac{\ud}{\ud x}\left(x^{-2}\right)-4\frac{\ud}{\ud x}\left(x^{4}\right)+2\frac{\ud }{\ud x}\left(x\right)\\
			&=6\times (-2) x^{-3}-4\times 4 x^{3}+2\\
			&=-\frac{12}{x^{3}}-16x^{3}+2.
\end{align*}
\end{ex}

\begin{exercise}
Calculate the derivative of 
\begin{equation*}
y=8x^{2}+2x-\frac{1}{x}.
\end{equation*}
\end{exercise}

\subsection*{Product rule}
So far we have not discussed differentiating the product of two functions, unless you count $x^{a+b}=x^{a}x^{b}$. A very useful formula is the \textbf{Leibniz} or \textbf{product rule} which tells us how to differentiate the product of two functions. The product rule is
\begin{equation}
\frac{\ud }{\ud x}\left(f(x)g(x)\right)=\frac{\ud f}{\ud x}g(x)+f(x)\frac{\ud g}{\ud x}. \label{eq: product rule}
\end{equation}

You may think that it is disappointing that the derivative of a product is not just the product of the derivatives. However, you will come to appreciate the product rule as you make use of it. It will become particularly useful once we have discussed more about how to differentiate trig functions and exponentials. 

\begin{ex}
For the product of two functions $\sqrt{x^{3}}\sin(x)$ we find the derivative as follows. Let $f(x)=\sqrt{x^{3}}$ and $g(x)=\sin(x)$ and calculate the individual derivatives to be
\begin{equation*}
f'(x)=(x^{\frac{3}{2}})'=\frac{3}{2}x^{\frac{3}{2}-1}=\frac{3}{2}\sqrt{x}, \qquad g'(x)=(\sin(x))'=\cos(x).
\end{equation*}
Then we use product rule to calculate
\begin{align*}
\left(\sqrt{x^{3}}\sin(x)\right)'	&=\left(f(x)g(x)\right)'\\
					&=f'(x)g(x)+f(x)g'(x)\\
					&=\frac{3}{2}\sqrt{x}\sin(x)+\sqrt{x^{3}}\cos(x).
\end{align*}
\end{ex}

\begin{ex}
We can use the product rule to calculate the derivative of $f(x)=\sin^{2}(x)$. In this case the two functions are the same so we can evaluate the derivative as follows
\begin{align*}
f'(x)	&=\left(\sin^{2}(x)\right)'\\
	&=2\sin(x)\left(\sin(x)\right)'\\
	&=2\sin(x)\cos(x)\\
	&=\cos(2x),
\end{align*}
where we have used one of the trig double angle identities in the last line.
\end{ex}

\begin{exercise}
Use the product rule to calculate the derivative of 
\begin{equation*}
f(x)g(x)=e^{x}\sin(x).
\end{equation*}
\end{exercise}

Note that the product rule applies if we have the product of more than two functions, we just have to iterate it for each pair of functions.

\subsection*{Quotient rule}
If instead of a product of functions we have the ratio of two functions then there is a rule for that, the \textbf{quotient} rule\footnote{You may be looking at these two formula and thinking that we could just use the product rule with $f(x)$ and $(g(x))^{-1}$. If you do this you will get an answer that looks just like the quotient rule. There are some technical differences between this approach and the quotient rule we give here, but as we are not mathematicians here we do not need to worry about them. }. In Leibniz notation the quotient rule is 
\begin{equation}
\frac{\ud }{\ud x}\left(\frac{f(x)}{g(x)}\right)=\frac{\frac{\ud f}{\ud x}g(x)-f(x)\frac{\ud g}{\ud x}}{g^{2}(x)}. \label{eq: quotient rule}
\end{equation}

Remember that in most circumstances you can use either the product rule or the quotient rule, it depends how you want to approach the problem.\\

\begin{ex}
Consider the function
\begin{equation*}
F(x)=\frac{\sin(x)}{e^{x}}.
\end{equation*}
Note that we could use the product rule on $e^{-x}\sin(x)$ but will instead use the quotient rule here. Spilt $F(x)$ into the two functions $f(x)=\sin(x)$ and $g(x)=e^{x}$, applying the quotient rule \cref{eq: quotient rule} gives
\begin{align*}
\frac{\ud }{\ud x}\left(\frac{f(x)}{g(x)}\right)&=\frac{\frac{\ud f}{\ud x}g(x)-f(x)\frac{\ud g}{\ud x}}{g^{2}(x)}\\
							&=\frac{e^{x}\cos(x)-\sin(x)e^{x}}{e^{2x}}\\
							&=e^{-x}\left(\cos(x)-\sin(x)\right).
\end{align*}
\end{ex}

\begin{ex}
Consider 
\begin{equation*}
F(x)=\frac{1}{\sin^{2}(x)},
\end{equation*}
and split it into two functions $f(x)=1$ and $g(x)=\sin^{2}(x)$, which have derivatives
\begin{equation*}
\frac{\ud f}{\ud x}=0, \qquad \frac{\ud g}{\ud x}=2\sin(x)\cos(x)=\sin(2x).
\end{equation*}
The quotient rule thus gives that
\begin{align*}
\frac{\ud }{\ud x}\left(\frac{f(x)}{g(x)}\right)&=\frac{\frac{\ud f}{\ud x}g(x)-f(x)\frac{\ud g}{\ud x}}{g^{2}(x)}\\
							&=\frac{0-2\sin(x)\cos(x)}{\sin^{4}(x)}\\
							&=-2\frac{\cot(x)}{\sin^{2}(x)}\\
							&=-2\cot(x)\csc^{2}(x).
\end{align*}
\end{ex}

\subsection*{Derivatives of trig functions}
Armed with the quotient rule we can now give the derivatives of the six trig functions
\begin{align}
\frac{\ud}{\ud x}\sin(ax)&=a\cos(ax),\label{eq: sine derivative}\\
\frac{\ud}{\ud x}\cos(ax)&=-a\sin(ax),\label{eq: cos derivative}\\
\frac{\ud}{\ud x}\tan(ax)&=a\sec^{2}(ax),\label{eq: tan derivative}\\
\frac{\ud}{\ud x}\cot(ax)&=-a\csc^{2}(ax),\label{eq: cot derivative}\\
\frac{\ud}{\ud x}\sec(ax)&=a\sec(ax)\tan(ax), \label{eq: sec derivative}\\
\frac{\ud}{\ud x}\csc(ax)&=-a\csc(ax)\cot(ax).\label{eq: csc derivative}
\end{align}

We have seen how to prove two of these and can now calculate the derivative of $\tan(x)$ the other three will be left as exercises.
\begin{ex}
Consider $f(x)=\tan(x)$, which can be expressed as
\begin{equation*}
f(x)=\tan(x)=\frac{\sin(x)}{\cos(x)}.
\end{equation*}
Using the quotient rule we have that
\begin{align*}
\frac{\ud}{\ud x}\tan(x)	&=\frac{\ud }{\ud x}\left(\frac{\sin(x)}{\cos(x)}\right)\\
				&=\frac{1}{(\cos(x))^{2}}\left(\cos(x)\frac{\ud }{\ud x}\sin(x)-\sin(x)\frac{\ud }{\ud x}\cos(x)\right)\\
				&=\frac{1}{\cos^{2}(x)}\left(\cos(x)\cos(x)-\sin(x)(-\sin(x))\right)\\
				&=\frac{1}{\cos^{2}(x)}\left(\cos^{2}(x)+\sin^{2}(x)\right)\\
				&=\frac{1}{\cos^{2}(x)}\\
				&=\sec^{2}(x).
\end{align*}
Where we have used the identity that $\cos^{2}(x)+\sin^{2}(x)=1$ and the definition of $\sec(x)=1/\cos(x)$.
\end{ex}

\begin{exercise}
Compute the derivatives of $\cot(x), \sec(x),\csc(x)$.
\end{exercise}

\subsection*{Chain rule}
If we have a function of a function, e.g. $f(x)=\exp\left(x^{2}+x\right)$ or $g(x)=\cos\left(x+c\right)$, none of the rules that we have given so far will work. We could go back to first principles, which is exactly what we did when calculating the derivative of $1/(x+1)$, but this would mean that we had to do lots of long and tricky calculations. Fortunately this is not necessary.\\

Consider the function
\begin{equation}
f(x)=\sqrt{4x+2},
\label{eq: function of a function1}
\end{equation}
we can write this as the \textbf{composition} of two functions if we think of $g(x)=\sqrt{x}$ and $h(x)=4x+2$,
\begin{equation*}
f(x)=\left(g\circ h\right)(x)=g(h(x))=g\left(4x+2\right)=\sqrt{4x+2}.
\end{equation*} 

The \textbf{chain rule} is the, fairly, simple method to differentiate such compositions of functions.  As long as both functions are differentiable, we have that if $f(x)=(g\circ h)(x)$ then
\begin{equation}
f'(x)=g'(h(x))g'(x). \label{eq: chain rule 1}
\end{equation}

An alternative way of writing the chain rule, that is easier to understand in Leibniz notation, works when $y=f(u)$ and $u=g(x)$ then
\begin{equation}
\frac{\ud y}{\ud x}=\frac{\ud y}{\ud u}\frac{\ud u}{\ud x}. \label{eq: chain rule 2}
\end{equation}
This second way of phrasing the chain rule makes it look like we are treating the derivative like a fraction, we are not, it is just a coincidence that the formula looks like this.\\


Armed with the chain rule, \cref{eq: chain rule 1} we can return to \cref{eq: function of a function1}  and calculate its derivative
\begin{ex}
Consider the function
\begin{equation*}
f(x)=\sqrt{4x+2},
\end{equation*}
and let $g(x)=\sqrt{x}$ and $h(x)=4x+2$. The rules we already know about differentiation tell us that
\begin{equation*}
g'(x)=\frac{1}{2\sqrt{x}}, \qquad h'(x)=4.
\end{equation*}
Thus applying the chain rule gives
\begin{align*}
f'(x)	&=g'(h(x))h'(x)\\
	&=g'(4x+2)(4)\\
	&=\frac{4}{2\sqrt{4x+2}}\\
	&=\frac{2}{\sqrt{4x+2}}.
\end{align*}
\end{ex}

While we have kept tract of the function composition here, this is not how we proceed in general, particularly since this can become very complicated if we have a matryoshka doll like set up with many nested functions. Usually we will just think about an \textit{inside} and \textit{outside} function and then differentiate the \textit{outside} function using the rules that we already know for powers, trig, exponential, or logarithmic functions, then multiply this by the derivative of the \textit{inside} function. \\

This may still sound quite complicated, and as is always the case in mathematics, the best way to get to grips with the concept is by solving lots of examples.

\begin{ex}
Consider the function
\begin{equation*}
g(x)=\cos\left(x+c\right).
\end{equation*}

Its derivative is given by
\begin{align*}
g'(x)	&=-\sin\left(x+c\right)\left(x+c\right)'\\
	&=-\sin\left(x+c\right).
\end{align*}
\end{ex}

\begin{ex}
Consider the function 
\begin{equation*}
f(x)=\left(2x^{2}+\cos(x)\right)^{2}.
\end{equation*}
Its derivative is given by
\begin{align*}
f'(x)	&=2\left(2x^{2}+\cos(x)\right)\left(2x^{2}+\cos(x)\right)'\\
	&=2\left(2x^{2}+\cos(x)\right)\left(4x-\sin(x)\right)
\end{align*}
\end{ex}

\begin{exercise}
Calculate the derivative of
\begin{equation*}
 f(x)=\exp\left(x^{2}+x\right).
\end{equation*}
\end{exercise}

\section{Multiple derivatives}
So far we have only talked about calculating the derivative once. However, for a function $f(x)$ its derivative is also a function $f'(x)$ which we could differentiate again. The notation for taking the second derivative is
\begin{equation*}
f''(x)=\frac{\ud^{2}f}{\ud x^{2}}.
\end{equation*}
We can keep doing this and the notation for the $n'$th derivative is
\begin{equation*}
f^{(n)}=\frac{\ud^{n}f}{\ud x^{n}}.
\end{equation*}

\begin{ex}
Consider the function $f(x)=2x^{3}+4x$, its second derivative is
\begin{align*}
\frac{\ud^{2}f}{\ud x^{2}}	&=\frac{\ud}{\ud x}\left(\frac{\ud f}{\ud x}\right)\\
					&=\frac{\ud}{\ud x}\left(6x^{2}+4\right)\\
					&=12x.
\end{align*}
The third derivative is
\begin{align*}
\frac{\ud^{3}f}{\ud x^{3}}	&=\frac{\ud}{\ud x}\left(\frac{\ud^{2}f}{\ud x^{2}}\right)\\
					&=\frac{\ud}{\ud x}\left(12x\right)\\
					&=12.
\end{align*}
\end{ex}

We will not do much with second, or higher, derivatives, other than when discussing optimisation problems in \cref{sec:advanced topics}. However, you may come across them in your future studies so it is worth having some exposure to them.

\section{Applications of differentiation}
\label{sec: apps of diff}
So, now we have met the derivative and learnt some rules for using it. Some of you reading this are likely to be saying ``Ok, we can calculate derivatives. So what\textinterrobang'' Well, now we will look at a few applications of differentiation.

\subsection*{Finding critical points}
A critical point of a function $f(x)$ is a point $a$ where the derivative,  $f'(a)$, vanishes. Sometimes critical points are called \textbf{stationary points}\footnote{This is because of the main examples of a first derivative is speed being the derivative of the position in mechanics. In this case a vanishing derivative means that the speed is zero so the object is stationary.}.

\begin{mdiv}
Strictly speaking there are a couple of extra points to consider. First we need that at the point $a$ we have that $f(a)$ exists. Then $a$ is a critical point if either $f'(a)=0$ or $f'(a)$ does not exist. In \citep{calcI} there is a discussion of this emphasising the fact that the point $a$ needs to be in the domain of the function $f$.
\end{mdiv}

If we have a function like $f(x)=4x^{2}+3x-2$ we can ask, does it have critical points, if so what are they and what does the function look like near them. The way that we answer the first two parts of this is to calculate the the derivative of the function, set it to zero, and then solve the algebraic equation that we get. In other words, finding a critical point boils down to solving an algebraic equation. The easiest way to understand this is by considering an explicit example. \\

\begin{figure}[ht]
    \centering
\ThisAltText{Graph of a quadratic with its critical point marked.}
 %   \pdftooltip{
    \begin{tikzpicture}[line width=1pt,line cap=round,line join=round, smooth,variable=\x, yscale = 1, xscale = 3]
     \draw[->] (-1.5,0) -- (0.8,0) node[above] {$x$};
  \draw[->] (0,-3) -- (0,3.3)node[above]{$y$};
 \draw[color=CDnavy, domain=-1.45:0.7]   plot[samples=300] (\x,{(4*\x*\x + 3*\x -2}) node[right] {$f(x)=4x^{2}+3x-2$};
  \draw[color=CDgreen, domain=-0.8:0.1]   plot[samples=300] (\x,{(-2.5625}) node[right] {$f'(x)$};
\filldraw[black] (0.5,0) circle (0.5pt) node[anchor=north]{$0.5$};
\filldraw[black] (-0.5,0) circle (0.5pt) node[anchor=north]{$-0.5$};
\filldraw[black] (-1,0) circle (0.5pt) node[anchor=north]{$-1$};
\filldraw[black] (-1.5,0) circle (0.5pt) node[anchor=north]{$-1.5$};
\filldraw[black] (0,-3) circle (0.5pt) node[anchor=east]{$-3$};
\filldraw[black] (0,-2) circle (0.5pt) node[anchor=east]{$-2$};
\filldraw[black] (0,-1) circle (0.5pt) node[anchor=east]{$-1$};
\filldraw[black] (0,1) circle (0.5pt) node[anchor=east]{$1$};
\filldraw[black] (0,2) circle (0.5pt) node[anchor=east]{$2$};
\filldraw[black] (0,3) circle (0.5pt) node[anchor=east]{$3$};
\filldraw[CDred] (-0.375,-2.5625) circle (0.5pt) node[below]{$f'(x)=0$};
    \end{tikzpicture}
%    }{plots of an exponential and logarithm }
    \caption{A plot of the function $f(x)=4x^{2}+3x-2$ with its critical point marked.}
        \label{fig: crit points 1}
\end{figure}

\begin{ex}
Consider the function
\begin{equation*}
f(x)=4x^{2}+3x-2.
\end{equation*}

We find its critical points by calculating $f'(x)$ and setting it to zero. The derivative is
\begin{align*}
\frac{\ud f}{\ud x}	&=\frac{\ud}{\ud x}\left(4x^{2}+3x-2\right)\\
			&=8x+3.
\end{align*}
Setting this equal to zero gives 
\begin{align*}
8x+3&=0\\
x=-\frac{3}{8}.
\end{align*}

This polynomial is plotted in \cref{fig: crit points 1} with its critical point marked and its derivative plotted in green. Notice that the critical point corresponds to the minimum of the function.
\end{ex}

This is a general observation, critical points are related to where a function changes direction or stops moving. One way to understand this is that the derivative measures how ``fast'' a function is changing, it is the \textbf{rate of change} of the function. \\

As a check to see that this makes sense consider the plots of sine and cosine in \cref{fig: trig functions}. Notice that when sine has its maxima and minima cosine is zero and the other way around. Recall that cosine is the derivative of sine, and that the derivative of cosine is minus sine, so the maxima and minima are critical points.\\

We need to be aware that not all critical points are maxima or minima. Even if the function does not change direction at a point, its derivative can still vanish there. These are called \textbf{points of inflection}, sometimes they are called \textbf{saddle points} as for functions of more than one variable they often have the shape of a saddle. See \cref{fig: crit points 2} for an example of a point of inflection.\\

\begin{figure}[ht]
    \centering
\ThisAltText{Graph of a cubic with a point of inflection.}
 %   \pdftooltip{
    \begin{tikzpicture}[line width=1pt,line cap=round,line join=round, smooth,variable=\x, yscale = 1, xscale = 3]
     \draw[->] (-1.6,0) -- (1.6,0) node[above] {$x$};
  \draw[->] (0,-3.5) -- (0,3.5)node[above]{$y$};
 \draw[color=CDnavy, domain=-1.5:1.5]   plot[samples=300] (\x,{(\x*\x*\x }) node[right] {$f(x)=x^{3}$};
 % \draw[color=CDgreen, domain=-0.8:0.1]   plot[samples=300] (\x,{(-2.5625}) node[right] {$f'(x)$};
\filldraw[black] (0.5,0) circle (0.5pt) node[anchor=north]{$0.5$};
\filldraw[black] (-0.5,0) circle (0.5pt) node[anchor=north]{$-0.5$};
\filldraw[black] (-1,0) circle (0.5pt) node[anchor=north]{$-1$};
\filldraw[black] (1,0) circle (0.5pt) node[anchor=north]{$1$};
\filldraw[black] (0,-3) circle (0.5pt) node[anchor=east]{$-3$};
\filldraw[black] (0,-2) circle (0.5pt) node[anchor=east]{$-2$};
\filldraw[black] (0,-1) circle (0.5pt) node[anchor=east]{$-1$};
\filldraw[black] (0,1) circle (0.5pt) node[anchor=east]{$1$};
\filldraw[black] (0,2) circle (0.5pt) node[anchor=east]{$2$};
\filldraw[black] (0,3) circle (0.5pt) node[anchor=east]{$3$};
\filldraw[CDred] (0,0) circle (0.5pt) node[above]{$f'(x)=0$};
    \end{tikzpicture}
%    }{plots of an exponential and logarithm }
    \caption{A plot of the function $f(x)=x^{3}$. The function has a critical point at $x=0$ but it does not change direction there. }
        \label{fig: crit points 2}
\end{figure}

The other point to be aware of is that we are detecting critical, which can be \textbf{local} maxima or minima rather than just the \textbf{global} maxima and minima. For example if $f(x)=x^{3}/2 +4x^{2}/5$, which is plotted in \cref{fig: crit points 3}, then we find the critical points as follows:
\begin{align*}
0	&=f'(x)\\
	&=\frac{3}{2}x^{2}+\frac{8}{5}x\\
	&=x^{2}+\frac{16}{15}x.
\end{align*}
This is a quadratic equation which we solve either by using the quadratic formula or by extracting a common factor as follows,
\begin{align*}
0	&=x^{2}+\frac{16}{15}x\\
	&=x\left(x+\frac{16}{15}\right),
\end{align*}
so the critical points are at $x=0$ and $x=-16/15\simeq -1.067$. By inspecting the graph we see that $x=0$ is a local minima and $x=-16/15$ is a local maxima. They are local since for $x<-1.6$ $f(x)$ is below $f(x)$, and similarly for positive $x$ $f(x)$ reaches values larger than at $x=-16/15$. The true maxima and minima are only reached asymptotically.\\

\begin{figure}[ht]
    \centering
\ThisAltText{Graph of a cubic with its critical points marked.}
 %   \pdftooltip{
    \begin{tikzpicture}[line width=1pt,line cap=round,line join=round, smooth,variable=\x, yscale = 3, xscale = 3]
     \draw[->] (-2,0) -- (1,0) node[above] {$x$};
  \draw[->] (0,-0.5) -- (0,1)node[above]{$y$};
 \draw[color=CDnavy, domain=-2:1]   plot[samples=300] (\x,{(0.5*\x*\x*\x+0.8*\x*\x }) node[right] {$f(x)=\frac{1}{2}x^{3} +\frac{4}{5}x^{2}$};
 % \draw[color=CDgreen, domain=-0.8:0.1]   plot[samples=300] (\x,{(-2.5625}) node[right] {$f'(x)$};
\filldraw[black] (0.4,0) circle (0.5pt) node[anchor=north]{$0.4$};
\filldraw[black] (-0.4,0) circle (0.5pt) node[anchor=north]{$-0.4$};
\filldraw[black] (-0.8,0) circle (0.5pt) node[anchor=north]{$-0.8$};
\filldraw[black] (-1.2,0) circle (0.5pt) node[anchor=north]{$-1.2$};
\filldraw[black] (-1.6,0) circle (0.5pt) node[anchor=north]{$-1.6$};
\filldraw[black] (0.8,0) circle (0.5pt) node[anchor=north]{$0.8$};
\filldraw[black] (0,-0.4) circle (0.5pt) node[anchor=east]{$-0.4$};
\filldraw[black] (0,0.8) circle (0.5pt) node[anchor=east]{$0.8$};
\filldraw[black] (0,0.4) circle (0.5pt) node[anchor=east]{$0.4$};

\filldraw[CDred] (0,0) circle (0.5pt) node[above]{$f'(x)=0$};
\filldraw[CDred] (-1.067,0.303) circle (0.5pt) node[above]{$f'(x)=0$};
    \end{tikzpicture}
%    }{plots of an exponential and logarithm }
    \caption{A plot of the function $f(x)=x^{3}/2 +4x^{2}/5$. The function has a local minima at $x=0$ but the global minima is at $x\to -\infty$. }
        \label{fig: crit points 3}
\end{figure}

Above we worked out whether a critical point was a maxima or a minima by consulting the graph. However, we can do this systematically without sketching the graph, which is handy since maxima and minima are useful to know before sketching a function. There are two ways to do this:

\begin{enumerate}
%\setlength{\itemsep}{-5pt}
    \item After finding a critical point $a$ substitute it into the function to find its value $f(a)$, then pick two values of $x$, one just less than $a$, $a_{-}$ and the other just bigger than $a$, $a_{+}$. Then substitute these into $f(a)$.
\begin{itemize}
%\setlength{\itemsep}{-5pt}
   	\item If $f(a)$ is larger than both $f(a_{-})$ and $f(a_{+})$ then $a$ is a local maxima.
	\item If $f(a)$ is lower than both $f(a_{-})$ and $f(a_{+})$ the $a$ is a local minima.
	\item If we find that $f(a_{-})>f(a)> f(a_{+})$ or $f(a_{+})>f(a)> f(a_{-})$ then $a$ is a point of inflection.
\end{itemize}
\item The other approach involves calculating the second derivative. If the derivative tells you about the rate of change of the function, then the second derivative tells you about the rate of change of the derivative.
\begin{itemize}
%\setlength{\itemsep}{-5pt}
   	\item If the derivative is decreasing at a critical point $a$, e.g. $f''(a)<0$ then the point is a maxima.
	\item If $f''(a)>0$ is the $a$ is a local minima.
	\item If $f''(a)=0$ then $a$ could be a a maxima, minima, or point of inflection and we need to examine the sign of $f'(x)$ on either side of $a$.
\end{itemize} 
\end{enumerate}

In this module you can follow whichever approach you want if you are asked to identify the critical values.

\begin{ex}
Consider the function
\begin{equation*}
f(x)=x^{3}-3x,
\end{equation*}
we find and classify the critical points as follows. For the critical points we find the derivative
\begin{equation*}
f'(x)=\left(x^{3}-3x\right)^{'}=3x^{2}-3,
\end{equation*}
and set it to zero,
\begin{align*}
0&=3x^{2}-3\\
	&=x^{2}-1,\\
x^{2}=1,\\
x&=\pm 1.
\end{align*}
So there are two critical points. To find the nature of the critical points we need the second derivative
\begin{equation*}
f''(x)=\left(3x^{2}-3\right)^{'}=6x.
\end{equation*}
At the critical points this is
\begin{align*}
f''(1)&=6,\\
f''(-1)&=-6.
\end{align*}
So $f''(1)$ is positive meaning that $x=1$ is a minimum, while $f''(-1)$ is positive so $x=-1$ is a maxima. When we plot the graph in \cref{fig: crit points 4} we see that these are only local minima and maxima.
\end{ex}
\begin{figure}[ht]
    \centering
\ThisAltText{Graph of a cubic with its critical points marked.}
 %   \pdftooltip{
    \begin{tikzpicture}[line width=1pt,line cap=round,line join=round, smooth,variable=\x]
     \draw[->] (-2.2,0) -- (2.2,0) node[above] {$x$};
  \draw[->] (0,-2.2) -- (0,2.2)node[above]{$y$};
 \draw[color=CDnavy, domain=-2:2]   plot[samples=300] (\x,{(\x*\x*\x-3*\x }) node[right] {$f(x)=x^{3} -3x$};
\filldraw[black] (1,0) circle (1pt) node[anchor=north]{$1$};
\filldraw[black] (-1,0) circle (1pt) node[anchor=north]{$-1$};
\filldraw[black] (-2,0) circle (1pt) node[anchor=north]{$-2$};
\filldraw[black] (2,0) circle (1pt) node[anchor=north]{$2$};
\filldraw[black] (0,-2) circle (1pt) node[anchor=east]{$-2$};
\filldraw[black] (0,-1) circle (1pt) node[anchor=east]{$-1$};
\filldraw[black] (0,1) circle (1pt) node[anchor=east]{$1$};
\filldraw[black] (0,2) circle (1pt) node[anchor=east]{$2$};
\filldraw[CDred] (-1,2) circle (0.5pt) node[above]{$f'(x)=0$};
\filldraw[CDred] (1,-2) circle (0.5pt) node[below]{$f'(x)=0$};
    \end{tikzpicture}
%    }{plots of an exponential and logarithm }
    \caption{A plot of the function $f(x)=x^{3}-3x$ with its local minima and maxima marked. }
        \label{fig: crit points 4}
\end{figure}

\begin{exercise}
Find and classify the critical points of 
\begin{equation*}
f(x)=x^{4}-3x^{2}+2.
\end{equation*}
\end{exercise}



\subsection*{Linear approximations}
Now we are going to look at how to approximate functions using their tangent lines. This is called the \textbf{linear approximation} to the function, and can be good near the point that the line is tangent to. However, the approximation will get worse and worse the further away we go from the point\footnote{We could make the approximation better by using higher order derivative terms, but the approximation would no longer be linear and would now be via a polynomial. Considering this would lead us to the \textbf{Taylor} or \textbf{Maclaurin series} of a function. You may have met this in A-level maths if not we may meet this in \cref{sec:advanced topics} depending on which of the advanced topics we have time for. }.  There is a nice, if brief, discussion of linear approximation in \citep{calcI} which we draw on for the discussion here.\\

For a function $f(x)$, its tangent line at $x=a$ is given by the line
\begin{equation}
L(x)=f(a)+f'(a)\left(x-a\right).
\label{eq: linear approximation}
\end{equation}
For $f(x)=x^{2}/2+1/2$ the function and its linear approximation at $x=1$ are plotted in \cref{fig: linear approximation}. We see that near the point $x=1$ the tangent line is a good approximation to the full function as the graphs almost lie on top of each other. \\


\begin{figure}[ht]
    \centering
\ThisAltText{Graph of a quadratic with its linear approximation at x=1 marked.}
 %   \pdftooltip{
    \begin{tikzpicture}[line width=1pt,line cap=round,line join=round, smooth,variable=\x, scale =2]
     \draw[->] (-2.2,0) -- (2.2,0) node[above] {$x$};
  \draw[->] (0,0) -- (0,2.6)node[above]{$y$};
 \draw[color=CDnavy, domain=-2:2]   plot[samples=300] (\x,{(0.5*\x*\x+0.5 }) node[right] {$f(x)=\frac{1}{2}x^{2}+\frac{1}{2}$};
 \draw[color=CDgreen, domain=0:2]   plot[samples=300] (\x,{(\x }) node[right] {$L(x)=x$};
\filldraw[black] (1,0) circle (1pt) node[anchor=north]{$1$};
\filldraw[black] (-1,0) circle (1pt) node[anchor=north]{$-1$};
\filldraw[black] (-2,0) circle (1pt) node[anchor=north]{$-2$};
\filldraw[black] (2,0) circle (1pt) node[anchor=north]{$2$};
\filldraw[black] (0,1) circle (1pt) node[anchor=east]{$1$};
\filldraw[black] (0,2) circle (1pt) node[anchor=east]{$2$};
\filldraw[CDred] (1,1) circle (1pt) node[below, xshift=7pt]{$(1,1)$};
    \end{tikzpicture}
%    }{plots of an exponential and logarithm }
    \caption{A plot of the function $f(x)=x^{2}/2+1/2$ with its linear approximation at $x=1$ marked in red. }
        \label{fig: linear approximation}
\end{figure}


This means that if we want to find the values of $f(x)$ near the point $a$ we can use \cref{eq: linear approximation} instead of the full function.  You may think that this does not seem very useful since we can fairly easily work out the value of $f(x)$ for many of the functions that we have considered in this module. The main advantage comes when we have to work out lots of values of a function, such as if we are simulating something on a computer. Every time the function is evaluated this takes time and it is much quicker to use an approximation than to use the full function. \\

\begin{ex}
Consider $f(x)=\sin(x)$, its linear approximation at $x=0$ is given by \cref{eq: linear approximation}. To calculate this we need that
\begin{equation*}
f'(x)=\cos(x),
\end{equation*}
so we can calculate $f'(0)=1$. This means that \cref{eq: linear approximation} becomes
\begin{align*}
L(x)	&=\sin(0)+\cos(0)\left(x-0\right)\\
	&=0+1\left(x\right)\\
	&=0.
\end{align*}
So for small angles we can use the approximation that $\sin(x)\simeq x$. This is an incredibly useful approximation, particularly if you are interested in studying the physics of pendulums. 
\end{ex}

If we repeated the above calculation for $\cos(x)$ we would find that $L(x)=1$ at $x=0$. 

%\subsection*{$\star$ Mechanics $\star$}
%This is a non examinable application as it goes beyond the scope of this module. The original motivation for calculus, at least with regards to Newton, came from physics.  \textbf{Mechanics} is the study of moving objects, if you did maths or physics at A-level then you may have come across some of the ideas. It is all about how far an object is travelling, at what speed, and understanding if and how the speed is changing. If you want a refresher on these concepts then you can look at the lecture notes for the STEM foundation physics module that I teach \citep{STM0005}.\\
%
%Now that you have been exposed to calculus that hopefully makes you think that the derivatives will be useful here. In fact, mechanics is secretly all about calculus, both the derivatives that we have already met and the integrals that we will discuss soon. \\
% 




\newpage

%%%%%%%%%%%%%%%%%%%%%%%%%%%%%%%%%%%%%%%%%%%%%%

\chapter{Integration}
\label{sec:integration}
\epigraph{\textbf{Integration}: The inverse process to differentiation, i.e. the process of finding a function with a derivative that is a given function. }{\textit{Penguin Dictionary of Mathematics}}

\section{Integration and antiderivatives}
As the above quote from the \textit{Penguin Dictionary of Mathematics} says, integration is the reverse process to differentiation. If differentiation is all about finding the rates of change of functions then we an think of integration as being about the areas under curves.  As with differentiation it is a process that was discovered to handle problems from physics so you can find plenty of examples of applications by looking into topics from mechanics. We may meet some of those at the end of this chapter, but for now we will focus on the mathematics.\\

\begin{figure}[htbp]
    \centering
\ThisAltText{Graph of sin(x).}
  %  \pdftooltip{
    \begin{tikzpicture}[line width=1pt,line cap=round,line join=round,
    %domain=-0:4.712, 
    smooth,variable=\x
    ]
    \begin{axis}[
    axis lines = middle,
    xlabel = {$x$},
    ylabel = {$y$},
    xmin=0, xmax=4.8,
    ymin=-1.1, ymax=1.1,
   xtick={1.57,3.14,4.712},
   xticklabels={$\frac{\uppi}{2}$,$\uppi$,$\frac{3\uppi}{2}$}, ]
    \addplot [CDnavy,name path=A,domain=0:4.712]
        {sin(\x r)};
 
    \addplot [black, name path=B,domain=0:4.712,samples=2]
        {0};
 
    \addplot [CDred] fill between [of=A and B];
     \filldraw[black] (1.57,0.5) circle (0pt) node[anchor=north]{$A_{1}$};
  \filldraw[black] (4,-0.4) circle (0pt) node[anchor=north]{$A_{2}$};
\end{axis}
%     \draw[->] (0,0) -- (4.8,0)node[above] {$x$};
%  \draw[->] (0,-1.2) -- (0,1.2) node[above] {$y$};
% \draw[color=CDnavy]   plot (\x,{sin(\x r)}) node[right] {$f(x)=\sin(x)$};
% 
% \begin{axis}[
%    axis lines = middle,
%    xlabel = {$x$},
%    ylabel = {$y$},
%    xmin=0, xmax=4.8,
%    ymin=-1.1, ymax=1.1]
% 
%% Plot 1
%\addplot [name path = A,
%    -latex,
%    domain = 0:4.712,
%    samples = 1000] {sin(x)} 
%    node [very near end, right] {$y=\sin(x)$};
% 
%% Plot 2
%\addplot [name path = B,
%    -latex,
%    domain = 0:4.712] {0} ;
%
%% Fill area between paths
%\addplot [CDnavy] fill between [of = A and B, soft clip={domain=0:3.14}];
%\addplot [CDred] fill between [of = A and B, soft clip={domain=3.14:4.712}];
% \end{axis} 

    \end{tikzpicture}
  %  }{graph of sin}
    \caption{The graph of the sine function  $f(x)=\sin(x)$ with the area between the curve and the $x$-axis between $x=0$ and $x=3\uppi/2$ shaded. Note that the total shaded area is $A=A_{1}+A_{2}$.}
        \label{fig: sine function graph shaded}
\end{figure}


The integral of a function over an interval is a measure of the signed area between the curve and the $x$-axis, with the convention that areas above the $x$-axis are positive and areas under the $x$-axis are negative.  This is shown for $f(x)=\sin(x)$ between $0$ and $3\uppi/2$ in \cref{fig: sine function graph shaded}. When we discuss numerical methods of integration in \cref{sec:numerics} we will understand how to approximate these areas using thin rectangles as is shown in \cref{fig: integral approximation}. \\

For now we will approach integration as the reverse of differentiation. In other words, if we have a function $f(x)$ we want to find the function whose derivative is $f(x)$.

\begin{ex}
If we have the function
\begin{equation*}
f(x)=x^{4}+2x^{2}+2
\end{equation*}
what function is this the derivative of?\\

To find this we need to remember our rules for differentiation, in particular the monomial rule of \cref{eq: monomial derivative}, which says that  we need to find an $x^{n}, x^{m}$ and an $x^{p}$ such that
\begin{align*}
\frac{\ud (ax^{n})}{\ud x}&=anx^{n-1}=x^{4},\\
\frac{\ud (bx^{m})}{\ud x}&=bmx^{m-1}=2x^{2},\\
\frac{\ud (cx^{p})}{\ud x}&=cpx^{p-1}=2.
\end{align*}

By observation we find that $n=5, a=1/5$, $m=3, b=2/3$, $p=1$ and $c=2$. Thus if we have the function
\begin{equation*}
F(x)=\frac{1}{5}x^{5}+\frac{2}{3}x^{3}+2x,
\end{equation*}
its derivative is $F'(x)=f(x)$.\\

This looks like we are done. However, recall that in \cref{eq: derivative of constant} we said that the derivative of a constant is zero. Thus we could add any constant to $F(x)$ and still have a function whose derivative is $f(x)$. This means that we should write
\begin{equation*}
F(x)=\frac{1}{5}x^{5}+\frac{2}{3}x^{3}+2x+c,
\end{equation*}
where $c$ can be any constant such as $0, 10, \uppi, 11/25, -2, \dots$ , when we have our brief discussion in \cref{sec:diff eqs}, we will see that often there are conditions that we need to apply which determine the constant.  
\end{ex}

Notice that since the constant in the example is arbitrary there are actually an infinite number of $F(x)$'s such that their derivative is the given $f(x)$. We call such functions $F(x)$ \textbf{antiderivatives} of $f(x)$.\\

As an aside, you might be wondering why a constant shows up here, but we did not mention anything about a constant when giving the intuition of an integral as the area under a curve. Surely the area is a fixed concept and cannot be arbitrarily changed by a constant? This is true. The difference is because when we talked about the area under a curve, we specified a region of the $x$-axis that we were interested in.\\

This leads to what is called a \textbf{definite integral} which we will study in more detail later. When we are constructing an antiderivative like $F(x)$, we are not specifying any limits which is why we need to add the constant. You will gain more experience with this as we look at some more examples.\\

In general we say that given a function $f(x)$, an antiderivative of $f(x)$ is any function $F(x)$ such that 
\begin{equation}
\frac{\ud F}{\ud x}=f(x).
\label{eq: antiderivative}
\end{equation}

When we are explicitly including the arbitrary constant $c$ we refer to the \textbf{indefinite integral} of $f(x)$ denoted by
\begin{equation}
\int f(x) \ud x=F(x)+c.
\label{eq: indefinite integral}
\end{equation}

The symbol $\int$ is the \textbf{integral} or \textbf{integration} symbol and in the above expression $f(x)$ is called the \textbf{integrand}, the function being integrated, while $x$ is called the \textbf{integration variable} and $c$ is the \textbf{constant of integration}.\\

As with many things in this module, the easiest way to get to grips with the content and the new terminology is through solving example problems.
\begin{ex}
The indefinite integral 
\begin{equation*}
\int \left(x^{4}+2x^{2}+2\right)\ud x
\end{equation*}
is calculated as in the same way as the antiderivative above. Thus we have 

\begin{equation*}
\int \left(x^{4}+2x^{2}+2\right)\ud x=\frac{1}{5}x^{5}+\frac{2}{3}x^{3}+2x+c.
\end{equation*}
\end{ex}

It is important to include the $\ud x$ after the $\int$ so that it is clear what is being integrated over. I also like to put brackets round the integrand when it involves multiple terms to make it unambiguous what is being integrated over. \\

\textbf{Warning!} If you are asked to calculate an indefinite integral you \textbf{must} include the constant of integration for your answer to be counted as correct.\\

Note that the integral has several nice properties which we will not prove here:

\begin{align}
\int c f(x) \ud x &= c\int f(x) \ud x, \label{eq: integral scalar multiplication}\\
\int \left(-f(x)\right)\ud x&=-\int f(x)\ud x, \\
\int \left(f(x)\pm g(x)\right)\ud x &=\int f(x)\ud x \pm \int g(x)\ud x. \label{eq: integral of sum}
\end{align}
These are analogous to \cref{eq: derivative of sum,eq: derivative of difference,eq: derivative scalar multiplication} for the derivative. We have implicitly been using them already.\\

Note that for all of the functions that we know the derivative of we now know the integral of by reversing the identities so we have that
\begin{align*}
\int\cos(x)\ud x &=\sin(x) +c,\\
\int \sin(x) \ud x &=-\cos(x)+c,\\
\int \frac{1}{x}\ud x &=\ln(x) +c,\\
\int e^{x}\ud x &=e^{x}+c.
\end{align*}

We can also invert the identity that 
\begin{equation*}
\frac{\ud x^{n}}{\ud x}=nx^{n-1}
\end{equation*}
to be
\begin{equation}
\int x^{n}\ud x =\frac{1}{n+1}x^{n+1} +c
\label{eq: integral of monomial}
\end{equation}
which holds for $n\neq -1$.
\begin{ex}
We know that the derivative of $\tan(x)$ is $\sec^{2}(x)$ which means that the antiderivative of $\sec^{2}(x)$ is
\begin{equation*}
F(x)=\tan(x),
\end{equation*}
and the indefinite integral is thus
\begin{equation*}
\int \sec^{2}(x)\ud x = \tan(x)+c.
\end{equation*}
\end{ex}

\begin{exercise}
Show that these trig functions have the claimed indefinite integrals:
\begin{align*}
\int \csc^{2}(x)\ud x &=-\cot(x) +c,\\
\int \sec(x)\tan(x)\ud x &=\sec(x) +c,\\
\int \csc(x)\cot(x)\ud x &=-\csc(x) +c.
\end{align*}
\end{exercise}

The process of calculating an integral by finding the antiderivative as the reverse of differentiation is sometimes called \textbf{integration by inspection}. This is very useful when you have a simple enough function that you already know what you differentiate to get it.\\

\begin{ex}
The integral
\begin{equation*}
\int \ud x,
\end{equation*}
is found by inverting the derivative identity $\left(x\right)'=1$ to give
\begin{equation*}
\int \ud x=\int 1\ud x =x+c.
\end{equation*}
\end{ex}


\begin{ex}
Using some of the standard integrals that we have given above we evaluate 
\begin{equation*}
\int \left(x^{2}+x^{-3}+x^{-1}\right)\ud x
\end{equation*}
as follows:
\begin{align*}
\int \left(x^{2}+x^{-3}+x^{-1}\right)\ud x&=\int x^{2}\ud x+\int x^{-3}\ud x+\int x^{-1}\ud x\\
							&=\frac{1}{3}x^{3}-\frac{1}{2}x^{-2}+\ln(x) +c\\
							&=\frac{x^{3}}{3}-\frac{1}{2x^{2}}+\ln(x)+c.
\end{align*}
Note that here we have used \cref{eq: integral of monomial} for both the positive and negative power terms.
\end{ex}


\section{Techniques for integration}

\subsection*{Integration by substitution}
Now that we know the basics of how to calculate indefinite integrals we can introduce some common techniques. The first of these is \textbf{integration by substitution}, which is a way to integrate more complicated expressions by making them look like expressions that we already know how to integrate. \\

The idea behind integration by substitution is to consider an integral like
\begin{equation*}
I= \int 3x^{2}\sqrt[3]{x^{3}+4}\ud x.
\end{equation*}

We then observe that if we let $u=x^{3}+4$ and differentiate $u$ with respect to $x$ we get
\begin{equation*}
\frac{\ud u}{\ud x}=3x^{2}.
\end{equation*}
Then we see that we can rewrite $I$ as
\begin{equation*}
I=\int \frac{\ud u}{\ud x}\sqrt[3]{u}\ud x.
\end{equation*}
Remember back in \cref{sec:differentiation} we said that in certain circumstances we could treat a derivative as a bit like a fraction, well we can do this under an integral sign. It turns out that if we define a function $u(x)$ then integrating over $u$ is related to integrating over $x$ in the following way
\begin{equation}
\int f(u)\ud u=\int f(u(x))\frac{\ud u}{\ud x}\ud x.
\label{eq: measure change of variable}
\end{equation}

This means that in our above integral we can rewrite $\frac{\ud u}{\ud x}\ud x=\ud u$ so that
\begin{equation*}
I=\int \frac{\ud u}{\ud x}\sqrt[3]{u}\ud x=\int \sqrt[3]{u}\ud u=\frac{3}{4}u^{\frac{4}{3}}+c=\frac{3}{4}\left(x^{3}+4\right)^{\frac{4}{3}}+c.
\end{equation*}

This is the same recipe that we will always use when integrating by substitution:

\begin{itemize}
\item Simplify the integrand as much as possible.
\item Guess at a simplifying substitution $u=u(x)$, or use the substitution that you are given.
\item Calculate the derivative $\ud u/\ud x$ and turn the integral over $x$ into an integral over $u$.
\item Evaluate the integral over $u$
\end{itemize}

Not all substitutions are as obvious as the above one, this is particularly true for trig substitutions where identifying the correct substitution to use takes practice.

\begin{ex}
Consider the integral 
\begin{equation*}
I=\int \frac{1}{x^{2}+1}\ud x.
\end{equation*}
We can evaluate this using the substitution $x=\tan(u)$, then since $\tan(u)=\sin(u)/\cos(u)$ we use the quotient rule to evaluate the derivative of $\sin(u)$
\begin{align*}
\frac{\ud}{\ud u}\tan(u)	&=\frac{\ud }{\ud u}\left(\frac{\sin(u)}{\cos(u)}\right)\\
					&=\frac{1}{\cos^{2}}\left(\cos^{2}(u)+\sin^{2}(u)\right)\\
					&=1+\tan^{2}(u).
\end{align*}
Recalling that $x=\tan(u)$ we have that
\begin{equation*}
\frac{\ud x}{\ud u}=1+x^{2},
\end{equation*}
which means that
\begin{align*}
I 	&=\int \frac{1}{x^{2}+1}\ud x\\
	&=\int \frac{1}{x^{2}+1}\frac{\ud x}{\ud u}\ud u\\
	&=\int \frac{1}{x^{2}+1}(x^{2}+1)\ud u\\
	&=\int \ud u\\
	&=u+c\\
	&=\arctan(x)+c.
\end{align*}
\end{ex}

Approaches like the above example are how we can calculate integral expressions for all of the inverse trig functions. We just need to know the correct substitution. 

\subsection*{Integration by parts}
\textbf{Integration by parts} is the integral equivalent of the product rule from differentiation. We can use it whenever we have an integral which is the product of two functions and we know how to differentiate one of them and how to integrate the other. \\

Recall that the product rule says that given two functions of $x$, $f(x),g(x)$ the derivative of their product is 
\begin{equation*}
\frac{\ud }{\ud x}\left(fg\right)=\frac{\ud f}{\ud x}g(x)+f(x)\frac{\ud g}{\ud x}.
\end{equation*}
If we integrate this expression we have that
\begin{align*}
\int \frac{\ud }{\ud x}\left(fg\right)\ud x 	&=f(x)g(x)+c,\\
								&=\int \left(\frac{\ud f}{\ud x}g(x)+f(x)\frac{\ud g}{\ud x}\right)\ud x\\
								&=\int \frac{\ud f}{\ud x}g(x)\ud x+\int f(x)\frac{\ud g}{\ud x}\ud x.
\end{align*}
rearranging this and setting $c$ to zero gives
\begin{equation}
\int f(x)\frac{\ud g}{\ud x}\ud x=f(x)g(x)-\int \frac{\ud f}{\ud x}g(x)\ud x.
\label{eq: integration by parts}
\end{equation}
The reason that we can set $c$ to zero is that we still have some other indefinite integrals that have not been evaluated and these contain a constant of integration. Thus we can absorb $c$ into the remaining integral.\\

The formula in \cref{eq: integration by parts} is known as the integration by parts formula. We will apply it for indefinite integrals first and then turn to definite integrals.\\

\begin{ex}
Consider the integral
\begin{equation*}
\int \ln(x)\ud x=\int \ln(x) 1 \ud x=\int \ln(x)\frac{\ud x}{\ud x}\ud x
\end{equation*}
this can be evaluated using integration by parts with $f(x)=\ln(x)$ and $g(x)=x$.

Applying \cref{eq: integration by parts}  gives
\begin{align*}
\int \ln(x)\ud x 	&=\int \ln(x)\frac{\ud x}{\ud x}\ud x\\
			&=\ln(x) x -\int \left(\frac{\ud}{\ud x}\ln(x)\right)x\ud x\\
			&=x\ln(x)-\int \frac{1}{x}x \ud x\\
			&=x\ln(x)-x+c.
\end{align*}
\end{ex}

The trick when applying integration by parts is to look for one function that is easy to integrate, which will be our $\ud g(x)/\ud x$, and another that is easy to differentiate, our $f(x)$.

\begin{ex}
Consider the integral
\begin{equation*}
\int x\sin(x)\ud x =\int x\frac{\ud}{\ud x}\left(-\cos(x)\right)\ud x.
\end{equation*}
We integrate this by parts taking $f(x)=x$ and $g(x)=-\cos(x)$, so that
\begin{align*}
\int x\sin(x)\ud x 	&=\int x\frac{\ud}{\ud x}\left(-\cos(x)\right)\ud x\\
				&=x(-\cos(x))-\int\left(\frac{\ud x}{\ud x}\right)(-\cos(x))\ud x\\
				&=-x\cos(x)+\int \cos(x)\ud x\\
				&=-x\cos(x)+\sin(x)+c.
\end{align*}
\end{ex}

\begin{exercise}
Use integration by parts to evaluate 
\begin{equation*}
I=\int x e^{2x}\ud x.
\end{equation*}
\end{exercise}

In this module we will only come across problems where you need to apply integration by parts once. However, some integrals require us to apply integration by parts more than once, and some require other tricks.

\begin{mdiv}
As an example of repeated integration by parts consider
\begin{equation*}
I=\int x^{2}\sin(x)\ud x=\int x^{2}\frac{\ud}{\ud x}(-\cos(x))\ud x.
\end{equation*}

We can carry out an integration by parts taking $f(x)=x^{2}$ and $g(x)=-\cos(x)$ to get
\begin{align*}
I 	&=\int x^{2}(-\cos(x))\ud x\\
	&=-x^{2}\cos(x)-\int 2x(-\cos(x))\ud x\\
	&=-x^{2}\cos(x)+2\int x\cos(x)\ud x\\
	&=-x^{2}\cos(x)+2\int x\frac{\ud}{\ud x}(\sin(x))\ud x.
\end{align*}
The integral in the last line is one that we have to do by parts taking $f(x)=x$ and $g(x)=\sin(x)$. This means that
\begin{align*}
I 	&=-x^{2}\cos(x)+2\int x\frac{\ud}{\ud x}(\sin(x))\ud x\\
	&=-x^{2}\cos(x)+2\left(x\sin(x)-\int \sin(x)\ud x\right)\\
	&=-x^{2}\cos(x)+2\left(x\sin(x)+\cos(x)+c\right)\\
	&=-x^{2}\cos(x)+2x\sin(x)+2\cos(x)+\tilde{c},
\end{align*}
where in the last line we have redefined the constant of integration to be $\tilde{c}=2c$.
\end{mdiv}

\subsection*{Partial fractions}
Another integration technique that is useful for integration fractions is known as \textbf{Partial fractions}. It is an example of a simplification method and is not examinable within this module. However, it may be useful for you to have seen it at least once.

The idea is to take an expression like $1/(x^{2}-1)$ and express it as the some of two fractions
\begin{equation*}
\frac{1}{x^{2}-1}=\frac{1}{(x-1)(x+1)}=\frac{A}{x+1}+\frac{B}{x-1},
\end{equation*}
for $A$ and $B$ constants that need to be determined. This is done by multiplying through by $(x+1)(x-1)$ to get
\begin{equation*}
1=A(x-1)+B(x+1),
\end{equation*}
which rearranges to 
\begin{equation*}
(A+B)x-A+B-1=0.
\end{equation*}
For this to be true we need to match up coefficients of powers of $x$ on both sides of the equation so that
\begin{align*}
A+B&=0,\\
B-A-1&=0,
\end{align*}
which is solved by 
\begin{align*}
A&=-\frac{1}{2},\\
B&=\frac{1}{2}.
\end{align*}
Which gives
\begin{equation*}
\frac{1}{x^{2}-1}=-\frac{1}{2}\left[\frac{1}{x+1}-\frac{1}{x-1}\right].
\end{equation*}
These are both expressions that we can integrate to get
\begin{align*}
\int\frac{1}{x^{2}-1}\ud x 	&=-\frac{1}{2}\left[\int \frac{1}{x+1}\ud x-\int \frac{1}{x-1}\ud x\right]\\
					&=-\frac{1}{2}\ln\left(x+1\right)+\frac{1}{2}\ln\left(x-1\right)+c\\
					&=\ln\left[\left(\frac{x-1}{x+1}\right)^{\frac{1}{2}}\right]+c. 
\end{align*}
Note that in terms of what we know at this stage, the right hand side is only valid for $x>1$. 

The general procedure is as follows:
\begin{itemize}
\item Factorise the denominator in terms of its roots.
\item Write down a sum of the reciprocals of the roots with arbitrary coefficients. This steps becomes complicated if there are repeated roots.
\item Multiply through by the denominator on the left hand side and express it as a polynomial in $x$.
\item Solve the system of equations for the coefficients $A, B, \dots{}$.
\item Substitute the coefficients into the partial fractions expression and carry out the integral making use of integration by parts or substitutions as necessary.
\end{itemize}

As mentioned in the second step, if there are repeated roots this becomes trickier, since this is already a non examinable diversion we will not give the details here. 

\section{Definite integrals}

In the previous section we focussed on antiderivatives and indefinite integrals, now we return to the original motivation for the integral, as the measure of the area under a curve. This is called a definite integral and returns a number rather than a function as it involves us carrying out a definite integral and then evaluating the result at the end points. This may sound confusing at first but we will learn how to do it by seeing some examples.\\

To start consider the function $f(x)=x^{2}+1$ on the interval $[0,2]$, the full details of this example are given in \cite{calcI}. The idea is that we want to find teh area enclosed between the curve and the $x$-axis, shown in \cref{fig: first integral}. Once we can integrate we will eb able to compute this exactly, but for now we should consider how to approximate this area.\\

\begin{figure}[ht]
    \centering
\ThisAltText{Graph of the function x squared plus one.}
    \begin{tikzpicture}[line width=1pt,line cap=round,line join=round, smooth,variable=\x]
     \draw[->] (-0.2,0) -- (5,0) node[below] {$x$};
  \draw[->] (0,-0.2) -- (0,5.5)node[above]{$y$};
 \draw[color=CDnavy, domain=0:4]   plot[samples=300] (\x ,{1+ ((0.5)^(2))*\x^(2)}) ;
 \draw[-, color =CDred ] (0,0) -- (0,1) ;
  \draw[-, color =CDred ] (4,0) -- (4,5);
\filldraw[black] (0,1) circle (1pt)node[anchor=east]{$1$};
\filldraw[black] (0,-0.2) circle (1pt) node[anchor=north]{$0$};
\filldraw[black] (4,0) circle (1pt) node[anchor=north]{$2$};
    \end{tikzpicture}
    \caption{A plot of our favourite function $f(x)=x^{2}+1$.}
        \label{fig: first integral}
\end{figure}

A shape that we understand how to calculate the area of really well is a rectangle. If a rectangle has height $\Delta y$ and base length $\Delta x$ then its area is $\Delta x \times \Delta y$. The aim is then to find the area under the curve by breaking it up into rectangles with a regular width. We do this by splitting up the interval, $[0,2]$ in this case, into $n$ subintervals. This will give us an approximation of the area which will improve the more sub integrals we take, in other words the larger that $n$ is. The width of a subinterval is then 
\begin{equation*}
\Delta x=\frac{2-0}{n}=\frac{2}{n}.
\end{equation*}
In general it will be $\Delta x=(b-a)/n$, where the interval is $[a,b]$. \\

We will choose the height of the intervals so that the agree with the value of $f$ at the right hand side of the subinterval. For example if $n=4$ we have four subintervals of length $1/2$, 
\begin{equation*}
\left[0,\frac{1}{2}\right], \quad \left(\frac{1}{2},1\right], \quad \left(1,\frac{3}{2}\right], \quad \left(\frac{3}{2},2\right].
\end{equation*}
In the each sub interval we take the height of our rectangle to be 
\begin{align*}
f\left(\frac{1}{2}\right)&=\frac{5}{4},\\
f(1)&=2,\\
f\left(\frac{3}{2}\right)&=\frac{13}{4},\\
f(2)&=5.
\end{align*}
Plotting the rectangles we get the picture in \cref{fig: approximate area}, note that the area of the rectangles is larger than the area under the curve. As an exercise consider the case where we take the rectangle to have the height of the function on the left boundary of the integral, what changes?\\


\begin{figure}[ht]
    \centering
\ThisAltText{Graph of the function x squared plus one with rectangles approximating the area under the curve.}
    \begin{tikzpicture}[line width=1pt,line cap=round,line join=round, smooth,variable=\x]
     \draw[->] (-0.2,0) -- (5,0) node[below] {$x$};
  \draw[->] (0,-0.2) -- (0,5.5)node[above]{$y$};
 \draw[color=CDnavy, domain=0:4]   plot[samples=300] (\x ,{1+ ((0.5)^(2))*\x^(2)}) ;
  \draw[-, color =CDgreen ] (4,0) -- (4,5);
\filldraw[black] (0,1) circle (1pt)node[anchor=east]{$1$};
\filldraw[black] (0,-0.2) circle (1pt) node[anchor=north]{$0$};
\filldraw[black] (4,0) circle (1pt) node[anchor=north]{$2$};
\draw[-, color =CDgreen ] (1,0) -- (1,2) ;
\draw[-, color =CDgreen ] (2,0) -- (2,3.25);
\draw[-, color =CDgreen ] (3,0) -- (3,5) ;
 \draw[-, color =CDgreen ] (0,1.25) -- (1,1.25);
\draw[-, color =CDgreen ] (1,2) -- (2,2);
\draw[-, color =CDgreen ] (2,3.25) -- (3,3.25) ;
\draw[-, color =CDgreen ] (3,5) -- (4,5);
    \end{tikzpicture}
    \caption{A plot of  $f(x)=x^{2}+1$ with rectangles approximating the area under the curve.}
        \label{fig: approximate area}
\end{figure}

Now lets calculate the area of each of these integrals by multiplying the width, $1/2$ for all four, by the height:
\begin{align*}
A_{1}&=\frac{1}{2}\times\frac{5}{4}=\frac{5}{8} ,\\
A_{2}&=\frac{1}{2}\times 2=1,\\
A_{3}&=\frac{1}{2}\times\frac{13}{4}=\frac{13}{8},\\
A_{4}&=\frac{1}{2}\times 5=\frac{5}{2}.
\end{align*}
Adding all of these together gives
\begin{align*}
A_{r} 	&=A_{1}+A_{2}+A_{3}+A_{4}\\
	&=\frac{5}{8}+1+\frac{13}{8}+\frac{5}{2}\\
	&=\frac{23}{4}=5.75.
\end{align*}

If we had taken the heights at the left end points we would have found $A_{l}=3.75$, which is an underestimate of the area, again I invite you to check the details for yourself, or look at \cite{calcI} which goes through it.\\

Finally, we could make a third choice where we take the height of the rectangle to be the value of the function at the midpoint, shown in \cref{fig: approximate area mid}, then we find that the area is $A_{m}=37/8=4.625$.\\

\begin{figure}[ht]
    \centering
\ThisAltText{Graph of the function x squared plus one with rectangles approximating the area under the curve.}
    \begin{tikzpicture}[line width=1pt,line cap=round,line join=round, smooth,variable=\x]
     \draw[->] (-0.2,0) -- (5,0) node[below] {$x$};
  \draw[->] (0,-0.2) -- (0,5.5)node[above]{$y$};
 \draw[color=CDnavy, domain=0:4]   plot[samples=300] (\x ,{1+ ((0.5)^(2))*\x^(2)}) ;
  \draw[-, color =CDgreen ] (4,0) -- (4,4.0625);
\filldraw[black] (0,1) circle (1pt)node[anchor=east]{$1$};
\filldraw[black] (0,-0.2) circle (1pt) node[anchor=north]{$0$};
\filldraw[black] (4,0) circle (1pt) node[anchor=north]{$2$};
\draw[-, color =CDgreen ] (1,0) -- (1,1.5625) ;
\draw[-, dashed] (0.5,0) -- (0.5,1.0625);
\draw[-, dashed](1.5,0)--(1.5,1.5625);
\draw[-, dashed] (2.5,0) -- (2.5,2.5625);
\draw[-, dashed](3.5,0)--(3.5,4.0625);
\draw[-, color =CDgreen ] (2,0) -- (2,2.5625);
\draw[-, color =CDgreen ] (3,0) -- (3,4.0625) ;
\draw[-, color =CDgreen ] (4,0) -- (4,4.0625) ;
 \draw[-, color =CDgreen ] (0,1.0625) -- (1,1.0625);
\draw[-, color =CDgreen ] (1,1.5625) -- (2,1.5625);
\draw[-, color =CDgreen ] (2,2.5625) -- (3,2.5625) ;
\draw[-, color =CDgreen ] (3,4.0625) -- (4,4.0625);
    \end{tikzpicture}
    \caption{A plot of  $f(x)=x^{2}+1$ with rectangles approximating the area under the curve.}
        \label{fig: approximate area mid}
\end{figure}

You will see that the value of the value of the area that we calculate depends on the size of our rectangles. This should not be surprising, but suggests that we need a more systematic way to do this, which enables us to find a unique answer. This is what integration will give us. Once we have the machinery of integration we can calculate the actual area under the curve to be $A=14/3=4.67$. So the mid point method above is the closest to the true value.\\

This leads us to the concept of the \textbf{Riemann sum}\footnote{Bernhard Riemann was a German Mathematician who contributed to many different areas of mathematics and is maybe most famous for the Riemann hypothesis about the distribution of prime numbers.}.  Consider a function plotted over an interval. Split the integral into $n$ subintervals each of width $\Delta x$ as above. Then pick a point $x_{i}^{\star}$ in each sub interval and use the rectangle of height $f(x_{i}^{*})$ as our approximation for the area of the function over that sub interval. The total area is then
\begin{equation}
A\simeq f(x_{1}^{*})\Delta x+f(x_{2}^{*})\Delta x +\cdots + f(x_{i}^{*})\Delta x+\cdots +f(x_{n}^{*})\Delta x=\sum_{i=1}^{n} f(x_{i}^{*})\Delta x.
\end{equation}

This approximation gets better and better the more subintervals we take, as they will become narrower and narrower, and their height becomes a better match for the functions value over that interval. If we could take the limit of $n\to \infty$ the sub intervals will become a single point and the height is the value of the function at that point. The sum would then become the exact area,
\begin{equation}
A=\lim_{n\to\infty}\sum_{i=1}^{n} f(x_{i}^{*})\Delta x.
\label{eq: Riemann Sum}
\end{equation}

It is important to note that if our function was below the $x$-axis, then we would get a negative answer for the area. This is because we take the heights of the rectangles to be the value of the value of the function at some point in the interval. So if the function takes negative values, we will find a negative area.\\

This is a feature and not a bug! The integral will be a signed count of the area under a curve. This does mean that if a function crosses the axis in the interval that we are interested in, we will need to be careful whether we are interested in the net area under the curve, the difference between the positive and negative areas, or the total area, which does not care about the sign.


\begin{exercise}
Consider the function $x^{2}-4$ on the interval $[0,2]$. Take $n=8$ and use the midpoint approach to estimate the area between the function and the $x$-axis. You should find $A_{m}=-171/32$.
\end{exercise}

The sum in \cref{eq: Riemann Sum} is taken as the definition of the definite integral:
\begin{equation}
\int_{a}^{b}f(x)\ud x = \lim_{n\to\infty}\sum_{i=1}^{n} f(x_{i}^{*})\Delta x.
\label{eq: definite integral}
\end{equation}
This is the integral of the function $f(x)$ over the interval $[a,b]$. The terminology that is usually used is that $a$ is called the lower limit and $b$ the upper limit.

\begin{mdiv}
If this was a module for mathematicians, we would spend a bit of time here discussing how to calculate this using the sum definition given on the right hand side of \cref{eq: definite integral}.  Doing this involves knowing some standard expressions for summations such as:
\begin{align*}
\sum_{i=1}^{n}1&=n,\\
\sum_{i=1}^{n}i&=\frac{n(n+1)}{2},\\
\sum_{i=1}^{n}i^{2}&=\frac{n(n+1)(2n+2)}{6}.
\end{align*}
Armed with these and a few other expressions one can evaluate the Riemann sum expression for a wide variety of functions. Including $f(x)=x^{2}+1$ that we discussed above. If you are interested in trying this it is worth doing at least once, though is an exercise for maths enthusiasts only as you are unlikely to need to use the Riemann sum definition of an integral directly.
\end{mdiv}

The properties that we discussed for indefinite integrals still hold here, but now we get some new properties:
\begin{align*}
\int_{a}^{b}f(x)\ud x&=-\int_{b}^{a}f(x)\ud x, \quad \text{swapping limits introduces a minus sign.}\\
\int_{a}^{a}f(x)\ud x&=0, \quad \text{if the limits agree the integral is zero.}\\
\int_{a}^{b}f(x)\ud x &=\int_{a}^{c}f(x)\ud x +\int_{c}^{b}f(x)\ud x, \quad \text{ where $c$ is any number between $a$ and $b$.}
\end{align*}

Recall that the antiderivative $F(x)$ for the function $f(x)$ is defined as any function such that
\begin{equation*}
\frac{\ud F}{\ud x}=f(x).
\end{equation*}
For a definite integral  antiderivatives make an appearance through the fundamental theorem of calculus. This states that for a function $f(x)$ which is continuous on $[a,b]$ and has antiderivative $F(x)$ then:
\begin{equation}
\int_{a}^{b}f(x)\ud x=\left[F(x)\right]_{a}^{b}=F(b)-F(a).
\label{eq: ftc I}
\end{equation}

In other words, a definite integral is just the difference between the antiderivative evaluated at the endpoints of the interval. This result is proved in \cite{calcI}, and a proof may be added to \cref{sec: proofs} at some stage.

\begin{ex}
Consider the function $f(x)=x^{2}+1$, we know that an antiderivative for this is 
\begin{equation*}
F(x)=\frac{1}{3}x^{3}+x,
\end{equation*}
where we have carried out the indefinite integral of $f(x)$. Using \cref{eq: ftc I} we get that the definite integral over the interval $[0,2]$ is
\begin{align*}
\int_{0}^{2}\left(x^{2}+1\right)\ud x 	&=\left[\frac{1}{3}x^{3}+x\right]_{0}^{2}\\
						&=\frac{1}{3}(2)^{3}+2-\left(\frac{1}{3}(0)^{3}+0\right)\\
						&=\frac{8}{3}+2-0\\
						&=\frac{14}{3}
\end{align*}
which is what we claimed was the exact value above.\\

Note that we can use any antiderivative since the constant piece will cancel out in the difference for example if we took
\begin{equation*}
F_{2}(x)=\frac{1}{3}x^{3}+c,
\end{equation*}
then 
\begin{equation*}
\left[\frac{1}{3}x^{3}+x+c\right]_{0}^{2}=\frac{14}{3}+c-\left(0+c\right)=\frac{14}{3}+c-c=\frac{14}{3}.
\end{equation*}
This is why we always take the simplest antiderivative when evaluating definite integrals.
\end{ex}


\begin{ex}
Consider the function $f(x)=x^{4}+2x^{2}+2$ and calculate its integral over the interval $[0,1]$. Recall that an antiderivative of this function is
\begin{equation*}
F(x)=\frac{1}{5}x^{5}+\frac{2}{3}x^{3}+2x,
\end{equation*}
so using \cref{eq: ftc I} the definite integral is
\begin{align*}
\int_{0}^{1}\left(x^{4}+2x^{2}+2\right)\ud x 	&=\left[\frac{1}{5}x^{5}+\frac{2}{3}x^{3}+2x\right]_{0}^{1}\\
								&=\frac{1}{5}+\frac{2}{3}+2 -0\\
								&=\frac{43}{15}.
\end{align*}
\end{ex}

As with derivatives you will not evaluate every integral by finding an antiderivative. You will become familiar with a range of standard integrals, and techniques for evaluating integrals so that you do not have to calculate an antiderivative every time. The techniques of substitution and integration by parts that we met above are very useful here. Though we now need to be careful about what happens to the limits when we perform a substitution and when dealing with the total derivative part of integration by parts.

\begin{ex}
Consider the integral 
\begin{equation*}
I=\int_{0}^{1}\frac{1}{x^{2}+1}\ud x.
\end{equation*}
We evaluated the indefinite integral above using the substitution $x=\tan(u)$ which meant that
\begin{equation*}
\int\frac{1}{x^{2}+1}\ud x=\int\ud u.
\end{equation*}
The new step is that we need to look at what happens to the limits of the integral upon substitution. e.g. find the values of $u$ such that $\tan(u)=0$ and $\tan(u)=1$, remembering that we need to work in radians. In this case $\tan(0)=0$ and $\tan(\uppi/4)=1$ so we have that 
\begin{equation*}
I=\int_{0}^{1}\frac{1}{x^{2}+1}\ud x=\int_{0}^{\frac{\uppi}{4}}\ud u=\left[u\right]^{\frac{\uppi}{4}}_{0}=\frac{\uppi}{4}.
\end{equation*}
\end{ex}

Any time you use integration by substitution for a definite integral you similarly need to keep track of how the limits transform under the substitution.

With integration by parts the identity in \cref{eq: integration by parts} is modified to become
\begin{equation}
\int^{b}_{a} f(x)\frac{\ud g}{\ud x}\ud x =\left[f(x)g(x)\right]_{a}^{b}-\int_{a}^{b}\frac{\ud f}{\ud x}g(x)\ud x.
\end{equation}
In other words we need to evaluate the total derivative at the limits of integration.

\begin{ex}
Consider the integral
\begin{equation*}
I=\int_{1}^{2}\ln(x)\ud x.
\end{equation*}
We saw above that integration by parts gave that 
\begin{equation*}
\int\ln(x)\ud x=x\ln(x)-x+c.
\end{equation*}
When we calculated this we took $f(x)=\ln(x)$ and $g(x)=x$ so that
\begin{align*}
I 	&=\int_{1}^{2}\ln(x)\ud x\\
	&=\int_{1}^{2}\ln(x)\frac{\ud x}{\ud x}\\
	&=\left[\ln(x)x\right]_{1}^{2}-\int{1}^{2}\ud x\\
	&=2\ln(2)-\ln(1)-\left[x\right]_{1}^{2}\\
	&=2\ln(2)-(2-1)\\
	&=2\ln(2)-1.
\end{align*}

\end{ex} 


\subsection*{Improper integrals}
An important type of definite integrals are the \textbf{improper integrals}. These are definite integrals
\begin{equation*}
\int_{a}^{b}f(x)\ud x,
\end{equation*}
where either $f(x)$ becomes infinite for $a\leq x\leq b$, or one of the limits goes to infinity. \\

The following integrals are improper:

\begin{itemize}
\item \begin{equation*}
\int_{1}^{\infty}\frac{1}{x^{2}}\ud x,
\end{equation*}
because the upper limit is infinity,
\item \begin{equation*}
\int_{0}^{1}\frac{1}{\sqrt{x}}\ud x
\end{equation*}
because $1/\sqrt{x}$ becomes infinite for $x=0$.
\item \begin{equation*}
\int_{1}^{\infty}\sin(x)\ud x,
\end{equation*}
because the upper limit is infinity,
\end{itemize}

It turns out that some improper integrals can still give sensible results, we just need to interpret them properly.\\

First up, we reinterpret them using a limit so that
\begin{align*}
\int_{1}^{\infty}\frac{1}{x^{2}}\ud x 	&=\lim_{X\to \infty}\int_{1}^{X}\frac{1}{x^{2}}\ud x\\
						&=\lim_{X\to \infty}\left[-\frac{1}{x}\right]_{0}^{X}\\
						&=\lim_{X\to \infty}\left(1-\frac{1}{X}\right)\\
						&=1.
\end{align*}
Thus the improper integral gives us a sensible answer.

Next consider
\begin{align*}
\int_{0}^{1}\frac{1}{\sqrt{x}}\ud x 	&=\lim_{X\to 0}\int_{X}^{1}\frac{1}{\sqrt{x}}\ud x\\
						&=\lim_{X\to 0}\left[2\sqrt{x}\right]_{X}^{1}\\
						&=\lim_{X\to 0}2\left(1-\sqrt{X}\right)\\
						&=2.
\end{align*}

When we can reinterpret improper integrals using limits and get a finite answer we say that the integral is \textbf{convergent}. However, this does not always work as sometimes the improper integral is \textbf{divergent}. 

\begin{ex}
Consider the integral 
\begin{equation*}
I=\int_{0}^{\infty}\sin(x)\ud x.
\end{equation*}
If we try to interpret it as above we find that
\begin{align*}
I 	&=\int_{0}^{\infty}\sin(x)\ud x\\
	&=\lim_{X\to\infty}\int_{0}^{X}\sin(x)\ud x\\
	&=\lim_{X\to\infty}\left[-\cos(x)\right]_{0}^{X}\\
	&\lim_{X\to\infty}\left(1-\cos(x)\right)
\end{align*}
which does not converge as $\cos(x)$ oscillates between $-1$ and $1$.
\end{ex}

\begin{ex}
Consider the integral 
\begin{equation*}
I=\int_{0}^{\infty}x\ud x.
\end{equation*}
We can attempt to evaluate this as follows:
\begin{align*}
I 	&=\int_{0}^{\infty}x\ud x\\
	&=\lim_{X\to \infty}\int_{0}^{X}x\ud x\\
	&=\lim_{X\to \infty}\left[\frac{1}{2}x^{2}\right]_{0}^{X}\\
	&=\lim_{X\to\infty}\frac{1}{2}X^{2},
\end{align*}
which tends to infinity, so this is another divergent integral.
\end{ex}

\begin{ex}
Consider the integral 
\begin{equation*}
I=\int_{0}^{1}\frac{1}{x}\ud x.
\end{equation*}
We attempt to evaluate this as follows:
\begin{align*}
I 	&=\int_{0}^{1}\frac{1}{x}\ud x\\
	&=\lim_{X\to 0^{+}}\int_{X}^{1}\frac{1}{x}\ud x\\
	&=\lim_{X\to 0^{+}}\left[\ln(x)\right]_{X}^{1}\\
	&=\lim_{X\to 0^{+}}\left(-\ln(X)\right),
\end{align*}
which again tends to infinity.
\end{ex}

\subsection*{Splitting the range}
Another useful technique for evaluating definite integrals is to split the range of integration. For example if we are interested in the integral of 
\begin{equation*}
\vert x\vert=\begin{cases}
&0 \quad x < 0\\
&1 \quad x\geq 0,
\end{cases}
\end{equation*}
between $-1$ and $1$. Because the function changes when the interval goes from negative to positive we need to split the integral using the identity that we learnt above:
\begin{align*}
\int_{-1}^{1}\vert x\vert\ud x 	&=\int_{-1}^{0}\vert x\vert\ud x+\int_{0}^{1}\vert x\vert\ud x\\
					&=-\int_{-1}^{0}x\ud x+\int_{0}^{1}x\ud x\\
					&=-\frac{1}{2}\left[x^{2}\right]_{-1}^{0}+\frac{1}{2}\left[x^{2}\right]_{0}^{1}\\
					&=\frac{1}{2}+\frac{1}{2}=1.
\end{align*}

\begin{exercise}
Evaluate the integral
\begin{equation*}
I=\int_{-\frac{\uppi}{2}}^{\frac{\uppi}{2}}\sin\left(\vert x\vert\right)\ud x
\end{equation*}
by splitting the range of the integral.
\end{exercise}

\subsection*{Symmetries of functions}
Before moving on to some applications of integration we have time to learn one more useful trick. We call a function $f(x)$ even if 
\begin{equation}
f(-x)=f(x)
\label{eq: even fn}
\end{equation}
and we call it odd if
\begin{equation}
f(-x)=-f(x).
\label{eq: odd fn}
\end{equation}
In other words an even function does not care about the sign of its argument, while and odd function changes sign if its argument does.

An example of an even function is $\cos(x)$ while $\sin(x)$ is an odd function.

\begin{exercise}
Can you come up with more examples of even and odd functions? 
\end{exercise}

When we integrate an even function over a symmetric range, $[-a,a]$ we have that 
\begin{equation}
\int_{-a}^{a}f(x)\ud x=2\int_{0}^{a}f(x)\ud x.
\end{equation}

while if we integrate an odd function over the same region it will vanish,
\begin{equation}
\int_{-a}^{a}f(x)\ud x=0.
\end{equation}

\begin{exercise}
Prove that these formulae are true. Hint: You will want to use the range splitting that we learnt about  previously.
\end{exercise}

\section{Applications of integrals}
As with the derivative, there are plenty of applications of integrals. We will touch on some of them here, but as usual there is a wealth of further details in both \citep{calcI} and \citep{riley_mathematical_2006}.\\

\subsection*{Averaging functions}

The average value of a continuous function $f(x)$ can be calculated by integration. For  a discrete set of $n$ numbers $\{x_{1},x_{2},\dots, x_{n}\}$ the average  is given by
\begin{equation*}
\bar{x}=\sum_{i=1}^{n}\frac{x_{i}}{n}.
\end{equation*}
If we went back to the Riemann sum definition of the integral in \cref{eq: Riemann Sum}, then it can be shown that for a continuous function on the interval $[a,b]$ that
\begin{equation}
f_{\text{avg}}=\frac{1}{b-a}\int_{a}^{b}f(x)\ud x.
\label{eq: average function}
\end{equation}

\begin{mdiv}
In this module you do not need to know where the identity in \cref{eq: average function} comes from, but if you are interested a proof is included in the Proof of Various Integral Properties section of \cite{calcI}. Going through the proof of the result shows you that this is the natural extension of the concept of an average to a continuous function. In a module for mathematicians, we would also be learning some interesting results on upper bounds of the value of an integral, called the \textbf{ML Lemma} in many places, where we can show that under some assumptions that 
\begin{equation*}
\left|\int_{a}^{b}f(x)\ud x\right| \leq M L,
\end{equation*}
where $L=b-a$ is the length of the interval, and $M$ is an upper bound on the value of $ \vert f(x)\vert$ over the interval.
\end{mdiv}

\begin{ex}
We can find the average of $f(x)=\sin(x)$ on the interval $[0,\uppi]$ as follows:
\begin{align*}
f_{\text{avg}} 	&=\frac{1}{b-a}\int_{a}^{b}f(x)\ud x\\
			&=\frac{\uppi-0}\int_{0}^{\uppi}\sin(x)\ud x\\
			&=\frac{1}{\uppi}\left[-\cos(x)\right]_{0}^{\uppi}\\
			&=\frac{2}{\uppi}
\end{align*}
\end{ex}

\begin{exercise}
Find the average value of $\cos(x)$ over the interval $[0,2\uppi]$.
\end{exercise}

In some applications it is the average value of the function so you can use this approach to calculate it.

\subsection*{Areas between curves}
\textcolor{red}{This subsection is currently lacking pictures, this will be added eventually.}\\


When we introduced the integral we gave the intuition to think of it as the area between the curve given by $y=f(x)$ and the $x$-axis.  Now the $x$-axis is just the curve given by $y=0$, so a natural question to ask is if we can use an integral to calculate the area between two curves. The answer is yes. \\

Consider two functions $f(x),g(x)$ on the interval $[a,b]$. The area is given by the integral of the difference between these two function $f(x)-g(x)$ or $g(x)-f(x)$, but we need to decide which way round to calculate the difference. The convention is to look at the graphs of the functions and subtract the lower function from the upper function.\\

This means that if $f(x)\geq g(x)$ for all $a\leq x\leq b$, the area between the curves is
\begin{equation}
A=\int_{a}^{b}\left(f(x)-g(x)\right)\ud x.
\label{eq: integral for area}
\end{equation}

If we took the functions the other way around we would get the same numerical answer but with a negative sign. It is important that you always check which function is greater over the interval. This can be checked by producing a plot of the functions. If the functions cross in the interval, then you should split the integral in to two regions and evaluate these separately to find the total area enclosed between the two curves. There are some examples of problems like this in the Area Between Curves section of \cite{calcI}.

\begin{ex}
Consider the two functions $f(x)=x^{2}$ and $g(x)=x^{3}$ shown in \cref{fig: area between functions}. Over the interval $[0,1]$ $f(x)\geq g(x)$ so the area is given by
\begin{equation*}
A=\int_{0}^{1}\left(x^{2}-x^{3}\right)\ud x.
\end{equation*}
Evaluating this integral we find that
\begin{align*}
A 	&=\int_{0}^{1}\left(x^{2}-x^{3}\right)\ud x\\
	&=\left[\frac{1}{3}x^{3}-\frac{1}{4}x^{4}\right]_{0}^{1}\\
	&=\frac{1}{3}-\frac{1}{4}-0\\
	&=\frac{1}{12}.
\end{align*}
\end{ex}
\begin{figure}[ht]
    \centering
\ThisAltText{A plot of the functions x squared and x cubed over the interval from zero to one.}
    \begin{tikzpicture}[line width=1pt,line cap=round,line join=round, smooth,variable=\x, scale =2]
     \draw[->] (-0.2,0) -- (5,0) node[below] {$x$};
  \draw[->] (0,-0.2) -- (0,2)node[above]{$y$};
 \draw[color=CDnavy, domain=0:4]   plot[samples=300] (\x ,{ ((0.25)^(2))*\x^(2)}) ;
  \draw[color=CDred, domain=0:4]   plot[samples=300] (\x ,{ ((0.25)^(3))*\x^(3)});
  \filldraw[black] (4,0) circle (1pt)node[anchor=north]{$1$};
\filldraw[black] (0,-0.2) circle (1pt) node[anchor=north]{$0$};
 \draw[-,dashed] (4,0) -- (4,1);
    \end{tikzpicture}
    \caption{A plot of the functions $f(x)=x^{2}$ and $g(x)=x^{3}$ over the interval $[0,1]$.}
        \label{fig: area between functions}
\end{figure}


\begin{exercise}
Calculate the area between the curves $\cos(x)$ and $\sin(x)$ over the interval $[0,\uppi/4]$.
\end{exercise}
\newpage

%%%%%%%%%%%%%%%%%%%%%%%%%%%%%%%%%%%%%%%%%%%%%%

\chapter{Differential Equations}
\label{sec:diff eqs}
This is a non-examinable section and is not taught every year. It is included in the indicative content of the module but is not assessed by the learning outcomes. No content has been included here yet, but when it is it can all be considered one big mathematical deviation.
\newpage

%%%%%%%%%%%%%%%%%%%%%%%%%%%%%%%%%%%%%%%%%%%%%%


\chapter{Numerical Methods}
\label{sec:numerics}

\section{Solving equations numerically}
It frequently happens that when confronted with a problem you end up with a mathematical problem or expression that cannot be evaluated analytically. This may be a differential equation that we cannot solve, an integral that has no closed form solution, or even an algebraic equation which does not yield an explicit solution. This is where we have to turn to computer packages or follow a numerical method. Nowadays many computer programs can implement these approaches as standard. However, it is a good idea to understand how to implement a few of these approaches by hand as you may end up writing one of your own at some stage.\\

Chapter 27 of \citep{riley_mathematical_2006}  contains a lot of information about this topic and is a good place to go for more details than we will discuss here.\\

Before moving on to numerical differentiation and integration we will first discuss how to solve algebraic equations numerically. This is all about finding the real roots of an equation $f(x)=0$. This equation can either be algebraic, with $f(x)$ being a polynomial, or transcendental, if $f(x)$ includes trig, log, or exponential terms. We will be using iterative schemes to solve the equations, where we make successive approximations to the true solution, which converge to the true solution. For a method to be successful we need both that the approximations converge to the true solution, and also that we only have a finite number of solutions.\\

It is important to note that different iteration methods will be better at solving certain kinds of problems, in practice some combination of methods is likely to be used. We will see four methods, \textbf{Rearrangement}, \textbf{Linear interpolation}, and \textbf{Bisection}, before discussing \textbf{Newton}'s method, which is the most important method to know in this module. 

\subsection*{Rearrangement}
To explain what we mean by an iterative scheme it is useful to consider the first example method, recasting our equation $f(x)=0$ as
\begin{equation*}
x=\phi(x),
\end{equation*}
for some slowly varying function $\phi(x)$. Our iteration scheme starts from a guess, $x_{0}$ which will not solve $f(x_{0})=0$ but should be picked so that it is close to zero\footnote{It is useful to plot or roughly sketch the function $f(x)$ as then it can be easier to pick a starting point.}. We then have that $x_{1}=\phi(x_{0})$ and can keep repeating this process, i.e. $x_{n+1}=\phi(x_{n})$. The solution we are looking for will be a value of $x$ that is left unchanged by applying $\phi$. \\

The different methods are different ways of building $\phi(x)$ so that we can start the iteration scheme.\\

As an example of how to use the rearrangement iteration consider
\begin{equation}
f(x)=x^{5}-3x^3+2x-4=0.
\label{eq: equation for iterating}
\end{equation}

\begin{figure}[ht]
    \centering
\ThisAltText{Graph of the function that we are studying via iteration.}
 %   \pdftooltip{
    \begin{tikzpicture}[line width=1pt,line cap=round,line join=round, smooth,variable=\x]
     \draw[->] (-0.3,0) -- (6.3,0) node[above] {$x$};
  \draw[->] (0,-4.5) -- (0,3)node[above]{$y$};
 \draw[color=CDnavy, domain=0:5.58]   plot[samples=300] (\x,{(0.33^(5))*\x^(5)-3*(0.33^(3))*\x^(3)+2*(0.33)*\x-4}) node[right] {$f(x)=x^{5}-3x^3+2x-4$};
\filldraw[black] (1.5,0) circle (1pt) node[anchor=north]{$0.5$};
\filldraw[black] (3,0) circle (1pt) node[anchor=north]{$1$};
\filldraw[black] (4.5,0) circle (1pt) node[anchor=north]{$1.5$};
\filldraw[black] (6,0) circle (1pt) node[anchor=north]{$2$};
\filldraw[black] (0,-4) circle (1pt) node[anchor=east]{$-4$};
\filldraw[black] (0,-3) circle (1pt) node[anchor=east]{$-3$};
\filldraw[black] (0,-2) circle (1pt) node[anchor=east]{$-2$};
\filldraw[black] (0,-1) circle (1pt) node[anchor=east]{$-1$};
\filldraw[black] (0,1) circle (1pt) node[anchor=east]{$1$};
\filldraw[black] (0,2) circle (1pt) node[anchor=east]{$2$};
    \end{tikzpicture}
%    }{plots of an exponential and logarithm }
    \caption{A plot of the function $f(x)=x^{5}-3x^3+2x-4$ for $0\leq x\leq 1.86$.}
        \label{fig: interpolation polynomial}
\end{figure}

By rearranging this equation we have that
\begin{equation*}
x=\left(3x^{3}-2x+4\right)^{\frac{1}{5}}.
\end{equation*}
From the plot we in \cref{fig: interpolation polynomial} we see that the root is between $x=1.5$ and $x=2$ and to see how the method works we will not attempt to start from that close to the true root. Here we take $x_{0}=1.8$ and then iterate using 
\begin{equation*}
x_{n+1}=\left(3x_{n}^{3}-2x_{n}+4\right)^{\frac{1}{5}}.
\end{equation*}

The successive values of $x_{n}$ and $f(x_{n})$ are shown in \cref{table:1} and they are closing in on
\begin{equation}
x=1.757632748,
\label{eq: precise root}
\end{equation} 
which is the value of the root to $9$ decimal places. We see from the table that once we got to $x_{5}$ we were only differing from the precise answer in the third decimal places, so are within two parts in $10^{3}$. We can also see that lots of iterations are needed to get an accurate answer. The precise number will depend on the specific problem. \\


\begin{table}[ht]
\centering
\caption{Successive approximations to the root of \cref{eq: equation for iterating} using the rearranged equation. As the convergence is relatively slow we jump several steps at the end to show you it getting closer.}

\vspace{2mm}

\label{table:1}



\begin{tabular}{|c|c|c|} 
 \hline
$n$& $x_{n}$ &$f(x_{n})$\\
 \hline
 $0$ & $1.8$ & $0.99968$ \\
 \hline
$1$ &$1.7805378$ & $0.52247969$ \\
\hline
$2$ &$1.7700175$ & $0.27736234$	\\
\hline
$3$ & $1.7643295$& $0.14849136$\\
\hline
$4$ & $1.761254$& $0.07986324$\\
\hline
$5$ & $1.7595909$ & $0.043059716$\\
\hline
$12$ &$1.75765592$ &$0.00058017842$\\
\hline
$18$ &$1.7576331$& $7.8434818\times10^{-6}$\\
\hline
\end{tabular}
\end{table}

\begin{exercise}
Implement this iteration method numerically  for \cref{eq: equation for iterating} using a programming language or computer program of your choice. I would suggest trying Python or Matlab, but you can use whatever you feel most comfortable with.
\end{exercise}

\subsection*{Linear interpolation}
The next method to look at is \textbf{linear interpolation}.  The idea here is to pick two points $A_{1},B_{1}$ on the graph $y=f(x)$ that lie on either side of the root, i.e. so that $f(A_{1})$ and $f(B_{1})$ have opposite signs. We can then think about the straight line\footnote{This straight line is known as a chord.} joining these points $(A_{1},f(A_{1}))$ to $(B_{1},f(B_{1}))$. An example of this is shown in \cref{fig: interpolation polynomial 2} where $A_{1}=1.5$ and $B_{1}=1.8$.\\

The intersection of this straight line with the $x$-axis is given by
\begin{equation*}
x_{1}=\frac{A_{1}f(B_{1})-B_{1}f(A_{1})}{f(B_{1})-f(A_{1})},
\end{equation*}
We then replace $A_{1}$ or $B_{1}$ by $x_{1}$, find the point on the curve corresponding to $x=x_{1}$ and then iterate this process using the interpolation formula
\begin{equation}
x_{n}=\frac{A_{n}f(B_{n})-B_{n}f(A_{n})}{f(B_{n})-f(A_{n})}.
\label{eq: linear interpolation formula}
\end{equation}


\begin{figure}[ht]
    \centering
\ThisAltText{Graph of the function that we are studying via iteration with the linear regression chord shown.}
 %   \pdftooltip{
    \begin{tikzpicture}[line width=1pt,line cap=round,line join=round, smooth,variable=\x]
     \draw[->] (-0.3,0) -- (6.3,0) node[above] {$x$};
  \draw[->] (0,-4.5) -- (0,3)node[above]{$y$};
 \draw[color=CDnavy, domain=0:5.58]   plot[samples=300] (\x,{(0.33^(5))*\x^(5)-3*(0.33^(3))*\x^(3)+2*(0.33)*\x-4}) node[right] {$f(x)=x^{5}-3x^3+2x-4$};
\filldraw[black] (1.5,0) circle (1pt) node[anchor=north]{$0.5$};
\filldraw[black] (3,0) circle (1pt) node[anchor=north]{$1$};
\filldraw[black] (4.5,0) circle (1pt) node[anchor=north]{$1.5$};
\filldraw[black] (6,0) circle (1pt) node[anchor=north]{$2$};
\filldraw[black] (0,-4) circle (1pt) node[anchor=east]{$-4$};
\filldraw[black] (0,-3) circle (1pt) node[anchor=east]{$-3$};
\filldraw[black] (0,-2) circle (1pt) node[anchor=east]{$-2$};
\filldraw[black] (0,-1) circle (1pt) node[anchor=east]{$-1$};
\filldraw[black] (0,1) circle (1pt) node[anchor=east]{$1$};
\filldraw[black] (0,2) circle (1pt) node[anchor=east]{$2$};
  \draw[color=CDred, -] (4.55,-3.53125) -- (5.45,0.99968 );
  \filldraw[color=CDred] (4.55,-3.53125) circle (1pt) node[anchor=east]{$A_{1}$};
    \filldraw[color=CDred] (5.45,0.99968) circle (1pt) node[anchor=east]{$B_{1}$};
    \end{tikzpicture}
%    }{plots of an exponential and logarithm }
    \caption{A plot of the function $f(x)=x^{5}-3x^3+2x-4$ for $0\leq x\leq 1.86$ with the first linear interpolation chords shown in red.}
        \label{fig: interpolation polynomial 2}
\end{figure}

With each iteration the chord will become shorter and the end points will move until the value of $x_{n}$ converges to the precise value of the root.  The first few steps of this interpolation starting from $A_{1}=1.5$ and $B_{1}=1.8$ are shown in \cref{table:2}.


\begin{table}[ht]
\centering
\caption{Successive approximations to the root of \cref{eq: equation for iterating} using linear interpolation. This is converging very slowly and changes by around $0.0003$ each iteration. }

\vspace{2mm}

\label{table:2}



\begin{tabular}{|c|c|c|c|c|c|c|} 
 \hline
$n$& $A_{n}$ &$f(A_{n})$& $B_{n}$ & $f(B_{n})$& $x_{n}$& $f(x_{n})$\\
 \hline
 $1$ & $1.5$ & $-3.5315$& $1.8$ &$0.99968$& $1.50003$&  $-3.5310$\\
 \hline
 $2$& $1.50003$&$-3.5310$&$1.8$ &$0.99968$& $1.50006$&$-3.53082$\\
 \hline
 $3$& & & & & & \\
 \hline
 $4$& & & & & &\\
 \hline
 $5$& & & & & &\\
 \hline
\end{tabular}
\end{table}


We can see from this example that the linear interpolation method is much slower than rearrangement method, this will not always be true, but is your first hint that you should play around with several approaches before settling on the one that works the best. If you are writing code for these problems, then it is a good idea to have code that can implement any of the iteration schemes that we are discussing here.

\subsection*{Bisection method}
Another method for finding roots is the Bisection method.  Consider a root $a$ of a function $f(x)$.  The idea here is to guess two points, one on either side of the root $a$, and then keep shrinking the size of the interval until we have an estimate of the value of the root. Since the function is zero at the root, we can check that we have actually picked points on either side of the root by checking that the function has different signs on either side  of the interval.\\

\begin{ex}
Consider the function $f(x)=x^{4}+2x-2$, we start by guessing that the root is contained in the interval $[1/2, 1]$. We can check this by computing that $f(1/2)=-0.9375$ while $f(1)=1$. Since the function changes sign across the interval, we know that it has a zero in the interval. We call the left end of teh interval $a_{0}$ and the right end $b_{0}$, and for each step of refining the interval we will have an $a_{r}$ and a $b_{r}$. To narrow down the position of the root we find the mid point of the interval and evaluate the function there:
\begin{equation*}
c_{1}=\frac{a_{1}+b_{1}}{2}=\frac{1/2 +1}{2}=\frac{3}{4}=0.75.
\end{equation*}
and $f(0.75)=-0.1836$. Since this is still negative we then know that the root lies in the interval $[0.75,1]$ and again we can find the new mid point:
\begin{equation*}
c_{2}=\frac{a_{2}+b_{2}}{2}=\frac{0.75 +1}{2}=0.875,.
\end{equation*}
with $f(0.875)=0.3362$. Since this is positive we can again restrict the interval to $[0.75,0.875]$. We can keep repeating this until we get the desired precision. Though it is much easier to do this in a table,  as seen in \cref{table:bisection}

\begin{table}[ht]
\centering
\caption{Successive approximations to the root of $x^{4}+2x-2$ using the bisection method. }

\vspace{2mm}

\label{table:bisection}



\begin{tabular}{|c|c|c|c|c|c|c|} 
 \hline
 $r$& $a_{r}$ &sign of $f(a_{r})$& $b_{r}$ & sign of $f(b_{r})$& $c_{r}$& sign of  $f(c_{r})$\\
 \hline
 $1$ & $0.500000$ & $-$& $1.000000$ &$+$& $0.750000$&  $-$\\
 \hline
 $2$& $0.750000$&  $-$&$1.000000$ &$+$& $0.875000$&$+$\\
 \hline
 $3$& $0.750000$&  $-$ &$0.875000$&$+$& $0.812500$ & $+$ \\
 \hline
 $4$& $0.750000$&  $-$ & $0.812500$ & $+$ & $0.781250$ & $-$\\
 \hline
 $5$& $0.781250$ & $-$ &$0.812500$ & $+$  & $0.796875$ & $-$\\
 \hline
 $6$& $0.796875$ & $-$ &$0.812500$ & $+$  & $0.804688$ & $+$\\
 \hline
  $7$& $0.796875$ & $-$ &$0.804688$ & $+$  & $0.800781$ & $+$\\
 \hline
   $8$& $0.796875$ & $-$ &$0.800781$ & $+$  & $0.798828$ & $+$\\
 \hline
\end{tabular}
\end{table}

If we want an answer to two decimal places we find that the root is at $x=0.80$. If we kept going we would find that to four decimal places the root is at $x=0.7976$.
\end{ex}


\begin{exercise}
The function $f(x)=x^{3}+3x-5$ has a root between $1$ and $2$, use the method of bisection four times to find the root.
\end{exercise}

\subsection*{Newton's method}
The main numerical method that is needed in this module is \textbf{Newton's method}, sometimes called the \textbf{Newton-Raphson} method. Newton's method finds the root by constructing the tangent to a curve through the initial point $x_{0}$, and then taking the next point $x_{1}$ to be where the tangent line intersects the $x$-axis.\\

\begin{table}[ht]
\centering
\caption{Successive approximations to the root of \cref{eq: equation for iterating} using Newton's method. For this example, Newton's method converges much quicker than the other methods.}

\vspace{2mm}

\label{table:3}



\begin{tabular}{|c|c|c|} 
 \hline
$n$& $x_{n}$ &$f(x_{n})$\\
 \hline
 $0$ & $1.8$ & $0.99968$ \\
 \hline
$1$ &$1.760530638$ & $0.06382986248$ \\
\hline
$2$ &$1.757647406$ & $0.00032122929$	\\
\hline
$3$ & $1.757632748$& $8.267167839\times10^{-9}$\\
\hline
$4$ & $1.757632748$& $-8.881784197\times10^{-16}$\\
\hline
$5$ & $1.757632748$ & $-8.881784197\times10^{-16}$\\
\hline
\end{tabular}
\end{table}

In other words we start from a guess, $x_{0}$, then construct the tangent line
\begin{equation*}
y(x)=\left(x-x_{0}\right)f'(x_{0})+f(x_{0}).
\end{equation*} 
Once we have the tangent we find the value of $x$ such that $y(x)=0$ and call this $x_{1}$. Doing this for a general point in the iteration $x_{n}$ the iteration scheme is
\begin{equation}
x_{n+1}=x_{n}-\frac{f(x_{n})}{f'(x_{n})}.
\label{eq: Newton iteration scheme}
\end{equation}

An important observation here is that if the points $x_{n}$ get close to a critical point of $f(x)$ then the scheme will break down as then $f'(x_{n})$ will be approaching zero. This means that we need to be very careful about picking our starting point, if we pick one that is on the opposite side of a critical point from a root then we may not be able to get to the root. \Cref{fig: Newton iteration} shows what the first step in Newton's method looks like graphically.\\

%%% This plot is not working very well as zooming in on the polynomial it looks to close a straight line so we cannot see the tangent.
%\begin{figure}[ht]
%    \centering
%\ThisAltText{Graph of the function that we are studying via iteration.}
% %   \pdftooltip{
%    \begin{tikzpicture}[line width=1pt,line cap=round,line join=round, smooth,variable=\x, domain=1.7:1.86, yscale = 1, xscale = 54]
%%     \draw[->] (4.2,0) -- (6.3,0) node[above] {$x$};
%%  \draw[->] (4.2,-4.2) -- (4.2,3)node[above]{$y$};
%% \draw[color=CDnavy, domain=4.2:5.58]   plot[samples=300] (\x,{(0.33^(5))*\x^(5)-3*(0.33^(3))*\x^(3)+2*(0.33)*\x-4}) node[right] {$f(x)=x^{5}-3x^3+2x-4$};
%%%\filldraw[black] (1.5,0) circle (1pt) node[anchor=north]{$0.5$};
%%%\filldraw[black] (3,0) circle (1pt) node[anchor=north]{$1$};
%%\filldraw[black] (4.5,0) circle (1pt) node[anchor=north]{$1.5$};
%%\filldraw[black] (6,0) circle (1pt) node[anchor=north]{$2$};
%%\filldraw[black] (4.2,-4) circle (1pt) node[anchor=east]{$-4$};
%%\filldraw[black] (4.2,-3) circle (1pt) node[anchor=east]{$-3$};
%%\filldraw[black] (4.2,-2) circle (1pt) node[anchor=east]{$-2$};
%%\filldraw[black] (4.2,-1) circle (1pt) node[anchor=east]{$-1$};
%%\filldraw[black] (4.2,1) circle (1pt) node[anchor=east]{$1$};
%%\filldraw[black] (4.2,2) circle (1pt) node[anchor=east]{$2$};
%
%% draw axes
%\draw [->] (1.7,0) -- (1.9,0) 	node [anchor=west]		{$x$};
%\draw [->] (1.7,-3) -- (1.7,3) 	node [anchor=south]	{$y$};
%
%% x-axis tick marks
%\foreach \x in {1.75, 1.80, 1.85}
%	\draw (\x, 3pt) -- (\x,-3pt) node [anchor=north] {$\x$};
%	
%% y-axis tick marks
%%\draw (1.6,-4)--(1.6,-4)node [anchor=east] {$-4$};
%\draw (1.7,-3)--(1.7,-3)node [anchor=east] {$-3$};
%\draw (1.7,-2)--(1.7,-2)node [anchor=east] {$-2$};
%\draw (1.7,-1)--(1.7,-1)node [anchor=east] {$-1$};
%\draw (1.7,1)--(1.7,1)node [anchor=east] {$1$};
%\draw (1.7,2)--(1.7,2)node [anchor=east] {$-2$};
%
%%Newton's method lines
%\draw (1.8,3pt)--(1.8,-3pt)node [anchor=south] {$x_{0}$};
%\draw[dotted] (1.8,0) -- (1.8,0.99968);
%\draw[color = CDred] (1.8,0.99968)--(1.76,0) node[anchor = south]{$x_{1}$};
%
%    % draw function
%    \draw[color=CDnavy] plot (\x,{\x^(5)-3*\x^(3)+2*\x-4});
%    \end{tikzpicture}
%%    }{plots of an exponential and logarithm }
%    \caption{A zoomed in plot of the function $f(x)=x^{5}-3x^3+2x-4$  showing the first step of Newton's method, after one step, the result is already so close that we cannot see the other steps if they are plotted.}
%        \label{fig: interpolation polynomial zoomed}
%\end{figure}

\begin{figure}[ht]
    \centering
    %\pdftooltip{
    \includegraphics[width=0.5\textwidth, alt ={A schematic of the first step of Newton's method for finding roots}]{figures/Newton_iteration}
    %}{A schematic of a derivative as a tangent to a curve. }
    \caption{The first step of Newton's iteration method to find the root of $f(x)$. We pick a point $x_{0}$ then find the tangent to $f(x)$ at $x_{0}$ and solve for the point $x_{1}$ where this tangent intersects the $x$-axis. This is repeated until we converge on the root.}
\label{fig: Newton iteration}
\end{figure}


For the specific equation in \cref{eq: equation for iterating} we have that
\begin{equation}
x_{n+1}=x_{n}-\frac{x_{n}^{5}-3x_{n}^3+2x_{n}-4}{5x_{n}^{4}-9x_{n}^{2}+2}.
\end{equation}

Starting from $x_{0}=1.8$ we get the sequence in \cref{table:3}. This iteration scheme converges to the precise answer of \cref{eq: precise root} much quicker than the other methods.


Newton's method is the approach that you need to be able to use during this module so we will treat a couple more examples using it. There are a couple more examples on \citep{calcI} if you want to see more.

\begin{ex}
Use Newton's method to determine an approximation to the solution to $\sin{x} = -x$ that lie in the interval $[-1,1]$. Find the approximation to $6$ decimal places.\\

First we need to pick a starting point $x_{0}$. Here we will take $x_{0}=0.5$, you can get an idea of what value to pick by plotting the function. For this example we have that
\begin{equation}
f(x)=\sin(x)+x=0,
\label{eq: sin +x roots}
\end{equation}
so the iteration equation for Newton's method \cref{eq: Newton iteration scheme} becomes
\begin{equation*}
x_{n+1}=x_{n}-\frac{\sin{x_{n}}+x_{n}}{\cos{x_{n}}+1}.
\end{equation*}

The first approximation is then
\begin{equation*}
x_{1}=\frac{1}{2}-\frac{\sin{\frac{1}{2}}+\frac{1}{2}}{\cos{\frac{1}{2}}+1}=-0.0216418
\end{equation*}

\begin{table}[ht]
\centering
\caption{Successive approximations to the root of \cref{eq: sin +x roots} using Newton's method.}

\vspace{2mm}

\label{table:4}



\begin{tabular}{|c|c|c|} 
 \hline
$n$& $x_{n}$ &$f(x_{n})$\\
 \hline
 $0$ & $0.5$ & $0.979426$ \\
 \hline
$1$ &$-0.021642$ & $-0.043282$ \\
\hline
$2$ &$2\times10^{-6}$ & $3.\times10^{-6}$	\\
\hline
$3$ & $0$& $0$\\
\hline
\end{tabular}
\end{table}

The further iteration steps are shown in \cref{table:4}, and we see that to 6 decimal places, it only takes $4$ iterative steps to find that the root is at $x=0$. If we did any more steps we would see that these remain at $x=0$.

\end{ex}

Now we can consider one of the examples from \citep{calcI} which shows when Newton's method does not work.

\begin{ex}
Starting from $x_{0}=1$ we will apply Newton's method to $\sqrt[3]{x}$.\\

Intuitively it is clear that the root is $x=0$. However, Newton's method will not find this root. For this example \cref{eq: Newton iteration scheme} becomes
\begin{equation*}
x_{n+1}=x_{n}-\frac{\sqrt[3]{x}}{\frac{1}{3}x^{-\frac{2}{3}}}=x_{n}-3x_{n}=-2x_{n}.
\end{equation*}

This already tells us that instead of converging to $x=0$, Newton's method will diverge with $x_{1}=-2$, $x_{2}=4$, $x_{3}=-8$, $x_{4}=16$, \dots{}. This is the opposite of what we want. Fortunately, it became obvious quite quickly that the method was not working and we did not have to waste much time before discovering that we needed to used a different model.
\end{ex}

\begin{exercise}
Write a computer program to implement Newton's method and use it to find the roots in the examples above, where Newton's method works.
\end{exercise}

\subsection*{Secant method}
A variant of Newton's method is the \textbf{Secant} method, where we start with two estimates of the root, $x_{0}$ and $x_{1}$ and estimate the derivative as being the Newton quotient
\begin{equation*}
Q(x_{0},x_{1})=\frac{f(x_{1})-f(x_{1})}{x_{1}-x_{0}},
\end{equation*}
so that
\begin{equation*}
x_{2}=x_{1}-\frac{f(x_{1})}{Q(x_{0},x_{1})}=x_{1}-\frac{f(x_{1})}{\frac{f(x_{1})-f(x_{1})}{x_{1}-x_{0}}}.
\end{equation*}

This means that the iteration scheme is given by
\begin{equation}
x_{n+1}=x_{n}-\frac{f(x_{n})}{Q(x_{n-1},x_{n})}=x_{n}-\frac{f(x_{n})}{\frac{f(x_{n})-f(x_{n-1})}{x_{n}-x_{n-1}}}
\label{eq: secant method}
\end{equation}

Geometrically, the secant method can also be viewed as constructing a line nearby our function and moving to the point where this line cuts the $x$-axis. However, this line is no longer a tangent line, but is a secant. 

\begin{ex}
For 
\begin{equation*}
f(x)=x^{2}-10,
\end{equation*}
and taking $x_{0}=2$, $x_{1}=3$ we find $x_{2}$ using the secant method as follows:

\begin{align*}
x_{2}&=x_{1}-\frac{f(x_{1})}{\frac{f(x_{1})-f(x_{0})}{x_{1}-x_{0}}}\\
	&=3-\frac{-1}{\frac{-1-(-6)}{3-2}}\\
	&=3+\frac{1}{7}\\
	&=3.14286.
\end{align*}
This is an equation that we can solve directly by the rearrangement method since $f(x)=0$ corresponds to $x^{2}=10$ so the exact answers are $x=-\sqrt{10}\simeq -3.16228$ and $x=\sqrt{10}\simeq 3.16228$. We can see that $x_{2}$ is already approaching this exact answer.

\end{ex}

\subsection*{Convergence and errors}
The absolute error of a numerical method is the difference between the true solution $x$ and the approximate solution $\xi$, $\epsilon =\vert x-\xi\vert$. It can be hard to estimate the absolute error unless we already know the exact solution. \\

We say that an iteration scheme converges when it subsequent iterations do not change the value of $x_{n}$, at least to the accuracy that we are looking for. Not every method will converge for a given problem, We have already seen that Newton's method does not converge for the cube root, $\sqrt[3]{x}$. You will gain experience of which methods works best for which type of of problem. In this module, you will mainly be using Newton's method so you need to remember that it can fail.

\section{Approximating functions}
Through much of this module we have been discussing functions and how to work with them. Now we will look at a method for approximating functions using polynomials. This is one of the reasons that we spent so long discussing polynomials at the start. If we can understand polynomials then we can get a good idea of how many functions behave.\\

In essence this function is all about fitting data points, or establishing if there is a relationship or trend between known data. We are not framing it like this here but if you go on to take any statistics or data analysis modules then the approximation ideas here will prove useful to you.

If we have time, in \cref{sec:advanced topics} we will discuss the continuous analogue of this which goes by the name of the \textbf{Taylor} or \textbf{Maclaurin} series. However, to understand these properly, we need to know how to calculate multiple derivative, which is just at the edge of what is included in this module. In this section we will be doing this using two discrete methods, one known as the \textbf{Newton interpolating polynomial}, or \textbf{Newton's divided difference method}, and the other \textbf{Lagrange's method}.

\subsection*{Newton's method}

The idea is that idea is that given some values of a function, we can find a polynomial which passes through them. The example given in \cite{lissamen2004mei} is to consider the growth rate per year for a group of young people. The results are given in \cref{table:heights} and can be plotted as five points on a graph. The question asked, and answered, in \cite{lissamen2004mei} is can a formula be found which fits all of these points. The formula given in \cite{lissamen2004mei} is
\begin{equation*}
f(x)=4.13984375+3.34791667x-0.91315104x^{2}+0.08177083x^{3}-0.00232747x^{4}.
\end{equation*}
The intuition behind this is that we first guess that the function has the form of a straight line $f(x)=mx+c$, notice that this does not fit the data, then add higher order monomials such as $x^{2},x^{3},x^{4}$ with appropriate sized coefficients until we get the correct shape and the curve passes through, or close to, the points. \\

By the end of this section you will know how to calculate the approximate function which fits the given data.\\


\begin{table}[ht]
\centering
\caption{Measurements of the average height increase per year at various ages taken from \cite{lissamen2004mei}.}

\vspace{2mm}

\label{table:heights}



\begin{tabular}{|c|c|c|c|c|c|} 
 \hline
Age in years& $2$ &$6$ & $10$& $14$& $18$\\
 \hline
Rate of Growth in cm per year & $7.8$ & $6.0$ & $4.8$&$7.0$&$1.1$\\
 \hline
\end{tabular}
\end{table}


\begin{figure}[ht]
    \centering
\ThisAltText{Graph of the function that fits the heights.}
    \begin{tikzpicture}[line width=1pt,line cap=round,line join=round, smooth,variable=\x]
     \draw[->] (-0.2,0) -- (9.2,0) node[below] {Age (years)};
  \draw[->] (0,-0.2) -- (0,8)node[above]{Growth rate (cm per year)};
 \draw[color=CDnavy, domain=0:9]   plot[samples=300] (\x ,{ 4.13984375 + (3.34791667*2)*\x - (0.91315104*(2)^(2))*\x^(2)+(0.08177083*(2)^(3))*\x^(3) - (0.00232747*(2)^(4))*\x^(4)}) ;
 \draw[step=1cm,gray,very thin] (0,0) grid (9,8);
 %node[right] {$f(x)=x^{5}-3x^3+2x-4$};
\filldraw[black] (1,7.8) circle (1pt);
\filldraw[black] (3,6.0) circle (1pt) ;
\filldraw[black] (5,4.8) circle (1pt);
\filldraw[black] (7,7.0) circle (1pt) ;
\filldraw[black] (9,1.1) circle (1pt) ;
    \end{tikzpicture}
    \caption{A plot of the function which fits the heights from \cref{table:heights} with an approximate polynomial fitted. The polynomial does not quite go through all of the points but passes close to all of them, this is an artefact of the plot being scaled, This will be fixed soon.}
        \label{fig: heights fit}
\end{figure}

Following \cite{lissamen2004mei}, we will use the notation in \cref{table:notation}. We start our labelling from the left most point and moving rightwards until we get to the last point. Here we are taking $x$ values that have a constant spacing which we denote by $h$\footnote{If this seems suggestive of our notation when doing differentiation from first principles, that is on purpose. The separation between points is precisely what we want to take to zero to get the derivative. }.\\


\begin{table}[ht]
\centering
\caption{Our standard notation for putting a list of $n$ points and the values the true function takes at these points into a table.}

\vspace{2mm}

\label{table:notation}



\begin{tabular}{|c|c|c|c|c|c|c|c|} 
 \hline
$x$& $x_{0}$ &$x_{1}$ & $x_{2}$& $x_{3}$& $x_{4}$&\dots&$x_{n}$\\
 \hline
$f(x)$ & $f(x_{0})$ & $f(x_{1})$ & $f(x_{2})$&$f(x_{3})$&$f(x_{4})$&\dots&$f(x_{n})$\\
 \hline
\end{tabular}
\end{table}

One approach to finding the polynomial is via a finite difference table. This is a means of keeping track of the points, the values of the function at the points, the differences between the values of the function at neighbouring points, the differences between the differences, and so on.\\

As usual we take $\Delta$ to mean the change in some quantity. Here it is technically being used to denote the \textbf{forward difference operator},
\begin{equation}
\Delta f_{0}=f_{1}-f_{0}, \quad \Delta f_{1}=f_{2}-f_{1}, \quad \dots, \Delta f_{n-1}=f_{n}-f_{n-1}.
\label{eq: forward differences}
\end{equation}

This is called the forward difference because it is defined as $f_{1}-f_{2}$, e.g we subtract the value of the function at the point on the left from the value at the point on the left. There is a corresponding \textbf{backward difference operator} coming from considering $f_{0}-f_{1}$. We will not consider backwards differences here, but you can come up with an alternative method using them. If you go into numerical analysis, or solving real world problems by writing computer programmes then you will need to be familiar with both methods.\\

The differences between differences are then denoted
\begin{equation}
\Delta^{2}f_{0}=\Delta f_{1}-\Delta f_{0}, \quad \Delta^{2}f_{1}=\Delta f_{2}-\Delta f_{1}, \dots.
\end{equation}
If we wanted to we can keep going with differences of differences of differences etc. How far you go will depend on the specific problem,  as well as how precise you want your approximation to be. 

A convenient way to keep track of everything is to put it all in a table as in \cref{table: fd1} which has four points and works up to third order differences $\Delta^{3}f_{i}$. If we wanted to include more differences then we would need to add more columns. 


\begin{table}[ht]
\centering
\caption{A finite difference table with four points working up to the third differences.}

\vspace{2mm}

\label{table: fd1}



\begin{tabular}{|c|c|c|c|c|} 
 \hline
$x_{i}$& $f_{i}$ & $\Delta f_{i}$ & $\Delta^{2} f_{i}$ & $\Delta^{3}f_{i}$\\
 \hline
$x_{0}$ & $f_{1}$ & & & \\
 \hline
 & & $\Delta f_{0}=f_{1}-f_{0}$ & & \\
 \hline
 $x_{1}$&$f_{1}$& &$\Delta^{2}f_{0}=\Delta f_{1}-\Delta f_{0} $ & \\
 \hline
 & & $\Delta f_{1}=f_{2}-f_{1}$ & &$\Delta^{3}f_{0}=\Delta^{2}f_{1}-\Delta^{2}f_{0}$ \\
 \hline
 $x_{2}$&$f_{2}$& &$\Delta^{2}f_{1}=\Delta f_{2}-\Delta f_{1}$  & \\
 \hline
  & &$\Delta f_{2}=f_{3}-f_{2}$ & & \\
 \hline
 $x_{3}$&$f_{3}$& & & \\
 \hline
\end{tabular}
\end{table}

\begin{ex}
If we consider the data in \cref{table:heights} and work to fourth order differences we get the table in \cref{table: fd ex1}. This finite difference table is what we can use to construct the approximate function.
\begin{table}[ht]
\centering
\caption{A finite difference table with four points working up to the third differences for the data in \cref{table:heights}.}

\vspace{2mm}

\label{table: fd ex1}



\begin{tabular}{|c|c|c|c|c|} 
 \hline
$x_{i}$& $f_{i}$ & $\Delta f_{i}$ & $\Delta^{2} f_{i}$ & $\Delta^{3}f_{i}$\\
 \hline
$2$ & $7.8$ & & & \\
 \hline
 & & $6-7.8 =-1.8$ & & \\
 \hline
 $4$&$6$& &$0-(-1.8)=1.8$ & \\
 \hline
 & & $6-6=0$ & &$0.5-1.8=-1.3$ \\
 \hline
 $6$&$6$& &$0.5-0=0.5$  & \\
 \hline
  & &$6.5-6=0.5$ & & \\
 \hline
 $8$&$6.5$& & & \\
 \hline
\end{tabular}
\end{table}
\end{ex}

\begin{ex}
Consider the quadratic function $f(x)=x^{2}+1$, working up to fourth order differences, this gives the finite difference table in \cref{table: fd ex2}. Notice that the third and fourth differences all vanish. 

\begin{table}[ht]
\centering
\caption{A finite difference table with four points working up to the fourth order differences for the function $f(x)=x^{2}+1$.}

\vspace{2mm}

\label{table: fd ex2}



\begin{tabular}{|c|c|c|c|c|c|} 
 \hline
$x_{i}$& $f_{i}$ & $\Delta f$ & $\Delta^{2} f$ & $\Delta^{3}f$&$\Delta^{4}f$\\
 \hline
$0$ 	& $1$	& 		&	 & 	& \\
 \hline
 	& 		& $1$ 	& 	& 	&\\
 \hline
 $1$	&$2$		& 		&$2$ & 	& \\
 \hline
	 &  		& $3$ 	& 	&$0$	& \\
 \hline
 $2$	&$5$		& 		&$2$  & 	&$0$\\
 \hline
  	& 		&$5$ 	& 	& $0$&\\
 \hline
 $3$	&$10$	& 		&$2$ & 	&\\
 \hline
  	& 		&$7$ 	& 	& 	&\\
 \hline
  $4$	&$17$	& 		& 	&	& \\
 \hline
\end{tabular}
\end{table}
\end{ex}

\begin{exercise}
Construct the finite difference table, up to third order differences, for the linear function $f(x)=2x+1$.
\end{exercise}
\begin{exercise}
Construct the finite difference table, up to third order differences, for a cubic function of your choice. How many differences do you think you need to include?
\end{exercise}

It is a general property of polynomials that for a polynomial with highest order term $x^{n}$, the $n$'th differences will be constant and all higher order differences will vanish.

The polynomial built from the finite difference table is known as the interpolating polynomial. This gives the interpolating, or approximating function as
\begin{equation}
f(x)=f_{0}+\frac{x-x_{0}}{h}\Delta f_{0}+\frac{(x-x_{0})(x-x_{1})}{2!h^{2}}\Delta^{2}f_{0}+\frac{(x-x_{0})(x-x_{1})(x-x_{2})}{3!h^{3}}\Delta^{3}f_{0}+\dots,
\label{eq: NIP}
\end{equation}
with $h$ the constant spacing between the $x$ values.\\

This may look complicated, but there is a set pattern to the terms, and the sum will terminate, either when all of the $\Delta^{a}f_{0}=0$ or because of the number of data points we are given.\\

When we only have two points this method is called \textbf{linear interpolation} as only the first two terms in \cref{eq: NIP} are non-zero.\\

\begin{ex}
Consider the finite difference table in \cref{table: fd ex2}. The corresponding Newton interpolating polynomial is found as follows:
\begin{itemize}
\item $h=1$,
\item $\Delta f_{0}=1$,\\
\item $\Delta^{2}f=2$, \\
\item $\Delta^{3}f=0$,
\end{itemize}
and we know what the $x_{i}$ are, so we can substitute everything in to find:
\begin{align*}
f(x) 	&= 1+x+\frac{x(x-1)}{2}2\\
	&=1+x-x+x^{2}=1+x^{2}.
\end{align*}
This is the polynomial that we started with which is a good sign.
\end{ex}

\begin{exercise}
Consider the finite difference table in \cref{table: fd ex1}, and find the Newton interpolation polynomial for this data.
\end{exercise}

\begin{exercise}
 Consider the data in \cref{table: data table for exercise} and find the polynomial that fits the data.
\begin{table}[ht]
\centering
\caption{A table of data that you need to find the Newton polynomial for.}

\vspace{2mm}

\label{table: data table for exercise}



\begin{tabular}{|c|c|c|c|c|} 
 \hline
$x$& $2$ &$3$ & $4$& $5$\\
 \hline
$f(x)$ & $1$ & $1$ & $2$&$2$\\
 \hline
\end{tabular}
\end{table}

Then use your approximation to find the value of $f(4.3)$.

\end{exercise}

Note that if all of the differences of a given order are approximately constant then we can sometimes get away with a lower order polynomial approximation, even if the higher order terms do not actually vanish.

\subsection*{Lagrange's method}
There is an alternative method due to Lagrange, where we do not need to start with equally spaced points. The easiest way to understand this method is to start by considering the linear case, then the quadratic case, before giving the general expression. This is also discussed briefly in \cite{lissamen2004mei}.\\

The linear case of Lagrange's equation is very similar to \cref{eq: linear interpolation formula} that we met when discussing linear interpolation as a method of root finding. \\

Given two points $(a,A)$ and $(b,B)$, the straight line  that passes through them is given by
\begin{equation*}
f(x)=\frac{A(x-b)}{a-b}+\frac{B(x-a)}{b-a}
\end{equation*}
where we need that $a\neq b$.\\

If we have three points, $(a,A), (b,B), (c,C)$ then they lie on a quadratic given by
\begin{equation*}
f(x)=\frac{A(x-b)(x-c)}{(a-b)(a-c)}+\frac{B(x-c)(x-a)}{(b-c)(b-a)}+\frac{C(x-a)(x-b)}{(c-a)(c-b)},
\end{equation*}
where we now need that $a,b,c$ are non equal.\\

This approach can be generalised to a polynomial going through $n$ points. This will be a polynomial of degree $n-1$ for $n$ points.

\begin{exercise}
Find the expression for the degree $3$ polynomial going through four points, then use the pattern that you can see between the line, quadratic, and cubic, cases to find the general expression for the polynomial passing through $n$ points.
\end{exercise}

\section{Numerical integration}
\textcolor{red}{This section needs more images added. There are graphs for the mid point rule case but not for the trapezium rule case. These will be added soon.}

We now turn to numerical methods of integration. We have already carried out some numerical integration when we discussed definite integrals and the relationship between integrals and sums. In fact the first method we will use to carry out numerical integration, the \textbf{midpoint rule}, is one we have already seen.  We will also discuss two other related rules that estimate the area under a graph, the \textbf{trapezium rule} and \textbf{Simpson's rule}. These are not independent methods, and Simpson's rule is actually a weighted average of the mid point and trapezium rule which improves the accuracy.\\

In practice, when you ask a computer to carry out a definite integration, unless you use a symbolic computation method in \textbf{Maple}, \textbf{Mathematica}, or similar, the computer will be implementing one of these rules. This means that by understanding how they work and how to implement them by hand, you will be able to code up your own numerical integrator. I recommend this as a challenge to an keen students who want to make practical use of what you are learning in this module and I will put an exercise at the end of each section suggesting that you do this for each of the rules.

\subsection*{Midpoint rule}

The mid point rule is the approximation approach that we met in \cref{sec:integration}  when approximating the area under the curve for the function $f(x)=x^{2}+1$. Let's recap it here.\\

Consider the figure in \cref{fig: first integral approx}, as we saw earlier we will be approximating the area between the curve and the $x$-axis, using rectangles of width $\Delta x$ and height $\Delta y$.

\begin{figure}[ht]
    \centering
\ThisAltText{Graph of the function x squared plus one.}
    \begin{tikzpicture}[line width=1pt,line cap=round,line join=round, smooth,variable=\x]
     \draw[->] (-0.2,0) -- (5,0) node[below] {$x$};
  \draw[->] (0,-0.2) -- (0,5.5)node[above]{$y$};
 \draw[color=CDnavy, domain=0:4]   plot[samples=300] (\x ,{1+ ((0.5)^(2))*\x^(2)}) ;
 \draw[-, color =CDred ] (0,0) -- (0,1) ;
  \draw[-, color =CDred ] (4,0) -- (4,5);
\filldraw[black] (0,1) circle (1pt)node[anchor=east]{$1$};
\filldraw[black] (0,-0.2) circle (1pt) node[anchor=north]{$0$};
\filldraw[black] (4,0) circle (1pt) node[anchor=north]{$2$};
    \end{tikzpicture}
    \caption{A plot of our favourite function $f(x)=x^{2}+1$.}
        \label{fig: first integral approx}
\end{figure}

As before, w e do this by splitting up the interval, $[0,2]$ in this case, into $n$ subintervals. This will give us an approximation of the area which will improve the more sub integrals we take, in other words the larger that $n$ is. The width of a subinterval is then 
\begin{equation*}
\Delta x=\frac{2-0}{n}=\frac{2}{n}.
\end{equation*}
In general it will be $\Delta x=(b-a)/n$, when the interval is $[a,b]$. \\

Before we said that we could pick the height of the intervals to be any value the function takes within the sub interval. As the numerical approach that we are following is called the midpoint rule, unsurprisingly, we take the height to be the value of the function at the midpoint of the subinterval. Recall that for an interval $[a,b]$ the midpoint is $c=(a+b)/2$. This method is shown in \cref{fig: approximate area mid point}. 

\begin{figure}[ht]
    \centering
\ThisAltText{Graph of the function x squared plus one with rectangles approximating the area under the curve.}
    \begin{tikzpicture}[line width=1pt,line cap=round,line join=round, smooth,variable=\x]
     \draw[->] (-0.2,0) -- (5,0) node[below] {$x$};
  \draw[->] (0,-0.2) -- (0,5.5)node[above]{$y$};
 \draw[color=CDnavy, domain=0:4]   plot[samples=300] (\x ,{1+ ((0.5)^(2))*\x^(2)}) ;
  \draw[-, color =CDgreen ] (4,0) -- (4,4.0625);
\filldraw[black] (0,1) circle (1pt)node[anchor=east]{$1$};
\filldraw[black] (0,-0.2) circle (1pt) node[anchor=north]{$0$};
\filldraw[black] (4,0) circle (1pt) node[anchor=north]{$2$};
\draw[-, color =CDgreen ] (1,0) -- (1,1.5625) ;
\draw[-, dashed] (0.5,0) -- (0.5,1.0625);
\draw[-, dashed](1.5,0)--(1.5,1.5625);
\draw[-, dashed] (2.5,0) -- (2.5,2.5625);
\draw[-, dashed](3.5,0)--(3.5,4.0625);
\draw[-, color =CDgreen ] (2,0) -- (2,2.5625);
\draw[-, color =CDgreen ] (3,0) -- (3,4.0625) ;
\draw[-, color =CDgreen ] (4,0) -- (4,4.0625) ;
 \draw[-, color =CDgreen ] (0,1.0625) -- (1,1.0625);
\draw[-, color =CDgreen ] (1,1.5625) -- (2,1.5625);
\draw[-, color =CDgreen ] (2,2.5625) -- (3,2.5625) ;
\draw[-, color =CDgreen ] (3,4.0625) -- (4,4.0625);
    \end{tikzpicture}
    \caption{A plot of  $f(x)=x^{2}+1$ with rectangles approximating the area under the curve.}
        \label{fig: approximate area mid point }
\end{figure}

As we are taking four subintervals they are
\begin{equation*}
\left[0,\frac{1}{2}\right], \quad \left[\frac{1}{2},1\right], \quad \left[1,\frac{3}{2}\right], \quad \left[\frac{3}{2},2\right],
\end{equation*}
which all have width $\Delta x=1/2$. The heights are
\begin{equation*}
f\left(\frac{1}{4}\right)=\frac{17}{16}, \quad f\left(\frac{3}{4}\right)=\frac{25}{16}, \quad f\left(\frac{5}{4}\right)=\frac{41}{16}, f\left(\frac{7}{4}\right)=\frac{65}{16}.
\end{equation*}
The area of each rectangle is then
\begin{align*}
A_{1}&=\Delta x\times f\left(\frac{1}{4}\right)=\frac{1}{2}\times\frac{17}{16}=\frac{17}{32},\\
A_{2}&=\Delta x\times f\left(\frac{3}{4}\right)=\frac{1}{2}\times\frac{25}{16}=\frac{25}{32},\\
A_{3}&=\Delta x\times f\left(\frac{5}{4}\right)=\frac{1}{2}\times\frac{41}{16}=\frac{41}{32},\\
A_{4}&=\Delta x\times f\left(\frac{7}{4}\right)=\frac{1}{2}\times\frac{65}{16}=\frac{65}{32}.
\end{align*}

The total area, $A_{m}$ since we are using the midpoint rule,  is thus
\begin{equation*}
A_{m}=A_{1}+A_{2}+A_{3}+A_{4}=\frac{17}{32}+\frac{25}{32}+\frac{41}{32}+\frac{65}{32}=\frac{148}{32}=\frac{37}{8}\simeq 4.625.
\end{equation*}

This was a result that we quoted in \cref{sec:integration}, but did not give all the details for. We also commented on the fact that this was pretty close to the true value of $A=14/3=4.67$ found by carrying out the definite integral.\\

This shows that we do not always need to use that many sub intervals to get a decent estimate of the integral using the midpoint rule.\\

We have given one example here, and you have checked at least one more on the problem sheets. However, this idea is general.\\

Given a function $f(x)$ and an interval $[a,b]$ you should use the following steps:
\begin{itemize}
\item pick a number of sub intervals, or use the number given in the question, $n$, and split your interval into $n$ sub intervals $[a_{0},a_{1}], [a_{1},a_{2}], \dots [a_{n-1},a_{n}]$, of fixed width $\Delta x =(b-a)/n$.
\item find the mid point of each sub interval $c_{i}=(a_{i}-a_{i-1})/2$ and take the height of the rectangle to be the value of the function at this point, $\Delta y_{i}=f(c_{i})$.
\item Calculate the area of each of the rectangles as $A_{i}=\Delta x \times \Delta y_{i}$.\\
\item Sum up all of the areas of the rectangles and get your estimate of the total area
\begin{equation*}
A_{m}=A_{1}+\dots A_{n}=\sum_{i=1}^{n}A_{i}.
\end{equation*}
and use this as your approximation to the definite integral.
\end{itemize}

In essence we are treating the problem the opposite way round to in \cref{sec:integration}, where we used this approximate picture and discussed how this became the definite integral in the limit of an infinite number of very thin rectangles. Here we are saying that we can use a finite number of rectangles to estimate the definite integral.\\

Note that in many of the cases you are asked about here you can also compute the integrals directly. This enables you to check how close your answer is to the true answer. The real power of this method is that you can approximate the value of integrals that cannot be calculated directly. or which are very hard to calculate directly.

\begin{ex}
Consider the definite integral
\begin{equation*}
I=\int_{0}^{1}\sqrt{x^{3}+1}\ud x,
\end{equation*}
we can evaluate this with the midpoint rule for $n=5$ strips as follows. Since $n=5$ the width of each strip is $\Delta x = (1-0)/5=1/5=0.2$.
\end{ex}

\begin{exercise}
Consider the function $f(x)=x^{2}-2$ on the interval $[0,2]$. Take $n=8$ and use the midpoint rule to approximate the definite integral
\begin{equation*}
I=\int_{0}^{2}\left(x^{2}-2\right)\ud x.
\end{equation*}
\end{exercise}


\begin{exercise}
Consider the function $f(x)=x^{3}+x^{2}-x-1$ on the interval $[-1,1]$. Take $n=6$ and use the midpoint rule to approximate the definite integral
\begin{equation*}
I=\int_{-1}^{1}\left(x^{3}+x^{2}-x-1\right)\ud x.
\end{equation*}
How does this compare to calculating the definite integral directly? 
\end{exercise}
\begin{exercise}
Consider the function $f(x)=\sin(x)$ on the interval $[0,\uppi]$. Take $n=8$ and use the midpoint rule to approximate the definite integral
\begin{equation*}
I=\int_{0}^{\uppi}{\sin(x)}\ud x.
\end{equation*}
\end{exercise}


\subsection*{Trapezium rule}

The next rule is known as the trapezium rule. It is very similar to the midpoint rule in that we are approximating the area under the curve using shapes that we know how to calculate the area of. However, here we use trapezia instead of rectangles. \\

A trapezium is like a rectangle glued onto a triangle, in general it is a four sided shape where at least one pair of sides are parallel. So squares and rectangles are examples of trapezia.\\

Consider a function $f(x)$ on the interval $[a,b]$. If the length of the left edge of the trapezium is $f(a)$ and the length of the right edge is $f(a+\Delta x)$, while its width is $\Delta x$, then the area is
\begin{equation*}
A_{\text{trap}}=\frac{\Delta x}{2}\left(f(a)+f(a+\Delta x)\right).
\end{equation*}

If we have an interval $[a,b]$ that we split into $n$ sub intervals of width $\Delta x=(b-a)/n$ as before, then we use $n$-trapezia, one for each sub intervals. The side lengths of the trapezia are:
\begin{itemize}
\item $f_{0}=f(a)$ for the left most trapezium,
\item $f_{1}=f(a+\Delta x)$ for the right side of the first sub interval and the left side of the second,
\item $f_{2}=f(a+2\Delta x)$ for the right side of the second sub interval and the left side of the third, and so on up to
\item $f_{n}=f(a+n\Delta x)=f(b)$ for the right side of the $n$'th sub interval.
\end{itemize}

The area of the trapezia are thus
\begin{align*}
A_{1}&=\frac{\Delta x}{2}\left(f_{0}+f_{1}\right),\\
A_{2}&=\frac{\Delta x}{2}\left(f_{1}+f_{2}\right),\\
\dots &\dots\\
A_{n}&=\frac{\Delta x}{2}\left(f_{n-1}+f_{n}\right).
\end{align*}

Summing these up gives
\begin{equation}
\begin{split}
A_{T} 	&=A_{1}+A_{2}+\dots +A_{n}\\
	&=\frac{\Delta x}{2}\left(f_{0}+f_{1}\right)+\frac{\Delta x}{2}\left(f_{1}+f_{2}\right)+\dots +\frac{\Delta x}{2}\left(f_{n-1}+f_{n}\right)\\
	&=\frac{\Delta x}{2}\left(f_{0}+2\left(f_{1}+f_{2}+\dots +f_{n-1}\right)+f_{n}\right).
\end{split}
\label{eq: trapezium rule}
\end{equation}
Thus to estimate the integral using the trapezium rule we just need to know the value of the function at the end points of the intervals, and the width of the intervals.

\begin{ex}
Consider $f(x)=\cos(x)$ over the interval $[0,\uppi/2]$. In  \cite{lissamen2004mei}, they discuss this example for $1,2,4$ and $8$ sub intervals, here we will just treat the case of $n=4$, so $\Delta x= \uppi/8$. The easiest way to keep track of the information we need is in a table of points, the boundaries of the interval, and the value of the function at these points. This table is shown in \cref{table: trapezium data}. taking these five values of $f(x)$ and substituting into \cref{eq: trapezium rule} we get

\begin{table}[ht]
\centering
\caption{A table of data that you need to implement the trapezium rule.}

\vspace{2mm}

\label{table: trapezium data}



\begin{tabular}{|c|c|c|c|c|c|} 
 \hline
$x$& $0$ &$\frac{\uppi}{8}$ & $\frac{\uppi}{4}$& $\frac{3\uppi}{8}$&$\frac{\uppi}{2}$\\
 \hline
$f(x)$ & $1$ & $0.92$ & $\frac{1}{\sqrt{2}}$&$0.38$&$0$\\
 \hline
\end{tabular}
\end{table}

\begin{align*}
A_{T} &=\frac{\Delta x}{2}\left(f_{0}+2\left(f_{1}+f_{2}+f_{3}\right)+f_{4}\right)\\
	&=\frac{\uppi}{16}\left(1+2(0.92+\frac{1}{\sqrt{2}}+0.38)+0\right)\\
	&=0.98,
\end{align*}
which is close to the true value of $1$.
\end{ex}

Note that both the trapezium rule and the midpoint rule can be used to find the area under a curve even if we do not know the form of the function.

\begin{ex}

Consider that we are  given the data in \cref{table: trapezium data 2}, using just this information we can find 
\begin{equation*}
\int_{1}^{3}f(x)\ud x,
\end{equation*}
without knowing the form of $f(x)$ using the trapezium rule. 

\begin{table}[ht]
\centering
\caption{A table of data that you need to implement the trapezium rule.}

\vspace{2mm}

\label{table: trapezium data 2}



\begin{tabular}{|c|c|c|c|c|c|} 
 \hline
$x$& $1$ &$1.5$ & $2$& $2.5$&$3$\\
 \hline
$f(x)$ & $0$ & $0.41$ & $0.69$&$0.92$&$1.10$\\
 \hline
\end{tabular}
\end{table}

This works as follows. We have $n=4$ sub intervals defined by the five values of $x$ we are given. Thus we can use \cref{eq: trapezium rule}, and $\Delta x=(3-1)/2=1$ to find
\begin{align*}
A_{T} &=\frac{\Delta x}{2}\left(f_{0}+2\left(f_{1}+f_{2}+f_{3}\right)+f_{4}\right)\\
	&=\frac{1}{2}\left(0+2\left(0.41+0.69+0.92\right)+1.10\right)\\
	&=2.57.
\end{align*}

Note that we could use Newton's divided difference method to find an approximation to this function, but we do not need to do this to calculate the integral.
\end{ex}

\begin{exercise}
Use the trapezium rule with 6 sub intervals to calculate
\begin{equation*}
I= \int_{-\frac{\uppi}{2}}^{\frac{\uppi}{2}}\sin(x)\ud x.
\end{equation*}
You can do this either by hand, or set up a spread sheet or computer programme to solve this for you.
\end{exercise}

\begin{exercise}
Use the trapezium rule with 4 steps to approximate the definite integral
\begin{equation*}
I=\int_{0}^{2}\left(x^{2}+1\right)\ud x
\end{equation*}
and compare it to the answer that we got using the mid point rule in \cref{sec:integration}.
\end{exercise}

\begin{exercise}
Write a computer programme to implement the trapezium rule and use it to check the examples and exercises in this section.
\end{exercise}




\subsection*{Simpson's rule}

The final numerical method that we will discuss is Simpson's rule. This is a weighted average of the trapezium rule and the mid point rule.  We could actually consider any weighted average of the the two methods to get an approximation to the definite integral. However, if we were analysing the error we would see that there is one particular choice which minimises the error and gets as close as possible to the true value of the integral. This is Simpson's rule,

\begin{equation}
A_{S}=\frac{2A_{m}+A_{T}}{3}.
\label{eq: simpsons rule}
\end{equation}

Note that here we would use the same number of intervals for both the midpoint rule and the trapezium rule.

\begin{ex}
Let's return to one of our favourite examples
\begin{equation*}
I=\int_{0}^{\frac{\uppi}{2}}\cos(x)\ud x
\end{equation*}
and calculate this for $n=4$ sub intervals using Simpson's rule.\\

The data to calculate this is given in \cref{table: trapezium data}, though we need to find the value of the function at the midpoints for the midpoint rule. We already know that $A_{T}=0.98$ so we just need to find $A_{m}$. The four mid points and the value of the function at them are:

\begin{itemize}
\item $c_{0}=(\uppi/8 +0)/2=\uppi/16$ and $\cos(c_{0})=0.98$
\item  $c_{1}=(\uppi/4 +\uppi/8)/2=3\uppi/16$ and $\cos(c_{1})=0.83$,\\
\item $c_{2}=(3\uppi/8 +\uppi/4)/2=5\uppi/16$ and $\cos(c_{2})=0.556$ \\
\item  $c_{3}=(\uppi/2 +3\uppi/8)/2=7\uppi/16$ and $\cos(c_{3})=0.195$
\end{itemize}

We thus have that 
\begin{align*}
A_{m}&=\Delta x\left(f(c_{0})+f(c_{1})+f(c_{2})+f(c_{3})\right)\\
	&=\frac{\uppi}{8}\left(0.98+0.83+0.556+0.195\right)\\
	&=1.0057.
\end{align*}

Simpson's rule then gives that 
\begin{equation*}
A_{S}=\frac{2A_{m}+A_{T}}{3}=0.997\simeq 1
\end{equation*}
where we have rounded to two decimal places.
\end{ex}

\begin{exercise}
Go through the case from the example above, 
\begin{equation*}
I=\int_{0}^{\frac{\uppi}{2}}\cos(x)\ud x,
\end{equation*}
and get the answer to 9 decimal places.
\end{exercise}

\begin{exercise}
Use Simpson's rule with $n=6$ sub intervals to calculate
\begin{equation*}
I=\int_{0}^{2}\sqrt{1+\sin(x)+\cos(x)}\ud x,
\end{equation*}
Give your answer to 6 decimal places.
\end{exercise}


\section{Numerical differentiation}


\section{Numerical approaches to differential equations}
This is a non-examinable section and is not taught every year. It is included in the indicative content of the module but is not assessed in the learning outcomes. Content will be added here at the same time as content is added to \cref{sec:diff eqs} since that is the theoretical background for what will be discussed numerically here.
\newpage

%%%%%%%%%%%%%%%%%%%%%%%%%%%%%%%%%%%%%%%%%%%%%%

\chapter{Calculus in Computer Science}
\label{sec:CS calc}
In most of this module we have focussed on the mathematics, and even when trying to motivate the connection to computer science we have been fairly sketchy. In this chapter we are going to rectify that and discuss some concrete examples of where the material that you have studied here is of relevance to Computer Science, AI, and Machine Learning.\\

If you try to read this chapter before covering all of the basics you may find the examples hard to follow, but if you have finished everything up to \cref{sec:integration}, then you should be fine.\\

This chapter is split into three sections, one for each degree scheme, \textbf{Computer Science, Computer Science and AI, and Robotics and AI}. This does not mean that you will not find examples of interest in other sections, it just means that I am trying to group the examples thematically.  As with several other chapters, this is a work in progress and will be adapted and expanded over time. Be sure to check back frequently if you want to see the most up to date examples.

\section{Computer Science}

\section{Computer Science and AI}
A key concept in machine learning is the \textbf{loss function}, sometimes called the cost or error function, which calculates the difference between the output and the true or expected value. Loss functions appear generally in decision theory and the study of optimisation problems. In Computer science they appear whenever you are training an AI or Large Language model, where they measure the deviation of the model's output against the true value for the training data. \\

It could be that you are checking if your model can perform a mathematical calculation, in which case you will be comparing two numbers, the output number against the true solution. It could also be a more general problem where you are training a model to recognise images, in which case you will be comparing the answer output by your model to the true answer. The main idea is that this can always be described by some function. In fact a key step will be deciding what is an appropriate loss function for your problem. \\

Once we have the loss function, training is then the process of tweaking our model to minimise the loss function, so that the output of your model most closely matches the true values.


\begin{ex}
Consider a quadratic loss function 
\begin{equation*}
\lambda(x)=C\left(X-x\right)^{2}
\label{eq: square loss function}
\end{equation*}
\end{ex}

\section{Robotics and AI}
\newpage

%%%%%%%%%%%%%%%%%%%%%%%%%%%%%%%%%%%%%%%%%%%%%%

\chapter{Advanced Topics}
\label{sec:advanced topics}
This section is non examinable and is included so that you have a feel for how the material in this module can be taken further.

\section{L'H\^{o}pital's rule for evaluating limits}
\label{sec:l'hopital}
We saw in \cref{sec:functions} that some limits lead to nonsense expressions. For example, If we were confronted with $\lim_{x\to 0}\sin(x)/x$, this looks like it will result in $0/0$ which is does not make sense. Remember that not all limits that look indeterminate really are. For example
\begin{equation*}
\lim_{x\to 4}\frac{x^{2}-16}{x-4}
\end{equation*}
looks at first glance like it will have the form $0/0$ since both the numerator and denominator vanish for $x=4$. However, the numerator can be factored as $x^{2}-16=(x-4)(x+4)$ so the limit simplifies to
\begin{equation*}
\lim_{x\to 4}\frac{x^{2}-16}{x-4}=\lim_{x\to 4}\frac{(x-4)(x+4)}{x-4}=\lim_{x\to4}(x+4)=8.
\end{equation*}
This is why we said that you should try to expand and simplify the function that you are taking the limit of as much as possible. \\

L'H\^{o}pital\footnote{Originally spelt L'Hospital, with a silent s and no circumflex}'s rule enable us to make sense of some of those limits which still look indeterminate after they have been simplified. To apply L'H\^{o}pital's rule, we have to be taking the limit of a ratio of functions $f(x),g(x)$, where both are differentiable, the derivative of $g(x)$ does not vanish, and in the limit $f(x)$ and $g(x)$ wither both go to zero or both go to infinity. Then we have that
\begin{equation}
\lim_{x\to a}\frac{f(x)}{g(x)}=\lim_{x\to a}\frac{f'(x)}{g'(x)},
\label{eq: l'hopital's rule}
\end{equation}
so we can replace the ratio of the functions with the ratio of their derivatives. If both of the derivatives still vanish, or diverge, in the limit, then process can be repeated and we end up with more derivatives
\begin{equation*}
\lim_{x\to a}\frac{f(x)}{g(x)}=\lim_{x\to a}\frac{f^{(n)}(x)}{g^{(n)}(x)},
\end{equation*}
where the superscript $(n)$ means that we are differentiating the functions $n$ times.\\

Note that if the limit does not lead to an indeterminate then L'H\^{o}pital's rule does not hold!

\begin{ex}
As a counter example consider the limit
\begin{equation*}
\lim_{x\to 1}\frac{f(x)}{g(x)}=\lim_{x\to 1}\frac{x+1}{2x+1}.
\end{equation*}
Both $\lim_{x\to 1}f(x)=2$ and $\lim_{x\to1}g(x)=3$ are finite and no-zero. Evaluating the limit directly gives
\begin{equation*}
\lim_{x\to 1}\frac{x+1}{2x+1}=\frac{2}{3}.
\end{equation*}
We can also calculate the limit of the ratio of derivatives,
\begin{equation*}
\lim_{x\to 1}\frac{f'(x)}{g'(x)}\lim_{x\to 1}\frac{1}{2}=\frac{1}{2}\neq \frac{2}{3}.
\end{equation*}
So the limit of the ratio does not match the limit of the ratio of derivatives.
\end{ex}


As long as our original limit looks like it is indeterminate we can use L'H\^{o}pital's rule.

\begin{ex}
The limit of $\sin(x)/x$ as $x\to 0$ is evaluated as follows
\begin{equation*}
\lim_{x\to0}\frac{\sin(x)}{x}=\lim_{x\to0}\frac{\cos(x)}{1}=1.
\end{equation*}
\end{ex}

We can also evaluate other limits that may not at first look like a ratio of functions.
\begin{ex}
Consider the limit $\lim_{x\to-\infty}xe^{x}$ this looks like it becomes the indeterminate $(-\infty)(0)$. If we recall that $1/e^{x}=e^{-x}$ then we can rewrite the limit as
\begin{equation*}
\lim_{x\to-\infty}xe^{x}=\lim_{x\to-\infty}\frac{x}{e^{-x}},
\end{equation*}
which looks like the indeterminate $-\infty/\infty$. Now we can apply L'H\^{o}pital's rule to get
\begin{equation*}
\lim_{x\to-\infty}\frac{x}{e^{-x}}=\lim_{x\to-\infty}\frac{1}{-e^{-x}}=0,
\end{equation*}
since $1/\infty$ is zero.
\end{ex}

\section{Functions of two variables}

\section{Multiple integrals}

\section{Optimisation Problems}

\section{Polynomial approximation}
\newpage

%%%%%%%%%%%%%%%%%%%%%%%%%%%%%%%%%%%%%%%%%%%%%%


\chapter{Background Mathematics}
\label{sec:background}

\epigraph{Try as you may you just can't get away from Mathematics }{\textit{That's Mathematics by Tom Lehrer}}

\section{Background and References}
As this is a maths module there is a lot of assumed background.  This means that to understand the material in the module and to be able to solve the tutorial problems, you need to have a experience with a variety of mathematical techniques such as:
\begin{itemize}
%\setlength{\itemsep}{-5pt}
    \item solving linear equation,
    \item solving quadratic equations
    \item using trigonometry,
    \item knowning some simple functions.
\end{itemize}

These topics will be familiar to many of you. However, it may have been a while since some of you studied mathematics and I am aiming to briefly introduce any new mathematical topics when we need them. However, I also want to link to some extra resources where you can brush up on your maths beforehand.\\


A great resource is the website \href{https://tutorial.math.lamar.edu/}{Pauls Online Math Notes}. The website contains notes for a variety of mathematics courses including algebra and calculus. The most useful background materials are the Algebra and trigonometry review \href{https://tutorial.math.lamar.edu/Extras/AlgebraTrigReview/AlgebraTrigIntro.aspx}{linked here} and the preliminaries section of the algebra notes, \href{https://tutorial.math.lamar.edu/Classes/Alg/Preliminaries.aspx}{linked here}.\\

As the module goes on I may add more background resources or add some examples.\\

%The essential maths skills needed for studying physics at this level are nicely summarised in \citep{garrett2015essential}. It is worth having a look at this book if you want to revise the maths background.

\section{Trigonometry Primer}
Since trigonometric functions come up a lot in this course a review of the basics is included here. If you are not familiar with it you can either check out the links suggested above or ask me to provide more background information.\\

Here we will only discuss trigonometry for right angled triangles, but in the main content of the module we will deal with general trig functions. \textbf{Trigonometry} is an area of mathematics related to the study of triangles and provides a way to compute the lengths and angles in a triangle provided that you now some of them already.

\begin{figure}[ht]
    \centering
   % \pdftooltip{
   \begin{tikzpicture}[scale=2]
  \coordinate [label=left:$\uptheta$] (C) at (-1.5cm,-1.cm);
  \coordinate (A) at (1.5cm,-1.0cm);
  \coordinate [label=above:$\upphi$] (B) at (1.5cm,1.0cm);
  \draw (C) -- node[above] {$a$} (B) -- node[right] {$c$} (A) -- node[below] {$b$} (C);
  \draw (1.25cm,-1.0cm) rectangle (1.5cm,-0.75cm);
  \tkzMarkAngle[size=1cm,color=blue](A,C,B)
  \tkzMarkAngle[size=1cm,color=blue](C,B,A)
\end{tikzpicture}
%}{A right angle triangle with all the angles and sides marked.}
    \caption{A right angle triangle with all the angles and sides marked.}
    \label{fig: Trig definitions}
\end{figure}

The typical mnemonic used to remember trigonometry is \textbf{SOH CAH TOA} which means Sine is opposite over hypotenuse
\begin{equation*}
\sin\uptheta=\frac{c}{a}, \qquad \sin\upphi =\frac{b}{a},
\end{equation*}
Cosine is adjacent divided by hypotenuse
\begin{equation*}
\cos\uptheta=\frac{b}{a}, \qquad \cos\upphi =\frac{c}{a},
\end{equation*}
and Tangent is opposite divided by adjacent
\begin{equation*}
\tan\uptheta=\frac{c}{b}, \qquad \sin\upphi =\frac{b}{c}.
\end{equation*}
There are associated inverse functions $\arcsin,\arccos,\arctan$ which convert ratios of sides into angles. All of these functions are available on your calculator. \\

It is important to be careful with which units you are using to express angles. It is likely that you will have come across degrees where going around a full circle is represented by an angle of $360^{\circ}$. However, it is often convenient to work with a different measure of angles called radians, in this case we take a full circle to be $2\uppi\text{rad}$ and express angles as a number between $0$ and $2\uppi$. In this module you can probably get away with always working in degrees, other than for some of the formulas quoted for the pendulum. However, if you go further with maths or physics you will find that radians as much more convenient to work with.

\section{Rearranging Equations}
A very important skill for solving mathematics problems, and finding the roots of functions, is to be able to rearrange equations. This is sometimes referred to as changing the subject of an equation. Newcastle University have a webpage, available \href{https://www.mas.ncl.ac.uk/ask/numeracy-maths-statistics/core-mathematics/pure-maths/algebra/rearranging-equations.html}{here} that goes through some examples of how to rearrange equations. The webpage also has some self test questions that you can look at if you want more practice. Some of the details are reviewed here, along with examples for the specific equations that we have been using in this module.\\


In the equation
\begin{equation*}
x=5y+4z,
\end{equation*}
$x$ is called the subject, which is just a fancy way of saying that $x$ is expressed in terms of the other variables. When we talk about rearranging an equation, we mean that we change the subject of the equation from $x$ to another variable like $y$ and $z$. We do this by performing a variety of mathematical operations to both sides of the equation to swap some of the variables from one side to the other. \\

These operations can include: adding or subtracting a quantity from both sides, multiplying or dividing by a quantity, taking logarithms of or exponentiating both sides of the equation, raising both sides of the equation to any non-zero power.\\
\begin{ex}
Returning to the above equation $x=5y+4x$, we can rearrange it to make $z$ the subject. Again, this means that we will perform mathematical operations on both sides of the equation to put it in the form $z=\dots{}$ : 
\begin{align*}
x&=5y+4z, \quad \text{ first subtract $5y$ from both sides},\\
x-5y&=5y+4z-5y=4z, \quad \text{then divide both sides by 4},\\
\frac{x-5y}{4}&=\frac{4z}{4}=z.
\end{align*}
This gives us
\begin{equation*}
z=\frac{x-5y}{4},
\end{equation*}
with $z$ now the subject of the equation.
\end{ex}

\begin{ex}
As another example consider an equation from physics $v=u+at$, which says that the if an object starts off with a velocity $v$ and accelerates at a rate $a$ then after time $t$ its velocity will be $v$. Here we go through this step by step what we do when solving for $a$:
\begin{align*}
v&=u+at, \quad \text{subtract $u$ from both sides},\\
v-u&=u+at -u=at, \quad \text{divide both sides by $t$},\\
\frac{v-u}{t}&=\frac{at}{t}=a,
\end{align*}
which gives us
\begin{equation*}
a=\frac{v-u}{t}.
\end{equation*}
\end{ex}

\begin{ex}
Another example is rearranging a different equation from physics,
\begin{equation*}
s=ut+\frac{1}{2}at^{2},
\end{equation*}
to solve for either $a$ or $t$.\\

To make $a$ the subject we proceed as follows:
\begin{align*}
s&=ut+\frac{1}{2}at^{2}, \quad \text{subtract $ut$ from both sides},\\
s-ut&=ut+\frac{1}{2}at^{2}-ut=\frac{1}{2}at^{2}, \quad \text{multiply both sides by $2$},\\
2\left(s-ut\right)&=2\left(\frac{1}{2}at^{2}\right)=at^{2}, \quad \text{divide both sides by $t^{2}$},\\
\frac{2\left(s-ut\right)}{t^{2}}&=\frac{at^{2}}{t^{2}}=a,
\end{align*}
thus the equation with $a$ being the subject is 
\begin{equation*}
a=\frac{2\left(s-ut\right)}{t^{2}}.
\end{equation*}

If instead we wanted $t$ to be the subject then it is easier to put it in the form of a quadratic equation and then use the quadratic formula:
\begin{align*}
s&=ut+\frac{1}{2}at^{2}, \quad \text{subtract $s$ from both sides},\\
0&=\frac{1}{2}at^{2}+ut-s,\quad \text{this is a qudratic equation for $t$ and is solved by}\\
t&=\frac{-u\pm\sqrt{u^{2}+2as}}{a}.
\end{align*}
\end{ex}

There are a few special cases where we do not need to use the quadratic formula that it is worth being aware of. If $s=0$ then we have
\begin{align*}
0&=ut+\frac{1}{2}at^{2}, \quad \text{factor out the common factor},\\
0&=t\left(u+\frac{1}{2}at\right),
\end{align*}
This has two solutions $t=0$, which is the start of the motion, and
\begin{align*}
0&=u+\frac{1}{2}at, \quad \text{subtract $u$ from both sides},\\
-u&=\frac{1}{2}at, \quad \text{multiply both sides by $2$},\\
-2u&=at, \quad \text{divide both sides by $a$},\\
-\frac{2u}{a}&=t.
\end{align*}
This case often appear when considering projectile motion and you want to calculate the total length of time that the projectile is in the air for.\\

The other common example is if $u=0$, then:
\begin{align*}
s&=\frac{1}{2}at^{2}, \quad \text{multiply both sides by $2$},\\
2s&=at^{2}, \quad \text{divide both sides by $a$},\\
\frac{2s}{a}&=t^{2}, \quad \text{take the square root of both sides},\\
\sqrt{\frac{2s}{a}}&=t.
\end{align*}
In the last line we have dropped the $\pm$ that should be in front of the square root since we do not consider negative time. However, if you were just solving a quadratic equation then you would have to remember to include that.  Both of these special cases can also be found by direct substitution into the quadratic formula of $s=0$ or $u=0$ respectively.\\

\begin{ex}
The final example is to rearrange an equation involving a square root,
\begin{equation*}
T=2\uppi\sqrt{\frac{l}{g}}.
\end{equation*}
If you are asked to make $l$ the subject of this equation we proceed as follows:
\begin{align*}
T&=2\uppi\sqrt{\frac{l}{g}}, \quad \text{first divide both sides by $2\uppi$},\\
\frac{T}{2\uppi}&=\frac{2\uppi}{2\uppi}\sqrt{\frac{l}{g}}=\sqrt{\frac{l}{g}}, \quad \text{then square both sides of the equation},\\
\left(\frac{T}{2\uppi}\right)^{2}&=\left(\sqrt{\frac{l}{g}}\right)^{2}=\frac{l}{g}, \quad \text{now multiply both sides by $g$},\\
g\left(\frac{T}{2\uppi}\right)^{2}&=g\times\frac{l}{g}=l.
\end{align*}
This leaves us with
\begin{equation*}
l=g\left(\frac{T}{2\uppi}\right)^{2}=\frac{gT^{2}}{4\uppi^{2}}.
\end{equation*}
This is the equation for a straight line $y=mx+c$ where $l$ plays the role of $y$, the $y$-intercept $c=0$, $\left(\frac{T}{2\uppi}\right)^{2}$ plays the role of $x$, and $m=g$ is the gradient of the straight line. When analysing your data you would use plot your data and should observe a straight line whose gradient is $g$.\\

At the last step of this rearrangement we could instead make $g$ the subject. To do this we proceed as follows:
\begin{align*}
\left(\frac{T}{2\uppi}\right)^{2}&=\left(\sqrt{\frac{l}{g}}\right)^{2}=\frac{l}{g}, \quad \text{now multiply both sides by $g$},\\
g\left(\frac{T}{2\uppi}\right)^{2}&=g\times \frac{l}{g}=l, \quad \text{divide both sides by }\left(\frac{T}{2\uppi}\right)^{2}, \\
g&=\frac{l}{\left(\frac{T}{2\uppi}\right)^{2}}=\frac{4\uppi^{2} l}{T^{2}}.
\end{align*}
\end{ex}

If you want more examples there is a \textbf{Transposition of Formulae} workbook, designed by mathcentre,  available \href{https://www.mathcentre.ac.uk/resources/uploaded/mc-ty-transposition-2009-1.pdf}{here}.


\newpage

%%%%%%%%%%%%%%%%%%%%%%%%%%%%%%%%%%%%%%%%%%%%%%

\chapter{Further Reading}
\label{sec:further reading}

\epigraph{Mathematics, you see, is not a spectator sport. To understand mathematics means to be able to do mathematics. And what does it mean doing mathematics? In the first place, it means to be able to solve mathematical problems. }{\textit{How to Solve It by George P\'{o}lya}}

What we discuss in the lectures is just a guide to calculus its many uses. Depending on the modules that you take later on in your degree you will meet Calculus in different guises, particularly when you come across optimisation problems . In this section I will provide links to further resources and suggestions for further reading based on the material that we met each week. If you have any questions about the topics linked here or you want to get even more information then drop me an \href{mailto:rossc@edgehill.ac.uk}{email}.\\

It is important to remember that the lectures are their to introduce you to topics and signpost where to find out more information. If you are just attending the lectures and not doing any further reading or solving practice problems then you will struggle to pass the course. Each 20 credit module is considered 200 hours of work, only 36 hours of which are the lectures and seminars. You are expected to put in around 164 hours of self study during a 12 week module. The resources linked here will help with this self study.\\

\section{Why Calculus Extra Reading}
MIT have a course called \textbf{Calculus for Beginners and Artists} which contains some overlap with this module.  The webpage for the course is available \href{https://math.mit.edu/~djk/calculus_beginners/index.html}{here} and some sections of it, particularly \textbf{Chapter 0: Why Study Calculus?} are worth a read to complement what we will do in this module. I will occasionally suggest reading sections of these notes or sections of \citep{calcI} for a complementary explanation. 



\section{Functions Extra Reading}


\section{Differentiation Extra Reading}
The main reference for this section and many of the other sections of these lecture notes is the wonderful book \textbf{Mathematical Methods for Physics and Engineering}, \citep{riley_mathematical_2006}. There are several copies of this book and its student solution manual in the library and I recommend that you try and have a look at this at some stage.\\

\section{Integration Extra Reading}

\section{Differential Equations Extra Reading}

\section{Numerical Methods Extra Reading}

\section{Advanced Topics Extra Reading}

\newpage

%%%%%%%%%%%%%%%%%%%%%%%%%%%%%%%%%%%%%%%%%%%%%%

\chapter{Tutorial Sheets}
\label{sec: tutorial sheets}

Here we collect all of the tutorial problems for the module. They are split into different weeks depending on the topic they relate to and when they were given out.\\

Many of these questions are taken from or adapted from the recommended  books for the module or from some of the linked resources. These problems are to be attempted in the tutorial sessions and are there to help you familiarise yourself with the material that we have covered in the lectures.\\

Problems marked with a star, $(\star)$ are particularly worth attempting. Problems marked with a dagger, $(\dagger)$, are more challenging and often go beyond what we directly discussed in the lectures.\\

The challenge problem sections contain extra problems. Some of them are just there for extra practice, but others are significantly more difficult than what you need to be able to solve to pass the module. If you are finding the content too easy then have a go at the challenge problems. Sometime the challenge problems from one week will be quite similar to the ordinary problems of the next week. as the problems will become more accessible the more material that we cover.

\section{Week 1}
\label{sec: Tutorial sheet 1}
\paragraph{Functions}

\begin{problem}[$\star$]
Find the roots of the polynomial
\begin{equation*}
g(x)=x^2-2x-12
\end{equation*}
and plot the function.
\end{problem}

\begin{problem}
Find the roots of the polynomial
\begin{equation*}
g(x)=x^3+x^2-x-1
\end{equation*}
and plot the function.
\end{problem}

\begin{problem}[$\star$]
Consider the function
\begin{equation*}
g(x)=x^2+2x+2,
\end{equation*}
produce a plot of the function by calculating its value at a selection of points. What do you notice as $x$ gets very large? What happens for $x=0$?
\end{problem}

\begin{problem}[$\star$]
Consider the function
\begin{equation*}
g(x)=\frac{1}{x-4},
\end{equation*}
and plot the function.what happens as $x$ gets large? What happens as $x$ approaches $4$?\\

Plot the function and comment on its behaviour.
\end{problem}

\begin{problem}
Draw a schematic of a one-to-one function between the sets
\begin{equation*}
\begin{split}
&\left\{1,2,3,4,5\right\}\\
&\left\{a,b,c,d,e,f,g\right\}.
\end{split}
\end{equation*}
Is there only one way to do this?
\end{problem}




\begin{problem}[$\star$]
Given the function
\begin{equation*}
f(x)=x^3-2x^2-x+2
\end{equation*}
find:
\begin{itemize}
    \item $f(0)$,
    \item $f(1)$,
    \item $f(-1)$,
    \item $f(2)$,
    \item $f(-2)$,
    \item $f(t)$,
    \item $f(x-1)$.
\end{itemize}
\end{problem}

\begin{problem}
Consider the function
\begin{equation*}
h(x)=\frac{x}{\sqrt{x^{2}-9}}.
\end{equation*}
Find the points where the denominator vanishes, then plot the function avoiding these points. What happens to the plot as the function approaches these points?
\end{problem}


\begin{problem}
Given the functions
\begin{align*}
f(x)&=x^2-x+1,\\
g(x)&=2-x,
\end{align*}
find:
\begin{itemize}
\item $\left(f\circ g\right)(2)$,
\item $\left(g\circ f\right)(2)$,
\item $\left(f\circ g\right)(x)$,
\item $\left(g\circ f\right)(x)$.
\end{itemize}
\end{problem}


\begin{problem}[$\star$]
Given the functions
\begin{align*}
f(x)&=3x-2\\
g(x)&=\frac{x}{3}+\frac{2}{3},
\end{align*}
find:
\begin{itemize}
\item $\left(f\circ g\right)(x)$,
\item $\left(g\circ f\right)(x)$,
\item What is the relationship between $f(x)$ and $g(x)$?
\end{itemize}
\end{problem}


\begin{problem}[$\dagger$]
Given the function
\begin{equation*}
h(x)=\frac{x+4}{2x-5},
\end{equation*}
identify when it has an inverse and calculate the inverse.
\end{problem}

\paragraph{Challenge Problems}

\begin{problem}
Consider the function 
\begin{equation*}
f(x)=\frac{x^{2}-x-12}{x-1}.
\end{equation*}
Identify the points where the numerator and denominator vanish. Plot the function and explain what happens to the function near these points.
\end{problem}

\begin{problem}
Given the function 
\begin{equation*}
f(x)=2x-3,
\end{equation*}
find the inverse function $f^{-1}(x)$.
\end{problem}


\begin{problem}[$\dagger\dagger$]
Build a schematic of a bijection between the natural numbers
\begin{equation*}
\N=\{1,2,3,4,5,\dots\},
\end{equation*}
and the integers
\begin{equation*}
\Z=\{0,1,-1,2,-2,3,-3,\dots\}.
\end{equation*}

Are there more integers than natural numbers?\\

Could you do the same for the real numbers $\R$? If you find this interesting you may want to explore the work of Cantor.
\end{problem}




\section{Week 2}
\label{sec: Tutorial sheet 2}

\paragraph{Polynomials}
\begin{problem}[$\star$]
For the polynomial equation
\begin{equation*}
x^{3}-3x+2=0,
\end{equation*}
express it as a product of its factors.\\

Hint: this means write it as 
\begin{equation*}
\left(x-a\right)\left(x-b\right)\left(x-c\right),
\end{equation*}
where $a,b,c$ are the roots of the polynomial.
\end{problem}

\begin{problem}
Find the roots of the following polynomial equation
\begin{equation*}
x^{4}-4x^{3}+6x^{2}-4x+1=0.
\end{equation*}
\end{problem}

\paragraph{Trig functions}
Remember to work in radians for any problems related to trigonometry.

\begin{problem}
Find the solutions to 
\begin{equation*}
\sqrt{2}\cos x =1.
\end{equation*}
\end{problem}

\begin{problem}
Identify any solutions to the equation
\begin{equation*}
\sin(2x)=-2.
\end{equation*}
\end{problem}

\begin{problem}[$\star$]
Solve
\begin{equation*}
2x\sin x = x.
\end{equation*}
\end{problem}

\paragraph{Exponentials and Logarithms}
\begin{problem}[$\star$]
Solve the equation
\begin{equation*}
x=xe^{4x}.
\end{equation*}
\end{problem}

\begin{problem}[$\star$]
Solve
\begin{equation*}
2\ln x -\ln(x+2)=1.
\end{equation*}
\end{problem}

\paragraph{Limits and Asymptotes}
\begin{problem}[$\star$]
For the function
\begin{equation*}
f(x)=\frac{x^{2}+4x+12}{x^{2}-2x}
\end{equation*}
evaluate the limit
\begin{equation*}
\lim_{x\to 2}f(x).
\end{equation*}
\end{problem}

\begin{problem}[$\star$]
For the function
\begin{equation*}
g(x)=\frac{2x^{4}-x^{2}+8x}{7-5x^{4}}
\end{equation*}
evaluate the limits
\begin{align*}
&\lim_{x\to -\infty}g(x),\\
&\lim_{x\to \infty}g(x).
\end{align*}
\end{problem}

\begin{problem}[$\dagger$]
For the function
\begin{equation*}
h(x)=\frac{6e^{4x}-e^{-2x}}{8e^{4x}-2e^{2x}+3e^{-x}}
\end{equation*}
evaluate the limit
\begin{align*}
&\lim_{x\to \infty}h(x).
\end{align*}
\end{problem}

\begin{problem}
Using an appropriate plot evaluate the limits
\begin{align*}
&\lim_{x\to \frac{\pi}{2}}\tan x,\\
&\lim_{x\to 0}\tan x.
\end{align*}
\end{problem}

\begin{problem}[$\star$]
Determine where the function
\begin{equation*}
g(x)=\frac{x^{3}+x^{2}-x-1}{x^{3}-2x^{2}-4x+8}
\end{equation*}
fails to be continuous. What do these points correspond to on a plot of $g(x)$?
\end{problem}

\paragraph{Challenge Problems}
\begin{problem}
By making a table of values of $f(x), x$ estimate the limit
\begin{equation*}
\lim_{x\to 2}f(x)
\end{equation*}
from above and below for the function
\begin{equation*}
f(x)=\frac{x^{2}+4x-12}{x^{2}-2x}.
\end{equation*}
\end{problem}

\begin{problem}
Use the definition of continuity to determine in the function
\begin{equation*}
g(x)=\frac{4x+10}{x^{2}-2x-15}
\end{equation*}
is continuous and if not find the points where it has discontinuities.
\end{problem}


\begin{problem}[$\dagger$]
For the function $g(x)$ in the previous problem look at the definition of differentiability and work out where $g(x)$ fails to be differentiable.
\end{problem}

\paragraph{Hyperbolic Trig functions}
\begin{problem}[$\dagger$]
Show that
\begin{equation*}
\sinh(x+y)=\sinh x \cosh y+\cosh x \sinh y.
\end{equation*}
\end{problem}

\begin{problem}[$\dagger$]
Following some of the examples in \cref{sec: hyperbolic functions} show that 
\begin{equation*}
\cosh^{-1}x=\ln\left(x\pm\sqrt{x^{2}+1}\right).
\end{equation*}
\end{problem}

\section{Week 3}
\label{sec: Tutorial sheet 3}
\paragraph{Differentiation}

\begin{problem}[$\star$]
Find the derivative of 
\begin{equation*}
f(x)=7x^{2}
\end{equation*}
\begin{itemize}
\item[a)] using first principles,
\item[b)] using the rule for differentiating monomials.
\end{itemize}
\end{problem}

\begin{problem}
Find the derivative of 
\begin{equation*}
f(x)=3x^{3}-8x.
\end{equation*}
\end{problem}

\begin{problem}[$\star$]
Find the derivative of 
\begin{equation*}
f(x)=(x+4)(x+2).
\end{equation*}
\end{problem}

\begin{problem}[$\star$]
Find the derivative of 
\begin{equation*}
g(x)=6x^{4}-x^{3}.
\end{equation*}
\end{problem}

\begin{problem}
Find the derivative of 
\begin{equation*}
y=\frac{x^{9}}{3}+\frac{x^{4}}{4}.
\end{equation*}
\end{problem}

\begin{problem}[$\star$]
Calculate the derivative of 
\begin{equation*}
y=\left(x^{4}-3\right)^{2}.
\end{equation*}
\end{problem}

\begin{problem}[$\dagger$]
Consider the function
\begin{equation*}
h(x)=\frac{x}{\sqrt{x^{2}-9}},
\end{equation*}
when can we calculate its derivative? When it exists calculate the derivative.
\end{problem}

\begin{problem}
Identify when we can differentiate the function
\begin{equation*}
y=\frac{3x^{3}\left(x^{2}-4x\right)}{x}
\end{equation*}
and calculate its derivative.
\end{problem}

\begin{problem}[$\star$]
Identify when we can differentiate the function
\begin{equation*}
f(x)=9x^{4}-\frac{4}{x^{3}}
\end{equation*}
and calculate its derivative.
\end{problem}
\begin{problem}[$\star$]
Differentiate the function
\begin{equation*}
f(x)=\sin(2x).
\end{equation*}
\end{problem}

\begin{problem}
Differentiate the function
\begin{equation*}
f(x)=\cos(3x).
\end{equation*}
\end{problem}

\begin{problem}[$\star$]
Differentiate the following expressions
\begin{itemize}
    	\item[a)] $\cos(x)-\sin(x)$
	\item[b)] $3\tan(x)-\cos(2x)$
	\item[c)] $4\sin(2x)+2\cos(5x)+5$
	\item[d)] $\ln(x+3)$
	\item[e)] $x\ln(x^{2})$
\end{itemize}
\end{problem}

\paragraph{Challenge Problems}

\begin{problem}
Find the gradient of the following curves at the indicated points
\begin{itemize}
    	\item[a)] $y=5\sin(2x)$ at the point $(x,y)=(\up/2,0)$
	\item[b)]$y=\tan(x/2)$ at the point $(x,y)=(\up/2,1)$
\end{itemize}
\end{problem}

\begin{problem}
Find the stationary points, the points where the derivative vanishes, for the following functions
\begin{itemize}
    	\item[a)] $y=\sin(2x)+\cos(x)$, with $0\leq x\leq \up$
	\item[b)]$f(x)=\log(\sin(x))$ for $0\leq x\leq \up$
\end{itemize}
\end{problem}

\begin{problem}[$\dagger$]
Calculate the derivative of 
\begin{equation*}
y=x^{x}.
\end{equation*}
Hint: You may want to take the logarithm of the function.
\end{problem}

\section{Week 4}
\label{sec: Tutorial sheet 4}

\paragraph{Trig, Exp, and Log}

\begin{problem}[$\star$]
Solve the following, leaving your answer in terms of logs:
\begin{itemize}
    	\item[a)] $2^{x+4}=6$
	\item[b)] $3^{2x-1}=17$
	\item[c)] $2^{1-4x}=5$
	\item[d)] $5^{3x+4}=31$
\end{itemize}
\end{problem}


\begin{problem}
Solve $0.6=2^{-x}$.
\end{problem}

\begin{problem}
The point $(K,5)$ lies on the curve $y=2^{x}$. Find $K$.
\end{problem}

\begin{problem}[$\star$]
Simplify the expressions:
\begin{itemize}
    	\item[a)] $\ln e^{\sin(x)}$
	\item[b)] $e^{2\ln(1+x)}$
	\item[c)] $e^{-\ln(5-x)}$
\end{itemize}
\end{problem}

\begin{problem}
Solve $e^{2x}-5e^{x}+4=0$.
\end{problem}

\begin{problem}[$\star$]
Evaluate the following limits:
\begin{itemize}
    	\item[a)] $\lim_{x\to \up/2}\left(\frac{x}{1+\sin(x)}\right)$
	\item[b)] $\lim_{x\to 1}\left(\frac{\ln(x)}{1+\ln(x)}\right)$
	\item[c)] $\lim_{x\to \infty}\left(\frac{3+2x}{2+3x}\right)$
	\item[d)] $\lim_{x\to\up^{-}/2}\left(\frac{x}{\tan{x}}\right)$
	\item[e)] $\lim_{x\to\up^{+}/2}\left(\frac{x}{\tan{x}}\right)$
\end{itemize}
\end{problem}


\section{Week 5}
\label{sec: Tutorial sheet 5}

\paragraph{Product, Quotient, and Chain Rules}
\begin{problem}[$\star$]
Given the function 
\begin{equation*}
h(x)=\frac{x+4}{2x-5}
\end{equation*}
identify when you can differentiate it and find its derivative.
\end{problem}

\begin{problem}[$\star$]
Calculate the derivative of 
\begin{equation*}
y=x^{5}\sin(x)
\end{equation*}
using the product rule.
\end{problem}

\begin{problem}
Calculate the derivative of 
\begin{equation*}
f(x)=3e^{x}\cos(x)
\end{equation*}
using the product rule.
\end{problem}

\begin{problem}[$\star$]
Calculate the derivative of 
\begin{equation*}
f(x)=\left(10x-3\right)^{4}
\end{equation*}
using the chain rule.
\end{problem}

\begin{problem}
Use the quotient rule to calculate the derivative of 
\begin{equation*}
y=\tan(x)=\frac{\sin(x)}{\cos(x)}.
\end{equation*}
Does it match your expectation?
\end{problem}

\begin{problem}[$\star$]
Use the quotient rule to calculate the derivative of 
\begin{equation*}
f(x)=\frac{3x^{3}+8x^{2}+2}{2x+1}.
\end{equation*}
\end{problem}

\begin{problem}
Calculate the derivative of 
\begin{equation*}
f(x)=\ln(x^{2}+1).
\end{equation*}
\end{problem}

\paragraph{Antiderivatives}


\begin{problem}[$\star$]
By identifying a function whose derivative is $1/x$ solve the integral
\begin{equation*}
I=\int \frac{1}{x}\ud x.
\end{equation*}
\end{problem}

\begin{problem}[$\star$]
Find the antiderivative of $e^{x}$
\end{problem}

\begin{problem}[$\star$]
Find the antiderivatives of the following functions:
\begin{itemize}
    	\item[a)] $\cos(x)$
	\item[b)] $x^{2}+2x$
	\item[c)] $\sqrt{X^{2}+2}$
	\item[d)] $x^{2}+1/x^{2}$
\end{itemize}
\end{problem}

\paragraph{Integration and Area}

\begin{problem}
Calculate the integral
\begin{equation*}
I=\int \left(5x^{2}-8x+5\right)\ud x.
\end{equation*}
\end{problem}

\begin{problem}[$\star$]
Calculate the integrals:
\begin{itemize}
    	\item[a)] $I=\int\left(x^{\frac{3}{2}}+2x+3\right)\ud x$
	\item[b)] $I=\int\left(\frac{8}{x}-\frac{5}{x^{2}}+\frac{6}{x^{3}}\right)\ud x$
	\item[c)] $I=\int\left(4e^{-7x}\right)\ud x$
\end{itemize}
\end{problem}

\begin{problem}[$\star$]
Calculate the integrals:
\begin{itemize}
    	\item[a)] $I=\int\frac{x^{3}+4}{x^{2}}\ud x$
	\item[b)] $I=\int\left(12x^{\frac{3}{4}}-9x^{\frac{5}{3}}\right)\ud x$
	\item[c)] $I=\int7\sin(x)\ud x$
	\item[d)] $I=\int 5\cos(x)\ud x$
\end{itemize}
\end{problem}

\begin{problem}[$\star$]
Calculate the following definite integrals:
\begin{itemize}
    	\item[a)] $I=\int^{4}_{1} \left(5x^{2}-8x+5\right)\ud x$
	\item[b)]$I=\int^{9}_{1}\left(x^{\frac{3}{2}}+2x+3\right)\ud x$
	\item[c)] $I=\int^{\frac{\up}{2}}_{0}\left(\frac{8}{x}-\frac{5}{x^{2}}+\frac{6}{x^{3}}\right)\ud x$
	\item[d)] $I=\int_{\frac{\up}{2}}^{\frac{3\up}{2}}\left(4e^{-7x}\right)\ud x$
\end{itemize}
\end{problem}

\paragraph{Challenge Problems}

\begin{problem}
Evaluate the integral
\begin{equation*}
I=\int \frac{3x}{4x-5}\ud x.
\end{equation*}
\end{problem}

\begin{problem}
Use integration by parts to evaluate
\begin{equation*}
I=\int xe^{-2x}\ud x.
\end{equation*}
\end{problem}

\begin{problem}[$\dagger$]
Look at the definition of differentiation from first principles and prove the product rule from first principles.
\end{problem}

\begin{problem}[$\star$]
Consider the function
\begin{equation*}
f(x)=x^{2}+2x
\end{equation*}
calculate its derivative. Is the derivative that you find a differentiable function? If it is, calculate its derivative, what do you get?\\

Do the same for $f(x)=\sin(x)$, what do you notice here?
\end{problem}




\section{Week 8}
\label{sec: Tutorial sheet 8}

\paragraph{Numerical Integration}


\begin{problem}[$\star$]
Consider the integral
\begin{equation*}
I=\int_{0}^{2}3^{x}\ud x
\end{equation*}
Evaluate this using:
\begin{itemize}
    	\item[a)] The midpoint rule with $4$ strips.
	\item[b)]The Trapezium rule with $4$ strips.
	\item[c)] Simpson's rule with $4$ strips.
\end{itemize}
\end{problem}

\begin{problem}[$\star$]
Consider the integral
\begin{equation*}
I=\int_{-1}^{1}\sqrt{x^{3}+1}\ud x
\end{equation*}
Evaluate this for $3$ strips using:
\begin{itemize}
    	\item[a)] The midpoint rule.
	\item[b)]The Trapezium rule.
	\item[c)] Simpson's rule.
\end{itemize}
\end{problem}

\begin{problem}
Consider the integral
\begin{equation*}
I=\int_{0}^{1}\frac{1}{x^{2}+1}\ud x
\end{equation*}
Evaluate this for $9$ strips using:
\begin{itemize}
    	\item[a)] The midpoint rule.
	\item[b)]The Trapezium rule.
	\item[c)] Simpson's rule.
\end{itemize}
\end{problem}

\begin{problem}[$\star$]
Consider the integral
\begin{equation*}
I=\int_{0}^{1}\sqrt{x(2x-1)}\ud x
\end{equation*}
Evaluate this for $4$ strips using:
\begin{itemize}
    	\item[a)] The midpoint rule.
	\item[b)]The Trapezium rule.
	\item[c)] Simpson's rule.
\end{itemize}
\end{problem}

\begin{problem}
Consider the integral
\begin{equation*}
I=\int_{1}^{2}\ln\left(1+\sqrt{x}\right)\ud x
\end{equation*}
Evaluate this for $5$ strips using:
\begin{itemize}
    	\item[a)] The midpoint rule.
	\item[b)]The Trapezium rule.
	\item[c)] Simpson's rule.
\end{itemize}
\end{problem}

\paragraph{Challenge Problems}

\begin{problem}
Use Simpson's rule with $6$ strips to calculate
\begin{equation*}
I=\int_{0}^{2}\sqrt{1+\sin(x)+\cos(x)}\ud x
\end{equation*}
to $6$ decimal places.
\end{problem}

\begin{problem}
Write a computer program to implement Simpson's rule and use it to check the problems on this sheet.
\end{problem}

\section{Week 9}
\label{sec: Tutorial sheet 9}

\paragraph{Numerical Differentiation}

\begin{problem}
Given $f(x)=\cos(x)$,
\begin{itemize}
    	\item[a)] find $f'\left(\frac{\up}{3}\right)$ using the forward difference method with $h=0.1,0.01, 0.001, 0.0001$.
	\item[b)]Now find $f'\left(\frac{\up}{3}\right)$ using the backward difference method with $h=0.1,0.01, 0.001, 0.0001$.
	\item[c)] Calculate the exact value of the derivative at $x=\frac{\up}{3}$ and compare it to the approximations.
\end{itemize}
\end{problem}

\begin{problem}[$\star$]
Use the forward difference method with $h=0.05$ to approximate the derivative of $f(x)=4e^{2x}$ at $x=1$ and compare this to the exact result.
\end{problem}

\begin{problem}[$\star$]
Given $f(x)=\cos(x)$ find the value of the derivative at $x=\frac{\up}{4}$ with step sizes $h=0.1$ and $h=0.05$ using:
\begin{itemize}
    	\item[a)] The forward difference method.
	\item[b)] The backwards difference method.
	\item[c)] The central difference method.
\end{itemize}
\end{problem}

\begin{problem}
Consider the function $f(x)=\ln(x)$ for step size $h=0.1$, use the central difference method to find the value of the derivative of $f(x)$ at $x=\frac{1}{2}$.
\end{problem}

\paragraph{Multiple Differentiation}
\begin{problem}
Find the second derivative of $f(x)=x^{2}$.
\end{problem}

\begin{problem}[$\star$]
Find the first and second derivative of $f(x)=\cos(x)$ and evaluate these at $x=0$. Then compare the values of $f(x)$ near zero to the values of $f(0)+f'(0)x+\frac{1}{2}f''(0)x^{2}$ near $x=0$.
\end{problem}


\paragraph{Challenge Problems}

\begin{problem}
Using the definition of the derivative and the forward difference method find a numerical expression for the second order derivative.
\end{problem}

\begin{problem}
Write a computer programme to implement the three different numerical differentiation methods and use it to check the problems on this sheet.
\end{problem}


\section{Week 10}
\label{sec: Tutorial sheet 10}

\paragraph{Optimisation}
\begin{problem}[$\star$]
For the function $f(x)=x^{3}-3x$ find and classify the critical points.
\end{problem}

\begin{problem}
Find the critical points of the function 
\begin{equation*}
f(x)=x^{4}-3x^{2}+2.
\end{equation*}
\end{problem}

\begin{problem}[$\star$]
By considering the first and second derivatives of $f(x)=\sin(x)$ find the maxima and minima.
\end{problem}

\begin{problem}
Suppose that the population of a certain type of insect after $t$ months is given by the formula
\begin{equation*}
P(t)=3t+\sin(4t)+100
\end{equation*}
determine the minimum and maximum population in the first four months.
\end{problem}

\paragraph{Applications}

\begin{problem}
Suppose we have a model $\hat{y}=wx$ with one parameter $w$, that takes an input value $x$, gives an output of $\hat{y}$, and has a target output of $y$ with loss function
\begin{equation*}
\lambda(w)=\left(wx-y\right)^{2}.
\end{equation*}
If the input value is $x=2$ and the target output is $y=5$, find the value of the parameter $w$ which minimises the loss function. Then find the optimal output.
\end{problem}

\begin{problem}
Consider the one parameter model
\begin{equation*}
\hat{y}=e^{kx}
\end{equation*}
with input $x$, output $\hat{y}$, target output $y$, and parameter $k$. If we take the loss function to be
\begin{equation*}
\lambda(w)=\left(e^{kx}-y\right)^{2}
\end{equation*}
with input value $x=1$ and target output $y=3$, minimise the loss function and find the output that corresponds to the minimum value of $k$.
\end{problem}

\begin{problem}[$\dagger$]
In the previous two problems does it matter if we change the loss function? What do you think would happen if our model had more than one parameter in it?
\end{problem}

\newpage

%%%%%%%%%%%%%%%%%%%%%%%%%%%%%%%%%%%%%%%%%%%%%%

\chapter{Extra Proofs and Derivations}
\label{sec: proofs}
The material in this chapter is not examinable and is included for completeness. Here we will give a selection of proof and derivations for results and formulas that were used in the rest of the notes. You could consider this whole chapter to be one big mathematical deviation.

To prove the product rule from first principles we proceed as follows. Consider $f(x)=p(x)q(x)$ then
\begin{align*}
\frac{\ud f}{\ud x}		&=\lim_{h\to 0}\frac{f(x+h)-f(x)}{h}\\
				&=\lim_{h\to 0}\frac{p(x+h)q(x+h)-p(x)q(x)}{h}\\
				&=\lim_{h\to 0}\frac{p(x+h)q(x+h)-p(x+h)q(x)+p(x+h)q(x)-p(x)q(x)}{h}\\
				&=\lim_{h\to0}\frac{p(x+h)\left(q(x+h)-g(x)\right)}{h}+\lim_{h\to 0}\frac{\left(p(x+h)-p(x)\right)q(x)}{h}\\
				&=\lim_{h\to 0}p(x+h)\frac{q(x+h)-q(x)}{h}+q(x)\lim_{h\to 0}\frac{p(x+h)-p(x)}{h}\\
				&=p(x)\lim_{h\to 0}\frac{q(x+h)-q(x)}{h}+q(x)\lim_{h\to 0}\frac{p(x+h)-p(x)}{h}\\
				&=p(x)\frac{\ud q}{\ud x}+q(x)\frac{\ud p}{\ud x},
\end{align*}
which is the product rule.\\

For the quotient rule, we can either prove it from first principles, which is fiddly, or we can use the product rule on $f(x)=p(x)g(x)$ where $g(x)=1/q(x)$.  Here we will take the second approach as hopefully that will be easier to follow.  Consider $f(x)=p(x)/q(x)$ and let $g(x)=1/q(x)$ then applying the product rule means that
\begin{align*}
\frac{\ud f}{\ud x}		&=\frac{\ud }{\ud x}\left(p(x)g(x)\right)\\
				&=p(x)\frac{\ud g}{\ud x}+g(x)\frac{\ud p}{\ud x}\\
				&=p(x)\frac{\ud}{\ud x}\left(q^{-1}\right)+\frac{1}{q(x)}\frac{\ud p}{\ud x}\\
				&=-p(x)q^{-2}\frac{\ud q}{\ud x}+\frac{1}{q(x)}\frac{\ud p}{\ud x}\\
				&=\frac{1}{q^{2}}\left(q(x)\frac{\ud p}{\ud x}-p(x)\frac{\ud q}{\ud x}\right).
\end{align*}

\begin{mdiv}
Note that in the above calculation we have used the chain rule, $\left((f\circ g)(x)\right)'=f'(g(x))g'(x)$, which we did not discuss until after we had introduced the quotient rule.  Also, note that if we were mathematicians we would need to carefully think about when $p(x)/q(x)$ is differentiable, and implementing the product rule does not need that, it just gives that $p(x)(q(x))^{-1}$ is differentiable if $p(x)$ and $(q(x))^{-1}$ is.  This is why for mathematicians, they would prove the quotient rule using differentiation from first principles, or using logarithmic differentiation. The interested reader should have a look at the Proof of various derivative properties section of \citep{calcI} to see the proof in this way.
\end{mdiv}

Next, following \citep{calcI}, we prove the chain rule as follows: let $y=f(u), u=g(x)$ then we know that
\begin{equation*}
\frac{\ud u}{\ud x}=\lim_{h\to 0}\frac{u(x+h)-u(x)}{h},
\end{equation*}
and that 
\begin{equation*}
\lim_{h\to 0}\left(\frac{u(x+h)-u(x)}{h}-\frac{\ud u}{\ud x}\right)=\lim_{h\to 0}\frac{u(x+h)-u(x)}{h}-\lim_{h\to 0}\frac{\ud u}{\ud x}=\frac{\ud u}{\ud x}-\frac{\ud u}{\ud x}=0.,
\end{equation*}
Now we can define
\begin{equation*}
v(h)=\begin{cases}
\frac{u(x+h)-u(x)}{h}-\frac{\ud u}{\ud x} \quad \text{if } h\neq 0\\
0 \quad \text{if } h=0
\end{cases}
\end{equation*}
which is continuous at $h=0$ since $\lim_{h\to 0}v(h)=0=v(0)$. \dots{}

\newpage

%%%%%%%%%%%%%%%%%%%%%%%%%%%%%%%%%%%%%%%%%%%%%%


\chapter{Standard Derivatives and Integrals}
\label{sec: deriv sheet}

There are several functions that it is worth knowing the derivatives and integrals of. When we first introduced their derivatives we either derived them from first principles or used techniques like the product rule, chain rule, or integration by parts, to derive them from already known results. However, this takes time. You do not need to memorise the following list of standard derivatives and integrals, but you may find the list to be a useful resource when going through the tutorial problems or when revising for the exam. \\

Remember, you should still be able to derive these results if you need to, the list is just intended as an aid.

\section{Derivatives}

\begin{table}[ht]
\centering

\ThisAltText{A table of standard derivatives.}

\caption{Table of standard derivatives}

\vspace{2mm}

\label{table: derivatives}

\begin{tabular}{|c|c|} 
 \hline
$y=f(x)$ & $\frac{\ud y}{\ud x}=f^{\prime}(x) $\\
 \hline
$n$ constant & $0$  \\
 \hline
$x$ &$1$  \\
\hline
$x^{n}$, $n$ constant &	$nx^{n-1}$	\\
\hline
$e^{kx}$, $k$ constant &	$ke^{kx}$	\\
\hline
$\ln(x)$ &	$\frac{1}{x}$ 	\\
\hline
$\sin(k x)$ &	$k\cos(k x)$ 	\\
\hline
$\cos(kx)$ &	$-k\sin(kx)$\\
\hline
$\tan(kx)$ &	$k\sec^{2}(kx)$	\\
\hline
$\arcsin(x)$ &	$\frac{1}{\sqrt{1-x^{2}}} $	\\
\hline
$\arccos(x)$ & $-\frac{1}{\sqrt{1-x^{2}}}$ \\
\hline
$\sinh(x)$& $\cosh(x)$\\
\hline
$\cosh(x)$ & $\sinh(x)$\\
\hline
$\tanh(x)$ & $\sech^{2}(x)$\\
\hline
$f(x)g(x)$ & $f^{\prime}(x)g(x)+f(x)g^{\prime}(x)$\\
\hline
$\frac{f(x)}{g(x)}$ & $\frac{f^{\prime}(x)g(x)-f(x)g^{\prime}(x)}{(g(x))^{2}} $\\
\hline
\end{tabular}
\end{table}


\section{Integrals}

\begin{table}[ht]
\centering

\ThisAltText{A table of standard derivatives.}

\caption{Table of standard integrals}

\vspace{2mm}

\label{table: integrals}

\begin{tabular}{|c|c|} 
 \hline
$y=f(x)$ & $\int f(x) \ud x$\\
 \hline
 constant $k$ & $kx+c$  \\
 \hline
$x^{n}$ &$\frac{x^{n+1}}{n+1}+c$  \\
\hline
$\frac{1}{x}$ &	$\ln(\vert x\vert )+c$	\\
\hline
$e^{kx}$, $k$ constant &	$\frac{e^{kx}}{k} +c$	\\
\hline
$x^{-n}$ &	$\frac{x^{-n+1}}{-n+1} +c$, $n\neq 1$ 	\\
\hline
$\cos(x)$ & $\sin(x) +c$\\
\hline
$\sin(x)$ & $-\cos(x)+c$\\
\hline
$f(x)g^{\prime}(x)$& $f(x)g(x)-\int f^{\prime}(x)g(x)\ud x $\\
\hline
\end{tabular}
\end{table}

%\newpage

%%%%%%%%%%%%%%%%%%%%%%%%%%%%%%%%%%%%%%%%%%%%%%
\clearpage

\ForceHTMLPage

\begin{warpprint}
\printglossaries

\end{warpprint}

\bibliographystyle{plainnat}
\bibliography{physics}

%\begin{warpprint} % For print output ...
%\cleardoublepage % ... a common method to place index entry into TOC.
%\phantomsection
%\addcontentsline{toc}{chapter}{\indexname}
%\end{warpprint}
%\ForceHTMLPage % HTML index will be on its own page.
%\ForceHTMLTOC % HTML index will have its own toc entry.
%\printindex
\end{document}