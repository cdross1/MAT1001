\chapter{Further Reading}
\label{sec:further reading}

\epigraph{Mathematics, you see, is not a spectator sport. To understand mathematics means to be able to do mathematics. And what does it mean doing mathematics? In the first place, it means to be able to solve mathematical problems. }{\textit{How to Solve It by George P\'{o}lya}}

What we discuss in the lectures is just a guide to calculus its many uses. Depending on the modules that you take later on in your degree you will meet Calculus in different guises, particularly when you come across optimisation problems . In this section I will provide links to further resources and suggestions for further reading based on the material that we met each week. If you have any questions about the topics linked here or you want to get even more information then drop me an \href{mailto:rossc@edgehill.ac.uk}{email}.\\

It is important to remember that the lectures are their to introduce you to topics and signpost where to find out more information. If you are just attending the lectures and not doing any further reading or solving practice problems then you will struggle to pass the course. Each 20 credit module is considered 200 hours of work, only 36 hours of which are the lectures and seminars. You are expected to put in around 164 hours of self study during a 12 week module. The resources linked here will help with this self study.\\

\section{Why Calculus Extra Reading}
MIT have a course called \textbf{Calculus for Beginners and Artists} which contains some overlap with this module.  The webpage for the course is available \href{https://math.mit.edu/~djk/calculus_beginners/index.html}{here} and some sections of it, particularly \textbf{Chapter 0: Why Study Calculus?} are worth a read to complement what we will do in this module. I will occasionally suggest reading sections of these notes or sections of \citep{calcI} for a complementary explanation. \\

The book \textbf{Everyday Calculus} by Oscar Fernandez, \citep{fernandez2017everyday}, contains a nice overview of many of the places that calculus appears in our everyday lives, even when we do not realise it. If you are ever struggling for motivation with why you should care about calculus, then have a look in this book and explore some of the applications of calculus. The library has a digital copy of this book available so it is easy to dip into sections of it.

\section{Functions Extra Reading}
The first three chapters of \citep{calc_deconstructed} cover some complementary material to what we met in this module. In particular, there is a discussion of finding limits and convergence, as well as some useful material on continuous functions. This book used to be an essential part of the reading list of this module. However, it is pitched at a slightly higher level than the material that we cover here. In particular, it is aimed at students taking a second course in calculus, so is more appropriate for those on a mathematics degree. There is still some good material in the book, and it can be useful both for those of you looking for material to help cement the basics and those looking to go beyond the material in the module. Students have access to a digital copy of this book through the library so you are able to dip in and out of it as you want to.\\

If you are looking for more material on trig functions then there is a nice popular science book by Matt Parker called \textbf{Love Triangle}, \citep{parker2025love}, which talks about many of the areas that trigonometry and trig functions appear in the real world. The book also has the nice feature that the page numbers are represented by the $\sin$ of the regular page number.\\

Another nice popular science book is \textbf{Once Upon a Prime} by Sarah Hart, \citep{hart2023once}, which covers the relationship between mathematics and literature. While not directly related to functions, it is a nice read that can help you to discover the mathematics inside some of your favourite stories. It can be a really good way to remove the scare factor that is sometimes associated with mathematics.\\

If you want more on limits there is some nice content available on the maths is fun website \citep{mathsisfun}. In particular, there is a nice discussion of limits available there which you might find useful to complement the discussion of limits in this module.


\section{Differentiation Extra Reading}
The main reference for this section and many of the other sections of these lecture notes is the wonderful book \textbf{Mathematical Methods for Physics and Engineering}, \citep{riley_mathematical_2006}. There are several copies of this book and its student solution manual in the library and I recommend that you try and have a look at this at some stage.\\

There is also lots of extra material in \citep{calcI} including more details on the applications of differentiation, though not the computer science focussed appliactions.

\section{Integration Extra Reading}
This is another section where the best resource is \citep{riley_mathematical_2006} or the integration section of \citep{calcI}, which contains lots of extra examples and more details.

\section{Differential Equations Extra Reading}

\section{Numerical Methods Extra Reading}
The first port of call to see more on numerical methods at the level of this module is in \citep{lissamen2004mei}, which contains a very readable discussion of how to approximate functions, integrals, and derivatives. It also spends more time on errors and convergence than we have done so far in this module. It was the main reference when \cref{sec:numerics} was being written so it should be fairly easy to dip into if you have been reading through the lecture notes. There are physical copies of this book in the library.

\section{Calculus in Computer Science}

\section{Advanced Topics Extra Reading}

A great book to look at for more on integration, both the case of one variable that we focussed on in this module and the more general case of multiple integrals, is the book \textbf{Inside Interesting integrals} by Paul Nahin, \citep{nahin2020inside}. This is a wonderful book if you want to be equipped to deal with any integral you come across. It goes far beyond the level of this module, and some of the examples are down right painful to compute, However, if you enjoy getting your hands dirty and solving every integral that you can get your hands on then this is the book for you.  I am recommending it here rather than in the integration section purely because most of the book is so far above the level of this module that it is only useful if you are already comfortable with the basics of integration. Personally I love this book and think that it is incredibly useful. 




