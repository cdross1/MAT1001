\chapter{Tutorial Sheets}
\label{sec: tutorial sheets}

Here we collect all of the tutorial problems for the module. They are split into different weeks depending on the topic they relate to and when they were given out.\\

Many of these questions are taken from or adapted from the recommended  books for the module or from some of the linked resources. These problems are to be attempted in the tutorial sessions and are there to help you familiarise yourself with the material that we have covered in the lectures.\\

Problems marked with a star, $(\star)$ are particularly worth attempting. Problems marked with a dagger, $(\dagger)$, are more challenging and often go beyond what we directly discussed in the lectures.\\

The challenge problem sections contain extra problems. Some of them are just there for extra practice, but others are significantly more difficult than what you need to be able to solve to pass the module. If you are finding the content too easy then have a go at the challenge problems. Sometime the challenge problems from one week will be quite similar to the ordinary problems of the next week. as the problems will become more accessible the more material that we cover.

\section{Week 1}
\label{sec: Tutorial sheet 1}
\paragraph{Functions}

\begin{problem}[$\star$]
Find the roots of the polynomial
\begin{equation*}
g(x)=x^2-2x-12
\end{equation*}
and plot the function.
\end{problem}

\begin{problem}
Find the roots of the polynomial
\begin{equation*}
g(x)=x^3+x^2-x-1
\end{equation*}
and plot the function.
\end{problem}

\begin{problem}[$\star$]
Consider the function
\begin{equation*}
g(x)=x^2+2x+2,
\end{equation*}
produce a plot of the function by calculating its value at a selection of points. What do you notice as $x$ gets very large? What happens for $x=0$?
\end{problem}

\begin{problem}[$\star$]
Consider the function
\begin{equation*}
g(x)=\frac{1}{x-4},
\end{equation*}
and plot the function.what happens as $x$ gets large? What happens as $x$ approaches $4$?\\

Plot the function and comment on its behaviour.
\end{problem}

\begin{problem}
Draw a schematic of a one-to-one function between the sets
\begin{equation*}
\begin{split}
&\left\{1,2,3,4,5\right\}\\
&\left\{a,b,c,d,e,f,g\right\}.
\end{split}
\end{equation*}
Is there only one way to do this?
\end{problem}




\begin{problem}[$\star$]
Given the function
\begin{equation*}
f(x)=x^3-2x^2-x+2
\end{equation*}
find:
\begin{itemize}
    \item $f(0)$,
    \item $f(1)$,
    \item $f(-1)$,
    \item $f(2)$,
    \item $f(-2)$,
    \item $f(t)$,
    \item $f(x-1)$.
\end{itemize}
\end{problem}

\begin{problem}
Consider the function
\begin{equation*}
h(x)=\frac{x}{\sqrt{x^{2}-9}}.
\end{equation*}
Find the points where the denominator vanishes, then plot the function avoiding these points. What happens to the plot as the function approaches these points?
\end{problem}


\begin{problem}
Given the functions
\begin{align*}
f(x)&=x^2-x+1,\\
g(x)&=2-x,
\end{align*}
find:
\begin{itemize}
\item $\left(f\circ g\right)(2)$,
\item $\left(g\circ f\right)(2)$,
\item $\left(f\circ g\right)(x)$,
\item $\left(g\circ f\right)(x)$.
\end{itemize}
\end{problem}


\begin{problem}[$\star$]
Given the functions
\begin{align*}
f(x)&=3x-2\\
g(x)&=\frac{x}{3}+\frac{2}{3},
\end{align*}
find:
\begin{itemize}
\item $\left(f\circ g\right)(x)$,
\item $\left(g\circ f\right)(x)$,
\item What is the relationship between $f(x)$ and $g(x)$?
\end{itemize}
\end{problem}


\begin{problem}[$\dagger$]
Given the function
\begin{equation*}
h(x)=\frac{x+4}{2x-5},
\end{equation*}
identify when it has an inverse and calculate the inverse.
\end{problem}

\paragraph{Challenge Problems}

\begin{problem}
Consider the function 
\begin{equation*}
f(x)=\frac{x^{2}-x-12}{x-1}.
\end{equation*}
Identify the points where the numerator and denominator vanish. Plot the function and explain what happens to the function near these points.
\end{problem}

\begin{problem}
Given the function 
\begin{equation*}
f(x)=2x-3,
\end{equation*}
find the inverse function $f^{-1}(x)$.
\end{problem}


\begin{problem}[$\dagger\dagger$]
Build a schematic of a bijection between the natural numbers
\begin{equation*}
\N=\{1,2,3,4,5,\dots\},
\end{equation*}
and the integers
\begin{equation*}
\Z=\{0,1,-1,2,-2,3,-3,\dots\}.
\end{equation*}

Are there more integers than natural numbers?\\

Could you do the same for the real numbers $\R$? If you find this interesting you may want to explore the work of Cantor.
\end{problem}




\section{Week 2}
\label{sec: Tutorial sheet 2}

\paragraph{Polynomials}
\begin{problem}[$\star$]
For the polynomial equation
\begin{equation*}
x^{3}-3x+2=0,
\end{equation*}
express it as a product of its factors.\\

Hint: this means write it as 
\begin{equation*}
\left(x-a\right)\left(x-b\right)\left(x-c\right),
\end{equation*}
where $a,b,c$ are the roots of the polynomial.
\end{problem}

\begin{problem}
Find the roots of the following polynomial equation
\begin{equation*}
x^{4}-4x^{3}+6x^{2}-4x+1=0.
\end{equation*}
\end{problem}

\paragraph{Trig functions}
Remember to work in radians for any problems related to trigonometry.

\begin{problem}
Find the solutions to 
\begin{equation*}
\sqrt{2}\cos x =1.
\end{equation*}
\end{problem}

\begin{problem}
Identify any solutions to the equation
\begin{equation*}
\sin(2x)=-2.
\end{equation*}
\end{problem}

\begin{problem}[$\star$]
Solve
\begin{equation*}
2x\sin x = x.
\end{equation*}
\end{problem}

\paragraph{Exponentials and Logarithms}
\begin{problem}[$\star$]
Solve the equation
\begin{equation*}
x=xe^{4x}.
\end{equation*}
\end{problem}

\begin{problem}[$\star$]
Solve
\begin{equation*}
2\ln x -\ln(x+2)=1.
\end{equation*}
\end{problem}

\paragraph{Limits and Asymptotes}
\begin{problem}[$\star$]
For the function
\begin{equation*}
f(x)=\frac{x^{2}+4x+12}{x^{2}-2x}
\end{equation*}
evaluate the limit
\begin{equation*}
\lim_{x\to 2}f(x).
\end{equation*}
\end{problem}

\begin{problem}[$\star$]
For the function
\begin{equation*}
g(x)=\frac{2x^{4}-x^{2}+8x}{7-5x^{4}}
\end{equation*}
evaluate the limits
\begin{align*}
&\lim_{x\to -\infty}g(x),\\
&\lim_{x\to \infty}g(x).
\end{align*}
\end{problem}

\begin{problem}[$\dagger$]
For the function
\begin{equation*}
h(x)=\frac{6e^{4x}-e^{-2x}}{8e^{4x}-2e^{2x}+3e^{-x}}
\end{equation*}
evaluate the limit
\begin{align*}
&\lim_{x\to \infty}h(x).
\end{align*}
\end{problem}

\begin{problem}
Using an appropriate plot evaluate the limits
\begin{align*}
&\lim_{x\to \frac{\pi}{2}}\tan x,\\
&\lim_{x\to 0}\tan x.
\end{align*}
\end{problem}

\begin{problem}[$\star$]
Determine where the function
\begin{equation*}
g(x)=\frac{x^{3}+x^{2}-x-1}{x^{3}-2x^{2}-4x+8}
\end{equation*}
fails to be continuous. What do these points correspond to on a plot of $g(x)$?
\end{problem}

\paragraph{Challenge Problems}
\begin{problem}
By making a table of values of $f(x), x$ estimate the limit
\begin{equation*}
\lim_{x\to 2}f(x)
\end{equation*}
from above and below for the function
\begin{equation*}
f(x)=\frac{x^{2}+4x-12}{x^{2}-2x}.
\end{equation*}
\end{problem}

\begin{problem}
Use the definition of continuity to determine in the function
\begin{equation*}
g(x)=\frac{4x+10}{x^{2}-2x-15}
\end{equation*}
is continuous and if not find the points where it has discontinuities.
\end{problem}


\begin{problem}[$\dagger$]
For the function $g(x)$ in the previous problem look at the definition of differentiability and work out where $g(x)$ fails to be differentiable.
\end{problem}

\paragraph{Hyperbolic Trig functions}
\begin{problem}[$\dagger$]
Show that
\begin{equation*}
\sinh(x+y)=\sinh x \cosh y+\cosh x \sinh y.
\end{equation*}
\end{problem}

\begin{problem}[$\dagger$]
Following some of the examples in \cref{sec: hyperbolic functions} show that 
\begin{equation*}
\cosh^{-1}x=\ln\left(x\pm\sqrt{x^{2}+1}\right).
\end{equation*}
\end{problem}

\section{Week 3}
\label{sec: Tutorial sheet 3}
\paragraph{Differentiation}

\begin{problem}[$\star$]
Find the derivative of 
\begin{equation*}
f(x)=7x^{2}
\end{equation*}
\begin{itemize}
\item[a)] using first principles,
\item[b)] using the rule for differentiating monomials.
\end{itemize}
\end{problem}

\begin{problem}
Find the derivative of 
\begin{equation*}
f(x)=3x^{3}-8x.
\end{equation*}
\end{problem}

\begin{problem}[$\star$]
Find the derivative of 
\begin{equation*}
f(x)=(x+4)(x+2).
\end{equation*}
\end{problem}

\begin{problem}[$\star$]
Find the derivative of 
\begin{equation*}
g(x)=6x^{4}-x^{3}.
\end{equation*}
\end{problem}

\begin{problem}
Find the derivative of 
\begin{equation*}
y=\frac{x^{9}}{3}+\frac{x^{4}}{4}.
\end{equation*}
\end{problem}

\begin{problem}[$\star$]
Calculate the derivative of 
\begin{equation*}
y=\left(x^{4}-3\right)^{2}.
\end{equation*}
\end{problem}

\begin{problem}[$\dagger$]
Consider the function
\begin{equation*}
h(x)=\frac{x}{\sqrt{x^{2}-9}},
\end{equation*}
when can we calculate its derivative? When it exists calculate the derivative.
\end{problem}

\begin{problem}
Identify when we can differentiate the function
\begin{equation*}
y=\frac{3x^{3}\left(x^{2}-4x\right)}{x}
\end{equation*}
and calculate its derivative.
\end{problem}

\begin{problem}[$\star$]
Identify when we can differentiate the function
\begin{equation*}
f(x)=9x^{4}-\frac{4}{x^{3}}
\end{equation*}
and calculate its derivative.
\end{problem}
\begin{problem}[$\star$]
Differentiate the function
\begin{equation*}
f(x)=\sin(2x).
\end{equation*}
\end{problem}

\begin{problem}
Differentiate the function
\begin{equation*}
f(x)=\cos(3x).
\end{equation*}
\end{problem}

\begin{problem}[$\star$]
Differentiate the following expressions
\begin{itemize}
    	\item[a)] $\cos(x)-\sin(x)$
	\item[b)] $3\tan(x)-\cos(2x)$
	\item[c)] $4\sin(2x)+2\cos(5x)+5$
	\item[d)] $\ln(x+3)$
	\item[e)] $x\ln(x^{2})$
\end{itemize}
\end{problem}

\paragraph{Challenge Problems}

\begin{problem}
Find the gradient of the following curves at the indicated points
\begin{itemize}
    	\item[a)] $y=5\sin(2x)$ at the point $(x,y)=(\up/2,0)$
	\item[b)]$y=\tan(x/2)$ at the point $(x,y)=(\up/2,1)$
\end{itemize}
\end{problem}

\begin{problem}
Find the stationary points, the points where the derivative vanishes, for the following functions
\begin{itemize}
    	\item[a)] $y=\sin(2x)+\cos(x)$, with $0\leq x\leq \up$
	\item[b)]$f(x)=\log(\sin(x))$ for $0\leq x\leq \up$
\end{itemize}
\end{problem}

\begin{problem}[$\dagger$]
Calculate the derivative of 
\begin{equation*}
y=x^{x}.
\end{equation*}
Hint: You may want to take the logarithm of the function.
\end{problem}

\section{Week 4}
\label{sec: Tutorial sheet 4}

\paragraph{Trig, Exp, and Log}

\begin{problem}[$\star$]
Solve the following, leaving your answer in terms of logs:
\begin{itemize}
    	\item[a)] $2^{x+4}=6$
	\item[b)] $3^{2x-1}=17$
	\item[c)] $2^{1-4x}=5$
	\item[d)] $5^{3x+4}=31$
\end{itemize}
\end{problem}


\begin{problem}
Solve $0.6=2^{-x}$.
\end{problem}

\begin{problem}
The point $(K,5)$ lies on the curve $y=2^{x}$. Find $K$.
\end{problem}

\begin{problem}[$\star$]
Simplify the expressions:
\begin{itemize}
    	\item[a)] $\ln e^{\sin(x)}$
	\item[b)] $e^{2\ln(1+x)}$
	\item[c)] $e^{-\ln(5-x)}$
\end{itemize}
\end{problem}

\begin{problem}
Solve $e^{2x}-5e^{x}+4=0$.
\end{problem}

\begin{problem}[$\star$]
Evaluate the following limits:
\begin{itemize}
    	\item[a)] $\lim_{x\to \up/2}\left(\frac{x}{1+\sin(x)}\right)$
	\item[b)] $\lim_{x\to 1}\left(\frac{\ln(x)}{1+\ln(x)}\right)$
	\item[c)] $\lim_{x\to \infty}\left(\frac{3+2x}{2+3x}\right)$
	\item[d)] $\lim_{x\to\up^{-}/2}\left(\frac{x}{\tan{x}}\right)$
	\item[e)] $\lim_{x\to\up^{+}/2}\left(\frac{x}{\tan{x}}\right)$
\end{itemize}
\end{problem}


\section{Week 5}
\label{sec: Tutorial sheet 5}

\paragraph{Product, Quotient, and Chain Rules}
\begin{problem}[$\star$]
Given the function 
\begin{equation*}
h(x)=\frac{x+4}{2x-5}
\end{equation*}
identify when you can differentiate it and find its derivative.
\end{problem}

\begin{problem}[$\star$]
Calculate the derivative of 
\begin{equation*}
y=x^{5}\sin(x)
\end{equation*}
using the product rule.
\end{problem}

\begin{problem}
Calculate the derivative of 
\begin{equation*}
f(x)=3e^{x}\cos(x)
\end{equation*}
using the product rule.
\end{problem}

\begin{problem}[$\star$]
Calculate the derivative of 
\begin{equation*}
f(x)=\left(10x-3\right)^{4}
\end{equation*}
using the chain rule.
\end{problem}

\begin{problem}
Use the quotient rule to calculate the derivative of 
\begin{equation*}
y=\tan(x)=\frac{\sin(x)}{\cos(x)}.
\end{equation*}
Does it match your expectation?
\end{problem}

\begin{problem}[$\star$]
Use the quotient rule to calculate the derivative of 
\begin{equation*}
f(x)=\frac{3x^{3}+8x^{2}+2}{2x+1}.
\end{equation*}
\end{problem}

\begin{problem}
Calculate the derivative of 
\begin{equation*}
f(x)=\ln(x^{2}+1).
\end{equation*}
\end{problem}

\paragraph{Antiderivatives}


\begin{problem}[$\star$]
By identifying a function whose derivative is $1/x$ solve the integral
\begin{equation*}
I=\int \frac{1}{x}\ud x.
\end{equation*}
\end{problem}

\begin{problem}[$\star$]
Find the antiderivative of $e^{x}$
\end{problem}

\begin{problem}[$\star$]
Find the antiderivatives of the following functions:
\begin{itemize}
    	\item[a)] $\cos(x)$
	\item[b)] $x^{2}+2x$
	\item[c)] $\sqrt{X^{2}+2}$
	\item[d)] $x^{2}+1/x^{2}$
\end{itemize}
\end{problem}

\paragraph{Integration and Area}

\begin{problem}
Calculate the integral
\begin{equation*}
I=\int \left(5x^{2}-8x+5\right)\ud x.
\end{equation*}
\end{problem}

\begin{problem}[$\star$]
Calculate the integrals:
\begin{itemize}
    	\item[a)] $I=\int\left(x^{\frac{3}{2}}+2x+3\right)\ud x$
	\item[b)] $I=\int\left(\frac{8}{x}-\frac{5}{x^{2}}+\frac{6}{x^{3}}\right)\ud x$
	\item[c)] $I=\int\left(4e^{-7x}\right)\ud x$
\end{itemize}
\end{problem}

\begin{problem}[$\star$]
Calculate the integrals:
\begin{itemize}
    	\item[a)] $I=\int\frac{x^{3}+4}{x^{2}}\ud x$
	\item[b)] $I=\int\left(12x^{\frac{3}{4}}-9x^{\frac{5}{3}}\right)\ud x$
	\item[c)] $I=\int7\sin(x)\ud x$
	\item[d)] $I=\int 5\cos(x)\ud x$
\end{itemize}
\end{problem}

\begin{problem}[$\star$]
Calculate the following definite integrals:
\begin{itemize}
    	\item[a)] $I=\int^{4}_{1} \left(5x^{2}-8x+5\right)\ud x$
	\item[b)]$I=\int^{9}_{1}\left(x^{\frac{3}{2}}+2x+3\right)\ud x$
	\item[c)] $I=\int^{\frac{\up}{2}}_{0}\left(\frac{8}{x}-\frac{5}{x^{2}}+\frac{6}{x^{3}}\right)\ud x$
	\item[d)] $I=\int_{\frac{\up}{2}}^{\frac{3\up}{2}}\left(4e^{-7x}\right)\ud x$
\end{itemize}
\end{problem}

\paragraph{Challenge Problems}

\begin{problem}
Evaluate the integral
\begin{equation*}
I=\int \frac{3x}{4x-5}\ud x.
\end{equation*}
\end{problem}

\begin{problem}
Use integration by parts to evaluate
\begin{equation*}
I=\int xe^{-2x}\ud x.
\end{equation*}
\end{problem}

\begin{problem}[$\dagger$]
Look at the definition of differentiation from first principles and prove the product rule from first principles.
\end{problem}

\begin{problem}[$\star$]
Consider the function
\begin{equation*}
f(x)=x^{2}+2x
\end{equation*}
calculate its derivative. Is the derivative that you find a differentiable function? If it is, calculate its derivative, what do you get?\\

Do the same for $f(x)=\sin(x)$, what do you notice here?
\end{problem}




\section{Week 8}
\label{sec: Tutorial sheet 8}

\paragraph{Numerical Integration}


\begin{problem}[$\star$]
Consider the integral
\begin{equation*}
I=\int_{0}^{2}3^{x}\ud x
\end{equation*}
Evaluate this using:
\begin{itemize}
    	\item[a)] The midpoint rule with $4$ strips.
	\item[b)]The Trapezium rule with $4$ strips.
	\item[c)] Simpson's rule with $4$ strips.
\end{itemize}
\end{problem}

\begin{problem}[$\star$]
Consider the integral
\begin{equation*}
I=\int_{-1}^{1}\sqrt{x^{3}+1}\ud x
\end{equation*}
Evaluate this for $3$ strips using:
\begin{itemize}
    	\item[a)] The midpoint rule.
	\item[b)]The Trapezium rule.
	\item[c)] Simpson's rule.
\end{itemize}
\end{problem}

\begin{problem}
Consider the integral
\begin{equation*}
I=\int_{0}^{1}\frac{1}{x^{2}+1}\ud x
\end{equation*}
Evaluate this for $9$ strips using:
\begin{itemize}
    	\item[a)] The midpoint rule.
	\item[b)]The Trapezium rule.
	\item[c)] Simpson's rule.
\end{itemize}
\end{problem}

\begin{problem}[$\star$]
Consider the integral
\begin{equation*}
I=\int_{0}^{1}\sqrt{x(2x-1)}\ud x
\end{equation*}
Evaluate this for $4$ strips using:
\begin{itemize}
    	\item[a)] The midpoint rule.
	\item[b)]The Trapezium rule.
	\item[c)] Simpson's rule.
\end{itemize}
\end{problem}

\begin{problem}
Consider the integral
\begin{equation*}
I=\int_{1}^{2}\ln\left(1+\sqrt{x}\right)\ud x
\end{equation*}
Evaluate this for $5$ strips using:
\begin{itemize}
    	\item[a)] The midpoint rule.
	\item[b)]The Trapezium rule.
	\item[c)] Simpson's rule.
\end{itemize}
\end{problem}

\paragraph{Challenge Problems}

\begin{problem}
Use Simpson's rule with $6$ strips to calculate
\begin{equation*}
I=\int_{0}^{2}\sqrt{1+\sin(x)+\cos(x)}\ud x
\end{equation*}
to $6$ decimal places.
\end{problem}

\begin{problem}
Write a computer program to implement Simpson's rule and use it to check the problems on this sheet.
\end{problem}

\section{Week 9}
\label{sec: Tutorial sheet 9}

\paragraph{Numerical Differentiation}

\begin{problem}
Given $f(x)=\cos(x)$,
\begin{itemize}
    	\item[a)] find $f'\left(\frac{\up}{3}\right)$ using the forward difference method with $h=0.1,0.01, 0.001, 0.0001$.
	\item[b)]Now find $f'\left(\frac{\up}{3}\right)$ using the backward difference method with $h=0.1,0.01, 0.001, 0.0001$.
	\item[c)] Calculate the exact value of the derivative at $x=\frac{\up}{3}$ and compare it to the approximations.
\end{itemize}
\end{problem}

\begin{problem}[$\star$]
Use the forward difference method with $h=0.05$ to approximate the derivative of $f(x)=4e^{2x}$ at $x=1$ and compare this to the exact result.
\end{problem}

\begin{problem}[$\star$]
Given $f(x)=\cos(x)$ find the value of the derivative at $x=\frac{\up}{4}$ with step sizes $h=0.1$ and $h=0.05$ using:
\begin{itemize}
    	\item[a)] The forward difference method.
	\item[b)] The backwards difference method.
	\item[c)] The central difference method.
\end{itemize}
\end{problem}

\begin{problem}
Consider the function $f(x)=\ln(x)$ for step size $h=0.1$, use the central difference method to find the value of the derivative of $f(x)$ at $x=\frac{1}{2}$.
\end{problem}

\paragraph{Multiple Differentiation}
\begin{problem}
Find the second derivative of $f(x)=x^{2}$.
\end{problem}

\begin{problem}[$\star$]
Find the first and second derivative of $f(x)=\cos(x)$ and evaluate these at $x=0$. Then compare the values of $f(x)$ near zero to the values of $f(0)+f'(0)x+\frac{1}{2}f''(0)x^{2}$ near $x=0$.
\end{problem}


\paragraph{Challenge Problems}

\begin{problem}
Using the definition of the derivative and the forward difference method find a numerical expression for the second order derivative.
\end{problem}

\begin{problem}
Write a computer programme to implement the three different numerical differentiation methods and use it to check the problems on this sheet.
\end{problem}


\section{Week 10}
\label{sec: Tutorial sheet 10}

\paragraph{Optimisation}
\begin{problem}[$\star$]
For the function $f(x)=x^{3}-3x$ find and classify the critical points.
\end{problem}

\begin{problem}
Find the critical points of the function 
\begin{equation*}
f(x)=x^{4}-3x^{2}+2.
\end{equation*}
\end{problem}

\begin{problem}[$\star$]
By considering the first and second derivatives of $f(x)=\sin(x)$ find the maxima and minima.
\end{problem}

\begin{problem}
Suppose that the population of a certain type of insect after $t$ months is given by the formula
\begin{equation*}
P(t)=3t+\sin(4t)+100
\end{equation*}
determine the minimum and maximum population in the first four months.
\end{problem}

\paragraph{Applications}

\begin{problem}
Suppose we have a model $\hat{y}=wx$ with one parameter $w$, that takes an input value $x$, gives an output of $\hat{y}$, and has a target output of $y$ with loss function
\begin{equation*}
\lambda(w)=\left(wx-y\right)^{2}.
\end{equation*}
If the input value is $x=2$ and the target output is $y=5$, find the value of the parameter $w$ which minimises the loss function. Then find the optimal output.
\end{problem}

\begin{problem}
Consider the one parameter model
\begin{equation*}
\hat{y}=e^{kx}
\end{equation*}
with input $x$, output $\hat{y}$, target output $y$, and parameter $k$. If we take the loss function to be
\begin{equation*}
\lambda(w)=\left(e^{kx}-y\right)^{2}
\end{equation*}
with input value $x=1$ and target output $y=3$, minimise the loss function and find the output that corresponds to the minimum value of $k$.
\end{problem}

\begin{problem}[$\dagger$]
In the previous two problems does it matter if we change the loss function? What do you think would happen if our model had more than one parameter in it?
\end{problem}